\section{Struttura del file di configurazione}
%--------1---------2---------3---------4---------5---------6---------7---------8
Per gestire la configurazione dei tracciati il plugin definisce la classe
\textsl{MasterBook}, vedi cod.~\ref{lst:MasterBook}, è divisa in due componenti,
la prima \texttt{defaults} è semplicemente il java-bean \textsl{FieldDefault}
(vedi~\ref{lst:FieldDefault}) messo a disposizione dalla 
\textsl{Service Provider Interface} per fornire i valori di default dei 
parametri ``poco variabili'' delle classi e dei campi.

\ifesource
\begin{figure*}[!htb]
\begin{lstlisting}[language=java, caption=classe di configurazione MasterBook, 
label=lst:MasterBook]
@Data
public class MasterBook {
    private (*\hyperref[lst:FieldDefault]{FieldDefault}*) defaults = new FieldDefault();
    private List<(*\hyperref[lst:ClassPackage]{ClassPackage}*)> packages = new ArrayList<>();
}
\end{lstlisting}
\end{figure*}
\else
\begin{elisting}[!htb]
\begin{javacode}
@Data
public class MasterBook {
    private |\hyperref[lst:FieldDefault]{FieldDefault}| defaults = new FieldDefault();
    private List<|\hyperref[lst:ClassPackage]{ClassPackage}|> packages = new ArrayList<>();
}
\end{javacode}
\caption{classe di configurazione MasterBook}
\label{lst:MasterBook}
\end{elisting}
\fi

%--------1---------2---------3---------4---------5---------6---------7---------8
Per semplificare la valorizzazione del file di configurazione yaml, viene usata
un funzionalità delle librerie yaml, che permette di definire nomi abbreviati o
alternativi dei parametri e dei valori dei campi di tipo enum.
I dettagli della componenti del campo \texttt{defaults} sarà mostrato insieme al
campo a cui fornisce il default del valore dei parametri.

%--------1---------2---------3---------4---------5---------6---------7---------8

\ifesource
\begin{figure*}[!htb]
\begin{lstlisting}[language=java, caption=classe di configurazione ClassPackage, 
label=lst:ClassPackage]
@Data
public class ClassPackage {
    private String name;     // package name
    private List<(*\hyperref[lst:TraitModel]{TraitModel}*)> interfaces = new ArrayList<>();
    private List<(*\hyperref[lst:ClassModel]{ClassModel}*)> classes = new ArrayList<>();
}
\end{lstlisting}
\end{figure*}
\else
\begin{elisting}[!htb]
\begin{javacode}
@Data
public class ClassPackage {
    private String name;     // package name
    private List<|\hyperref[lst:TraitModel]{TraitModel}|> interfaces = new ArrayList<>();
    private List<|\hyperref[lst:ClassModel]{ClassModel}|> classes = new ArrayList<>();
}
\end{javacode}
\caption{classe di configurazione ClassPackage}
\label{lst:ClassPackage}
\end{elisting}
\fi
%--------1---------2---------3---------4---------5---------6---------7---------8
La seconda componente di \textsl{MasterBook}, \texttt{packages}, è una lista di
\textsl{ClassPackage} (\ref{lst:ClassPackage}), cioè di package all'\,in\-ter\-no
dei quali vengono definiti un elenco di interfacce e classi.
Espandendo un esempio di questo oggetto in formato yaml (con il default 
relativo) abbiamo:

\ifesource
\begin{figure*}[!htb]
\begin{lstlisting}[language=yaml, 
caption={configurazione, area packages / interfaces / classes}, 
label=lst:pakg-conf]
defaults:
  cls:
    onOverflow: Trunc   # :ovf: Error, Trunc
    onUnderflow: Pad    # :unf: Error, Pad
    doc: true
packages:
  - name: com.example.test  # package name
    interfaces:
      - &IFoo         (*\color{purple}{\# interface reference}*)
        name: IFoo    # interface name
        length: 12    # :len: interface length
        fields:
          - ...
    classes:
      - name: Foo           # class name
        length: 10          # :len: class length
        onOverflow: Trunc   # :ovf: Error, Trunc
        onUnderflow: Pad    # :unf: Error, Pad
        fields:
          - ...
\end{lstlisting}
\end{figure*}
\else
\begin{elisting}[!htb]
\begin{yamlcode}
defaults:
  cls:
    onOverflow: Trunc   # :ovf: Error, Trunc
    onUnderflow: Pad    # :unf: Error, Pad
    doc: true
packages:
  - name: com.example.test  # package name
    interfaces:
      - &IFoo         # interface reference
        name: IFoo    # interface name
        length: 12    # :len: interface length
        fields:
          - ...
    classes:
      - name: Foo           # class name
        length: 10          # :len: class length
        onOverflow: Trunc   # :ovf: Error, Trunc
        onUnderflow: Pad    # :unf: Error, Pad
        fields:
          - ...
\end{yamlcode}
\caption{configurazione, area packages / interfaces / classes}
\label{lst:pakg-conf}
\end{elisting}
\fi
%--------1---------2---------3---------4---------5---------6---------7---------8
Nei commenti vengono mostrati gli eventuali nomi alternativi dei campi e 
l'\,elenco dei valori \textsl{enum} permessi.
Se non vengono usate interfacce, il nodo \texttt{interfaces} può essere omesso.
Sia per le classi che le interfacce il nome e la lunghezza del tracciato da 
associare devono essere impostate dall'\,utente, nella definizione della classe
può anche essere impostato il comportamento nel caso che venga fornita in fase
di de-serializzazione una struttura con una dimensione maggiore o minore di 
quella attesa.

\begin{table}[!htb]
\centering
\begin{tabular}{|c|>{\tt}l|>{\tt}c|>{\tt}c|c|l|}
\cline{2-6} \multicolumn{1}{c|}{}
&\multicolumn{5}{c|}{interfaces \quad \hyperref[lst:TraitModel]{TraitModel}}\\
\cline{2-6} \multicolumn{1}{c|}{}
&\multicolumn{1}{c|}{attributo} & \multicolumn{1}{c|}{alt} 
	& \multicolumn{1}{c|}{tipo} & \multicolumn{1}{c|}{O}
	& \multicolumn{1}{c|}{default} \\
\cline{2-6} \multicolumn{1}{c|}{}
&name     &     & String  & \ding{52} & \\
\cline{2-6} \multicolumn{1}{c|}{}
&length     & len & int     & \ding{52} & \\
\hline
\hyperref[lst:ClsDefault]{ClsDefault}
&doc        &     & boolean & & \texttt{\$\{defaults.cls.doc:true\}}\\
\hline \multicolumn{1}{c|}{} 
&fields     &     & array & \ding{52} & \\
\cline{2-6}
\end{tabular}
%--------1---------2---------3---------4---------5---------6---------7---------8
\caption{Attributi impostabili per la definizione di una interfaccia} 
\label{tab:attr.trait}
\end{table}


%--------1---------2---------3---------4---------5---------6---------7---------8
Anche se tutti i tipi di campo hanno una posizione di inizio e una lunghezza,
il dettaglio dei parametri di configurazione varia da campo a campo ed è 
necessario introdurre i parametri di configurazione campo per campo.

\begin{table}[!htb]
\centering
\begin{tabular}{|c|>{\tt}l|>{\tt}c|>{\tt}c|c|l|}
\cline{2-6} \multicolumn{1}{c|}{}
&\multicolumn{5}{c|}{classes \quad \hyperref[lst:ClassModel]{ClassModel}}\\
\cline{2-6} \multicolumn{1}{c|}{}
&\multicolumn{1}{c|}{attributo} & \multicolumn{1}{c|}{alt} 
	& \multicolumn{1}{c|}{tipo} & \multicolumn{1}{c|}{O}
	& \multicolumn{1}{c|}{default} \\
\cline{2-6} \multicolumn{1}{c|}{}
&name     &     & String  & \ding{52} & \\
\cline{2-6} \multicolumn{1}{c|}{}
&length     & len & int     & \ding{52} & \\
\hline
\parbox[t]{15mm}{\multirow{3}{*}{\rotatebox[origin=c]{45}{\hyperref[lst:ClsDefault]{ClsDefault}}}}
%\multirow{3}{*}{\hyperref[lst:ClsDefault]{ClsDefault}}
&onOverflow & ovf & \hyperref[lst:LoadOverflowAction]{enum} & & \texttt{\$\{defaults.cls.onOverflow:Trunc\}}\\
\cline{2-6} 
&onUnderlow & unf & \hyperref[lst:LoadUnderflowAction]{enum} & & \texttt{\$\{defaults.cls.onUnderflow:Pad\}}\\
\cline{2-6} 
&doc        &     & boolean & & \texttt{\$\{defaults.cls.doc:true\}}\\
\hline \multicolumn{1}{c|}{}
&fields     &     & array & \ding{52} & \\
\cline{2-6}
\end{tabular}
%--------1---------2---------3---------4---------5---------6---------7---------8
\caption{Attributi impostabili per la definizione di una classe} 
\label{tab:attr.class}
\end{table}

%--------1---------2---------3---------4---------5---------6---------7---------8
Per indicare esplicitamente il tipo di campo utilizzato vengono introdotti dei
\textsl{tag} da associare ad ogni campo, nella tabella~\ref{tab:tag.class} sono
mostrati i \textsl{tag} associati a ogni tipo di campo.

\begin{table}[!htb]
\centering
\begin{tabular}{|>{\tt}l|>{\tt}l|l|}
\hline
\multicolumn{3}{|c|}{Tag definizione campo}\\
\hline
\multicolumn{1}{|c|}{tag} & \multicolumn{1}{c|}{classe} 
	& \multicolumn{1}{c|}{note} \\
\hline
\hline
\hyperref[sub:yaml.abc]{!Abc} & \hyperref[lst:AbcModel]{AbcModel} & campo alfanumerico \\
\hline
\hyperref[sub:yaml.num]{!Num} & \hyperref[lst:NumModel]{NumModel} & campo numerico \\
\hline
\hyperref[sub:yaml.cus]{!Cus} & \hyperref[lst:CusModel]{CusModel} & campo custom \\
\hline
\hyperref[sub:yaml.nux]{!Nux} & \hyperref[lst:NuxModel]{NuxModel} & campo numerico nullabile \\
\hline
\hyperref[sub:yaml.dom]{!Dom} & \hyperref[lst:DomModel]{DomModel} & campo dominio \\
\hline
\hyperref[sub:yaml.fil]{!Fil} & \hyperref[lst:FilModel]{FilModel} & campo filler \\
\hline
\hyperref[sub:yaml.val]{!Val} & \hyperref[lst:ValModel]{ValModel} & campo costante \\
\hline
\hyperref[sub:yaml.grp]{!Grp} & \hyperref[lst:GrpModel]{GrpModel} & gruppo di campi \\
\hline
\hyperref[sub:yaml.occ]{!Occ} & \hyperref[lst:OccModel]{OccModel} & gruppo di campi ripetuti \\
\hline
\hyperref[sub:yaml.emb]{!Emb} & \hyperref[lst:EmbModel]{EmbModel} & campi inclusi da interfaccia \\
\hline
\hyperref[sub:yaml.igrp]{!GRP} & \hyperref[lst:GrpTraitModel]{GrpTraitModel} & gruppo di campi inclusi da interfaccia \\
\hline
\hyperref[sub:yaml.iocc]{!OCC} & \hyperref[lst:OccTraitModel]{OccTraitModel} & gruppo di campi ripetuti inclusi da interfaccia\\
\hline
\end{tabular}
%--------1---------2---------3---------4---------5---------6---------7---------8
\caption{Tag yaml per la identificazione del campo} 
\label{tab:tag.class}
\end{table}

\begin{quote}
%--------1---------2---------3---------4---------5---------6---------7---------8
Riguardo all'\,offeset dei campi, va segnalato, che alcune caratteristiche non 
dipendono dal \textsl{Service}, ma dal \textsl{Service Provider}:
l'\,offset minimo può essere zero o uno, l'\,impostazione dell'\,offset può
essere obbligatoria, o facoltativa (l'\,offset può essere calcolato 
automaticamente usando l'\,offset e la lunghezza del campo precedente), 
o non permessa.

%--------1---------2---------3---------4---------5---------6---------7---------8
Il \textsl{Service Provider} descritto nella sezione~\ref{sec:java.addon}
utilizza un offset minimo 1 e l'\,impostazione dell'\,offset è facoltativa.
Se si omette l'\,offset in un campo definito con \textsl{override}, si assume 
che  il campo ridefinisce il campo che lo precede nella definizione della 
struttura.
Nella definizione delle interfacce l'\,uso dell'\,offset è opzionale, ma a 
differenza delle classi, che richiedono un offset minimo 1, per le interfacce
può essere usato qualunque valore iniziale, l'\,offset effettivo viene corretto
quando l'\,interfaccia viene applicata alla classe.
\end{quote}


\section{Campi Singoli}

\subsection{Campo Alfanumerico} \label{sub:yaml.abc}
%--------1---------2---------3---------4---------5---------6---------7---------8
La definizione yaml del campo alfanumerico riflette la struttura imposta dalla
service provider interface, vedi~\ref{lst:AbcModel}.
Un campo alfanumerico è specificato indicando il \textsl{tag} 
\fcolorbox{black}{yellow!75}{\texttt{!Abc}}\,, 
un esempio di definizione di campi alfanumerici è mostrato nel 
cod.~\ref{lst:xmplAbc}, nell'\,esempio è mostrato anche il nodo del default
globale per i campi alfanumerici, i valori impostati sono quelli di default
della \textsl{service provider interface}, quindi non è necessario impostare
esplicitamente i parametri se si vuole impostare questi valori.

\ifesource
\begin{figure*}[!htb]
\begin{lstlisting}[language=yaml, caption={esempio definizione campi alfanumerici}, 
label=lst:xmplAbc]
defaults:
  abc:
    check: Ascii        # :chk: None, Ascii, Latin1, Valid
    onOverflow: Trunc   # :ovf: Error, Trunc
    onUnderflow: Pad    # :unf: Error, Pad
    normalize: None     # :nrm: None, Trim, Trim1
    checkGetter: true   # :get:
    checkSetter: true   # :set:
packages:
  - name: com.example.test
    classes:
      - name: Foo3111
        length: 50
        fields:
          - !Abc { name: cognome    , at:   1, len:    25 }
          - !Abc { name: nome       , at:  26, len:    20 }
          - !Abc { name: stCivile   , at:  46, len:     1 }
          - !Abc { name: nazionalita, at:  47, len:     3 }
          - !Abc { name: sesso      , at:  50, len:     1 }
\end{lstlisting}
\end{figure*}
\else
\begin{elisting}[!htb]
\begin{yamlcode}
defaults:
  abc:
    check: Ascii        # :chk: None, Ascii, Latin1, Valid
    onOverflow: Trunc   # :ovf: Error, Trunc
    onUnderflow: Pad    # :unf: Error, Pad
    normalize: None     # :nrm: None, Trim, Trim1
    checkGetter: true   # :get:
    checkSetter: true   # :set:
packages:
  - name: com.example.test
    classes:
      - name: Foo3111
        length: 55
        fields:
          - !Abc { name: firstName   , at:  1, len: 15 }
          - !Abc { name: lastName    , at: 16, len: 15 }
          - !Num { name: birthDate   , at: 31, len:  8 }
          - !Abc { name: birthPlace  , at: 39, len: 14 }
          - !Abc { name: birthCountry, at: 53, len:  3 }
\end{yamlcode}
\caption{esempio definizione campi alfanumerici}
\label{lst:xmplAbc}
\end{elisting}
\fi

%--------1---------2---------3---------4---------5---------6---------7---------8
Nell'\,esempio, il nodo di default dei campi alfanumerici, è impostato usando
i nomi canonici dei parametri. Il \textit{plugin} usa una funzionalità 
disponibile della libreria per leggere il file yaml, e definisce dei nomi
abbreviati dei parametri, che possono essere utilizzati come alternativa ai
nomi canonici.

\begin{table}[!htb]
\centering
\begin{tabular}{|c|>{\tt}l|>{\tt}c|>{\tt}c|c|l|}
\cline{2-6} \multicolumn{1}{c|}{}
&\multicolumn{5}{c|}{\texttt{!Abc}: \hyperref[lst:AbcModel]{AbcModel}}\\
\cline{2-6} \multicolumn{1}{c|}{}
&\multicolumn{1}{c|}{attributo} & \multicolumn{1}{c|}{alt} 
	& \multicolumn{1}{c|}{tipo} & \multicolumn{1}{c|}{O}
	& \multicolumn{1}{c|}{default} \\
\cline{2-6} \multicolumn{1}{c|}{}
&offset     & at  & int     & {\color{lightgray}\ding{52}} & auto-calcolato \\
\cline{2-6} \multicolumn{1}{c|}{}
&length     & len & int     & \ding{52} & \\
\cline{2-6} \multicolumn{1}{c|}{}
&name       &     & String  & \ding{52} & \\
\cline{2-6} \multicolumn{1}{c|}{}
&override   & ovr & boolean & & \texttt{false} \\
\hline
\parbox[t]{2.5mm}{\multirow{6}{*}{\rotatebox[origin=c]{90}{\hyperref[lst:AbcDefault]{AbcDefault}}}}
&onOverflow & ovf & \hyperref[lst:OverflowAction]{enum} & & \texttt{\$\{defaults.abc.onOverflow:Trunc\}}\\
\cline{2-6}
&onUnderlow & unf & \hyperref[lst:UnderflowAction]{enum} & & \texttt{\$\{defaults.abc.onUnderflow:Pad\}}\\
\cline{2-6}
&check      & chk & \hyperref[lst:CheckAbc]{enum} & & \texttt{\$\{defaults.abc.check:Ascii\}}\\
\cline{2-6}
&normalize  & nrm & \hyperref[lst:NormalizeAbcMode]{enum} & & \texttt{\$\{defaults.abc.normalize:None\}}\\
\cline{2-6}
&checkGetter & get & boolean & & \texttt{\$\{defaults.abc.checkGetter:true\}}\\
\cline{2-6}
&checkSetter & set & boolean & & \texttt{\$\{defaults.abc.checkSetter:true\}}\\
\hline
\end{tabular}
\caption{Attributi impostabili per un campo alfanumerico} \label{tab:attr.abc}
\end{table}

%--------1---------2---------3---------4---------5---------6---------7---------8
Nella tabella~\ref{tab:attr.abc} sono mostrati tutti gli attributi previsti per 
un campo alfanumerico, i relativi nomi abbreviati, il corrispondente tipo-dati,
se l'\,attributo è obbligatorio o facoltativo, e l'\,eventuale valore di 
default.

\subsection{Campo Numerico} \label{sub:yaml.num}
%--------1---------2---------3---------4---------5---------6---------7---------8
La definizione yaml del campo numerico riflette la struttura imposta dalla
service interface, vedi~\ref{lst:NumModel}
Un campo numerico è specificato indicando il \textsl{tag} 
\fcolorbox{black}{yellow!75}{\texttt{!Num}}\,, 
un esempio di definizione di campi numerici è mostrato nel 
cod.~\ref{lst:xmplNum}, nell'\,esempio è mostrato anche il nodo del default
globale per i campi numerici, i valori impostati sono quelli di default
della \textsl{service provider interface}, quindi non è necessario impostare
esplicitamente i parametri se si vuole impostare questi valori.

\ifesource
\begin{figure*}[!htb]
\begin{lstlisting}[language=yaml, caption={esempio definizione campi numerici}, 
label=lst:xmplNum]
defaults:
  num:
    onOverflow: Trunc   # :ovf: Error, Trunc
    onUnderflow: Pad    # :unf: Error, Pad
    normalize: None     # :nrm: None, Trim
    wordWidth: Int      # :wid: Byte(1,byte), Short(2,short), Int(4,int), Long(8,long)
    access: String      # :acc: String(Str), Numeric(Num), Both(All)
packages:
  - name: com.example.test
    classes:
      - name: Foo3112
        length: 8
        doc: No
        fields:
          - !Num { name: year , at: 1, len: 4 }
          - !Num { name: month, at: 5, len: 2 }
          - !Num { name: mday , at: 7, len: 2 }
\end{lstlisting}
\end{figure*}
\else
\begin{elisting}[!htb]
\begin{yamlcode}
defaults:
  num:
    onOverflow: Trunc   # :ovf: Error, Trunc
    onUnderflow: Pad    # :unf: Error, Pad
    normalize: None     # :nrm: None, Trim
    wordWidth: Int      # :wid: Byte(1,byte), Short(2,short), Int(4,int), Long(8,long)
    access: String      # :acc: String(Str), Numeric(Num), Both(All)
packages:
  - name: com.example.test
    classes:
      - name: Foo3112
        length: 8
        doc: No
        fields:
          - !Num { name: year , at: 1, len: 4 }
          - !Num { name: month, at: 5, len: 2 }
          - !Num { name: mday , at: 7, len: 2 }
\end{yamlcode}
\caption{esempio definizione campi numerici}
\label{lst:xmplNum}
\end{elisting}
\fi


\begin{table}[!htb]
\centering
\begin{tabular}{|c|>{\tt}l|>{\tt}c|>{\tt}c|c|l|}
\cline{2-6} \multicolumn{1}{c|}{}
&\multicolumn{5}{c|}{\texttt{!Num}: \hyperref[lst:NumModel]{NumModel}}\\
\cline{2-6} \multicolumn{1}{c|}{}
&\multicolumn{1}{c|}{attributo} & \multicolumn{1}{c|}{alt} 
	& \multicolumn{1}{c|}{tipo} & \multicolumn{1}{c|}{O}
	& \multicolumn{1}{c|}{default} \\
\cline{2-6} \multicolumn{1}{c|}{}
&offset     & at  & int     & {\color{lightgray}\ding{52}} & auto-calcolato \\
\cline{2-6} \multicolumn{1}{c|}{}
&length     & len & int     & \ding{52} & \\
\cline{2-6} \multicolumn{1}{c|}{}
&name       &     & String  & \ding{52} & \\
\cline{2-6} \multicolumn{1}{c|}{}
&override   & ovr & boolean & & \texttt{false} \\
\hline
\parbox[t]{2.5mm}{\multirow{5}{*}{\rotatebox[origin=c]{90}{\hyperref[lst:NumDefault]{NumDefault}}}}
&onOverflow & ovf & \hyperref[lst:OverflowAction]{enum} & & \texttt{\$\{defaults.num.onOverflow:Trunc\}}\\
\cline{2-6}
&onUnderlow & unf & \hyperref[lst:UnderflowAction]{enum} & & \texttt{\$\{defaults.num.onUnderflow:Pad\}}\\
\cline{2-6}
&access     & acc & \hyperref[lst:AccesMode]{enum} & & \texttt{\$\{defaults.num.access:String\}}\\
\cline{2-6}
&wordWidth  & wid & \hyperref[lst:WordWidth]{enum} & & \texttt{\$\{defaults.num.wordWidth:Int\}}\\
\cline{2-6}
&normalize  & nrm & \hyperref[lst:NormalizeNumMode]{enum} & & \texttt{\$\{defaults.num.normalize:None\}}\\
\hline
\end{tabular}
\caption{Attributi impostabili per un campo numerico} \label{tab:attr.num}
\end{table}
%--------1---------2---------3---------4---------5---------6---------7---------8
Nella tabella~\ref{tab:attr.num} sono mostrati tutti gli attributi previsti per 
un campo numerico, i relativi nomi abbreviati, il corrispondente tipo-dati,
se l'\,attributo è obbligatorio o facoltativo, e l'\,eventuale valore di 
default.
%--------1---------2---------3---------4---------5---------6---------7---------8
Anche se i parametri \texttt{acccess} e \texttt{wordWidth} sono stati introdotti
nella \S~\ref{sec:spi.num}, ricordo che un campo ``numerico'' può essere 
gestito come una stringa (dove sono ammessi solo caratteri numerici), o 
convertito in un formato numerico nativo, o entrambi. Il parametro 
\texttt{access} indica se creare soltanto i setter/getter stringa, creare 
soltanto i setter/getter numerici o entrambi.
%come mostrato nei commenti nel default del cod.~\ref{lst:xmplNum} i valori
%dell'\,enum \textsl{String, Numeric, Both} possono essere abbreviati nei 
%corrispondenti \textsl{Str, Num, All}.
%--------1---------2---------3---------4---------5---------6---------7---------8
Nel caso che venga utilizzata una rappresentazione numerica nativa, il parametro
\texttt{wordWidth} indica la rappresentazione nativa di dimensione minima da 
usare.
In generale il \textsl{Service Provider} selezionerà la dimensione della
rappresentazione nativa in base alla dimensione della stringa-dati che finirà 
con rappresentare il valore del campo.


\subsection{Campo Custom (alfanumerico)} \label{sub:yaml.cus}
%--------1---------2---------3---------4---------5---------6---------7---------8
La definizione yaml del campo custom riflette la struttura imposta 
dalla service provider interface, vedi~\ref{lst:CusModel}.
Un campo alfanumerico è specificato indicando il \textsl{tag} 
\fcolorbox{black}{yellow!75}{\texttt{!Cus}}\,, 
un esempio di definizione di campi custom è mostrato nel 
cod.~\ref{lst:xmplCus}, nell'\,esempio è mostrato anche il nodo del default
globale per i campi custom, i valori impostati sono quelli di default
della \textsl{service provider interface}, quindi non è necessario impostare
esplicitamente i parametri se si vuole impostare questi valori.

%--------1---------2---------3---------4---------5---------6---------7---------8
Un campo custom è una estensione di un campo alfanumerico. Un campo alfanumerico
è necessariamente allineato a sinistra, troncato, trim-ato a destra, pad-dato a
destra con spazi, inizializzato a spazi. In un campo custom è possibile 
scegliere l'\,allineamento del campo, il carattere di  pad-ding e di 
inizializzazione; ha un \texttt{check} esteso rispetto a quello alfanumerico,
infine, l'\,attributo \texttt{regex} può essere usato per validare i valori
ammessi per il campo (al posto di quello definito con \texttt{check}).

\ifesource
\begin{figure*}[!htb]
\begin{lstlisting}[language=yaml, caption={esempio definizione campi custom}, 
label=lst:xmplCus]
defaults:
  cus:
    padChar: ' '        # :pad:
    initChar: ' '       # :ini:
    check: Ascii        # :chk: None, Ascii, Latin1, Valid, Digit, DigitOrBlank
    align: LFT          # LFT, RGT
    onOverflow: Trunc   # :ovf: Error, Trunc
    onUnderflow: Pad    # :unf: Error, Pad
    normalize: None     # :nrm: None, Trim, Trim1
    checkGetter: true   # :get:
    checkSetter: true   # :set:
packages:
  - name: com.example.test
    classes:
      - name: Foo3113
        length: 8
        doc: No
        fields:
          - !Cus { name: year , at: 1, len: 4 }
          - !Cus { name: month, at: 5, len: 2 }
          - !Cus { name: mday , at: 7, len: 2 }
\end{lstlisting}
\end{figure*}
\else
\begin{elisting}[!htb]
\begin{yamlcode}
defaults:
  cus:
    padChar: ' '        # :pad:
    initChar: ' '       # :ini:
    check: Ascii        # :chk: None, Ascii, Latin1, Valid, Digit, DigitOrBlank
    align: LFT          # LFT, RGT
    onOverflow: Trunc   # :ovf: Error, Trunc
    onUnderflow: Pad    # :unf: Error, Pad
    normalize: None     # :nrm: None, Trim, Trim1
    checkGetter: true   # :get:
    checkSetter: true   # :set:
packages:
  - name: com.example.test
    classes:
      - name: Foo3113
        length: 8
        doc: No
        fields:
          - !Cus { name: year , at: 1, len: 4 }
          - !Cus { name: month, at: 5, len: 2 }
          - !Cus { name: mday , at: 7, len: 2 }
\end{yamlcode}
\caption{esempio definizione campi custom}
\label{lst:xmplCus}
\end{elisting}
\fi

\begin{table}[!htb]
\centering
\begin{tabular}{|c|>{\tt}l|>{\tt}c|>{\tt}c|c|l|}
\cline{2-6} \multicolumn{1}{c|}{}
&\multicolumn{5}{c|}{\texttt{!Cus}: \hyperref[lst:CusModel]{CusModel}}\\
\cline{2-6} \multicolumn{1}{c|}{}
&\multicolumn{1}{c|}{attributo} & \multicolumn{1}{c|}{alt} 
	& \multicolumn{1}{c|}{tipo} & \multicolumn{1}{c|}{O}
	& \multicolumn{1}{c|}{default} \\
\cline{2-6} \multicolumn{1}{c|}{}
&offset     & at  & int     & {\color{lightgray}\ding{52}} & auto-calcolato \\
\cline{2-6} \multicolumn{1}{c|}{}
&length     & len & int     & \ding{52} & \\
\cline{2-6} \multicolumn{1}{c|}{}
&name       &     & String  & \ding{52} & \\
\cline{2-6} \multicolumn{1}{c|}{}
&override   & ovr & boolean & & \texttt{false} \\
\hline
\parbox[t]{2.5mm}{\multirow{9}{*}{\rotatebox[origin=c]{90}{\hyperref[lst:CusDefault]{CusDefault}}}}
&onOverflow & ovf & \hyperref[lst:OverflowAction]{enum} & & \texttt{\$\{defaults.cus.onOverflow:Trunc\}}\\
\cline{2-6}
&onUnderlow & unf & \hyperref[lst:UnderflowAction]{enum} & & \texttt{\$\{defaults.cus.onUnderflow:Pad\}}\\
\cline{2-6}
&padChar    & pad & char    & & \texttt{\$\{defaults.cus.pad:' '\}}\\
\cline{2-6}
&initChar   & ini & char    & & \texttt{\$\{defaults.cus.ini:' '\}}\\
\cline{2-6}
&check      & chk & \hyperref[lst:CheckCus]{enum} & & \texttt{\$\{defaults.cus.check:Ascii\}}\\
\cline{2-6}
&align      &     & \hyperref[lst:AlignMode]{enum} & & \texttt{\$\{defaults.cus.align:LFT\}}\\
\cline{2-6}
&normalize  & nrm & \hyperref[lst:NormalizeAbcMode]{enum} & & \texttt{\$\{defaults.cus.normalize:None\}}\\
\cline{2-6}
&checkGetter & get & boolean & & \texttt{\$\{defaults.cus.checkGetter:true\}}\\
\cline{2-6}
&checkSetter & set & boolean & & \texttt{\$\{defaults.cus.checkSetter:true\}}\\
\hline \multicolumn{1}{c|}{}
&regex      &     & String  & & \texttt{null} \\
\cline{2-6}
\end{tabular}
\caption{Attributi impostabili per un campo custom} \label{tab:attr.cus}
\end{table}
%--------1---------2---------3---------4---------5---------6---------7---------8
Nella tabella~\ref{tab:attr.cus} sono mostrati tutti gli attributi previsti per 
un campo custom, i relativi nomi abbreviati, il corrispondente 
tipo-dati, se l'\,attributo è obbligatorio o facoltativo, e l'\,eventuale valore 
di default.



\subsection{Campo Numerico nullabile} \label{sub:yaml.nux}
%--------1---------2---------3---------4---------5---------6---------7---------8
La definizione yaml del campo numerico nullabile riflette la struttura imposta 
dalla service interface, vedi~\ref{lst:NuxModel}.
Un campo numerico nullabile è specificato indicando il \textsl{tag} 
\fcolorbox{black}{yellow!75}{\texttt{!Nux}}\,, 
un esempio di definizione di campi numerici nullabili è mostrato nel 
cod.~\ref{lst:xmplNux}, nell'\,esempio è mostrato anche il nodo del default
globale per i campi numerici nullabili, i valori impostati sono quelli di 
default della \textsl{service provider interface}, quindi non è necessario 
impostare esplicitamente i parametri se si vuole impostare questi valori.


\ifesource
\begin{figure*}[!htb]
\begin{lstlisting}[language=yaml, 
caption={esempio definizione campi numerici nullabili}, 
label=lst:xmplNux]
defaults:
  nux:
    onOverflow: Trunc   # :ovf: Error, Trunc
    onUnderflow: Pad    # :unf: Error, Pad
    normalize: None     # :nrm: None, Trim
    wordWidth: Int      # :wid: Byte(1,byte), Short(2,short), Int(4,int), Long(8,long)
    access: String      # :acc: String(Str), Numeric(Num), Both(All)
    initialize: Spaces  # :ini: Spaces(Space), Zeroes(Zero)
packages:
  - name: com.example.test
    classes:
      - name: Foo3114
        length: 8
        doc: No
        fields:
          - !Nux { name: year , at: 1, len: 4 }
          - !Nux { name: month, at: 5, len: 2 }
          - !Nux { name: mday , at: 7, len: 2 }
\end{lstlisting}
\end{figure*}
\else
\begin{elisting}[!htb]
\begin{yamlcode}
defaults:
  nux:
    onOverflow: Trunc   # :ovf: Error, Trunc
    onUnderflow: Pad    # :unf: Error, Pad
    normalize: None     # :nrm: None, Trim
    wordWidth: Int      # :wid: Byte(1,byte), Short(2,short), Int(4,int), Long(8,long)
    access: String      # :acc: String(Str), Numeric(Num), Both(All)
    initialize: Spaces  # :ini: Spaces(Space), Zeroes(Zero)
packages:
  - name: com.example.test
    classes:
      - name: Foo3114
        length: 8
        doc: No
        fields:
          - !Nux { name: year , at: 1, len: 4 }
          - !Nux { name: month, at: 5, len: 2 }
          - !Nux { name: mday , at: 7, len: 2 }
\end{yamlcode}
\caption{esempio definizione campi numerici nullabili}
\label{lst:xmplNux}
\end{elisting}
\fi

%--------1---------2---------3---------4---------5---------6---------7---------8
Un campo numerico nullabile è una estensione di un campo numerico ordinario,
la differenza è che nella rappresentazione stringa-dati può assumere il valore
spazio (tutti spazi), che corrisponde nella classe dati al valore \texttt{null}.
Conseguentemente nella definizione del campo è presente un parametro aggiuntivo 
per indicare se il campo deve essere inizializzato a \texttt{null} o a zero 
quando la classe-dati viene creata col costruttore vuoto.

\begin{table}[!htb]
\centering
\begin{tabular}{|c|>{\tt}l|>{\tt}c|>{\tt}c|c|l|}
\cline{2-6} \multicolumn{1}{c|}{}
&\multicolumn{5}{c|}{\texttt{!Nux}: \hyperref[lst:NuxModel]{NuxModel}}\\
\cline{2-6} \multicolumn{1}{c|}{}
&\multicolumn{1}{c|}{attributo} & \multicolumn{1}{c|}{alt} 
	& \multicolumn{1}{c|}{tipo} & \multicolumn{1}{c|}{O}
	& \multicolumn{1}{c|}{default} \\
\cline{2-6} \multicolumn{1}{c|}{}
&offset     & at  & int     & {\color{lightgray}\ding{52}} & auto-calcolato \\
\cline{2-6} \multicolumn{1}{c|}{}
&length     & len & int     & \ding{52} & \\
\cline{2-6} \multicolumn{1}{c|}{}
&name       &     & String  & \ding{52} & \\
\cline{2-6} \multicolumn{1}{c|}{}
&override   & ovr & boolean & & \texttt{false} \\
\hline
\parbox[t]{2.5mm}{\multirow{6}{*}{\rotatebox[origin=c]{90}{\hyperref[lst:NuxDefault]{NuxDefault}}}}
&onOverflow & ovf & \hyperref[lst:OverflowAction]{enum} & & \texttt{\$\{defaults.nux.onOverflow:Trunc\}}\\
\cline{2-6}
&onUnderlow & unf & \hyperref[lst:UnderflowAction]{enum} & & \texttt{\$\{defaults.nux.onUnderflow:Pad\}}\\
\cline{2-6}
&access     & acc & \hyperref[lst:AccesMode]{enum} & & \texttt{\$\{defaults.nux.access:String\}}\\
\cline{2-6}
&wordWidth  & wid & \hyperref[lst:WordWidth]{enum} & & \texttt{\$\{defaults.nux.wordWidth:Int\}}\\
\cline{2-6}
&normalize  & nrm & \hyperref[lst:NormalizeNumMode]{enum} & & \texttt{\$\{defaults.nux.normalize:None\}}\\
\cline{2-6}
&initialize & ini & \hyperref[lst:InitializeNuxMode]{enum} & & \texttt{\$\{defaults.nux.initialize:Space\}}\\
\hline
\end{tabular}
\caption{Attributi impostabili per un campo numerico nullabile} \label{tab:attr.nux}
\end{table}
%--------1---------2---------3---------4---------5---------6---------7---------8
Nella tabella~\ref{tab:attr.nux} sono mostrati tutti gli attributi previsti per 
un campo numerico, i relativi nomi abbreviati, il corrispondente tipo-dati,
se l'\,attributo è obbligatorio o facoltativo, e l'\,eventuale valore di 
default.

\subsection{Campo Dominio} \label{sub:yaml.dom}
%--------1---------2---------3---------4---------5---------6---------7---------8
La definizione yaml del campo dominio riflette la struttura imposta dalla
service interface, vedi~\ref{lst:DomModel}.
Un campo dominio è specificato indicando il \textsl{tag} 
\fcolorbox{black}{yellow!75}{\texttt{!Dom}}\,, 
un esempio di definizione di campi dominio è mostrato nel 
cod.~\ref{lst:xmplDom}, questo tipo di campo non ha nessun default globale.


\ifesource
\begin{figure*}[!htb]
\begin{lstlisting}[language=yaml, caption={esempio definizione campi dominio}, 
label=lst:xmplDom]
packages:
  - name: com.example.test
    classes:
      - name: Foo3115
        length: 12
        doc: No
        fields:
          - !Num { name: year , at: 1, len: 4 }
          - !Dom { name: month, at: 5, len: 3, 
                   items: [ Jan, Feb, Mar, Apr, May, Jun, Jul, Aug, Sep, Oct, Nov, Dec ] }
          - !Num { name: mday , at: 8, len: 2 }
          - !Dom { name: wday , at: 10, len: 3, 
                   items: [ Sun, Mon, Tue, Wed, Thu, Fri, Sat ] }
\end{lstlisting}
\end{figure*}
\else
\begin{elisting}[!htb]
\begin{yamlcode}
packages:
  - name: com.example.test
    classes:
      - name: Foo3115
        length: 12
        doc: No
        fields:
          - !Num { name: year , at: 1, len: 4 }
          - !Dom { name: month, at: 5, len: 3, 
                   items: [ Jan, Feb, Mar, Apr, May, Jun, Jul, Aug, Sep, Oct, Nov, Dec ] }
          - !Num { name: mday , at: 8, len: 2 }
          - !Dom { name: wday , at: 10, len: 3, 
                   items: [ Sun, Mon, Tue, Wed, Thu, Fri, Sat ] }
\end{yamlcode}
\caption{esempio definizione campi dominio}
\label{lst:xmplDom}
\end{elisting}
\fi

%--------1---------2---------3---------4---------5---------6---------7---------8
Un campo dominio è sostanzialmente un campo alfanumerico, che può assumere solo
un limitato numero di valori.

\begin{table}[!htb]
\centering
\begin{tabular}{|>{\tt}l|>{\tt}c|>{\tt}c|c|l|}
\hline
\multicolumn{5}{|c|}{\texttt{!Dom}: \hyperref[lst:DomModel]{DomModel}}\\
\hline
\multicolumn{1}{|c|}{attributo} & \multicolumn{1}{c|}{alt} 
	& \multicolumn{1}{c|}{tipo} & \multicolumn{1}{c|}{O}
	& \multicolumn{1}{c|}{default} \\
\hline
offset     & at  & int     & {\color{lightgray}\ding{52}} & auto-calcolato \\
\hline
length     & len & int     & \ding{52} & \\
\hline
name       &     & String  & \ding{52} & \\
\hline
override   & ovr & boolean & & \texttt{false} \\
\hline
items      &     & array  & \ding{52} & \\
\hline
\end{tabular}
\caption{Attributi impostabili per un campo dominio} \label{tab:attr.dom}
\end{table}
%--------1---------2---------3---------4---------5---------6---------7---------8
Nella tabella~\ref{tab:attr.dom} sono mostrati tutti gli attributi previsti per 
un campo dominio, i relativi nomi abbreviati, il corrispondente tipo-dati,
se l'\,attributo è obbligatorio o facoltativo, e l'\,eventuale valore di 
default.
Quando la classe-dati è creata col costruttore vuoto il campo viene 
inizializzato con il primo valore tra quelli forniti della lista dei possibili
valori.



\subsection{Campo Filler} \label{sub:yaml.fil}
%--------1---------2---------3---------4---------5---------6---------7---------8
La definizione yaml del campo filler riflette la struttura imposta dalla
service interface, vedi~\ref{lst:FilModel}.
Un campo filler è specificato indicando il \textsl{tag} 
\fcolorbox{black}{yellow!75}{\texttt{!Fil}}\,, 
un esempio di definizione di campi filler è mostrato nel 
cod.~\ref{lst:xmplFil}, nell'\,esempio è mostrato anche il nodo del default
globale per i campi filler, il valore impostato è quello di 
default della \textsl{service provider interface}, quindi non è necessario 
impostare esplicitamente il parametro se si vuole impostare questo valore.

\ifesource
\begin{figure*}[!htb]
\begin{lstlisting}[language=yaml, 
caption={esempio definizione campi filler}, 
label=lst:xmplFil]
defaults:
  fil:
    fill: 0             # \u0000
packages:
  - name: com.example.test
    classes:
      - name: Foo3116
        length: 10
        doc: No
        fields:
          - !Num { at: 1, len: 4, name: year }
          - !Fil { at: 5, len: 1, fill: '-' }
          - !Num { at: 6, len: 2, name: month }
          - !Fil { at: 8, len: 1, fill: '-' }
          - !Num { at: 9, len: 2, name: mday }
\end{lstlisting}
\end{figure*}
\else
\begin{elisting}[!htb]
\begin{yamlcode}
defaults:
  fil:
    fill: 0             # \u0000
packages:
  - name: com.example.test
    classes:
      - name: Foo3116
        length: 10
        doc: No
        fields:
          - !Num { at: 1, len: 4, name: year }
          - !Fil { at: 5, len: 1, fill: '-' }
          - !Num { at: 6, len: 2, name: month }
          - !Fil { at: 8, len: 1, fill: '-' }
          - !Num { at: 9, len: 2, name: mday }
\end{yamlcode}
\caption{esempio definizione campi filler}
\label{lst:xmplFil}
\end{elisting}
\fi
%--------1---------2---------3---------4---------5---------6---------7---------8
Un campo filler non è un campo vero e proprio, non vengono generati i
setter/getter, non viene fatto nessun controllo sul valore della stringa-dati
corrispondente. Indica semplicemente un'\,area della stringa-dati a cui non 
siamo interessati, ma che deve essere presente nella definizione della 
struttura per non lasciare aree non definite.

\begin{table}[!htb]
\centering
\begin{tabular}{|c|>{\tt}l|>{\tt}c|>{\tt}c|c|l|}
\cline{2-6} \multicolumn{1}{c|}{}
&\multicolumn{5}{c|}{\texttt{!Fil}: \hyperref[lst:FilModel]{FilModel}}\\
\cline{2-6} \multicolumn{1}{c|}{}
&\multicolumn{1}{c|}{attributo} & \multicolumn{1}{c|}{alt} 
	& \multicolumn{1}{c|}{tipo} & \multicolumn{1}{c|}{O}
	& \multicolumn{1}{c|}{default} \\
\cline{2-6} \multicolumn{1}{c|}{}
&offset     & at  & int     & {\color{lightgray}\ding{52}} & auto-calcolato \\
\cline{2-6} \multicolumn{1}{c|}{}
&length     & len & int     & \ding{52} & \\
\hline
\hyperref[lst:FilDefault]{FilDefault}
&fill       &     & char    & & \texttt{\$\{defaults.fil.fill:0\}}\\
\hline
\end{tabular}
\caption{Attributi impostabili per un campo filler} \label{tab:attr.fil}
\end{table}
%--------1---------2---------3---------4---------5---------6---------7---------8
Nella tabella~\ref{tab:attr.fil} sono mostrati tutti gli attributi previsti per 
un campo filler, i relativi nomi abbreviati, il corrispondente tipo-dati,
se l'\,attributo è obbligatorio o facoltativo, e l'\,eventuale valore di 
default.


\subsection{Campo Costante} \label{sub:yaml.val}
%--------1---------2---------3---------4---------5---------6---------7---------8
La definizione yaml del campo costante riflette la struttura imposta dalla
service interface, vedi~\ref{lst:ValModel}.
Un campo costante è specificato indicando il \textsl{tag} 
\fcolorbox{black}{yellow!75}{\texttt{!Val}}\,, 
un esempio di definizione di campi costanti è mostrato nel 
cod.~\ref{lst:xmplVal}, questo tipo di campo non prevede default globali.

\ifesource
\begin{figure*}[!htb]
\begin{lstlisting}[language=yaml, 
caption={esempio definizione campi costanti}, 
label=lst:xmplVal]
packages:
  - name: com.example.test
    classes:
      - name: Foo3117
        length: 10
        fields:
          - !Num { at: 1, len: 4, name: year }
          - !Val { at: 5, len: 1, val: "-" }
          - !Num { at: 6, len: 2, name: month }
          - !Val { at: 8, len: 1, val: "-" }
          - !Num { at: 9, len: 2, name: mday }
\end{lstlisting}
\end{figure*}
\else
\begin{elisting}[!htb]
\begin{yamlcode}
packages:
  - name: com.example.test
    classes:
      - name: Foo3117
        length: 10
        fields:
          - !Num { at: 1, len: 4, name: year }
          - !Val { at: 5, len: 1, val: "-" }
          - !Num { at: 6, len: 2, name: month }
          - !Val { at: 8, len: 1, val: "-" }
          - !Num { at: 9, len: 2, name: mday }
\end{yamlcode}
\caption{esempio definizione campi costanti}
\label{lst:xmplVal}
\end{elisting}
\fi

%--------1---------2---------3---------4---------5---------6---------7---------8
Un campo costante può essere visto come una variante di un campo filler, o come
un campo dominio con un solo valore. Per questo tipo di campo non vengono 
generati i setter/getter, ma il campo viene controllato per verificare che la
stringa-dati corrispondente al campo abbia il valore atteso.

\begin{table}[!htb]
\centering
\begin{tabular}{|>{\tt}l|>{\tt}c|>{\tt}c|c|l|}
\hline
\multicolumn{5}{|c|}{\texttt{!Val}: \hyperref[lst:ValModel]{ValModel}}\\
\hline
\multicolumn{1}{|c|}{attributo} & \multicolumn{1}{c|}{alt} 
	& \multicolumn{1}{c|}{tipo} & \multicolumn{1}{c|}{O}
	& \multicolumn{1}{c|}{default} \\
\hline
offset     & at  & int     & {\color{lightgray}\ding{52}} & auto-calcolato \\
\hline
length     & len & int     & \ding{52} & \\
\hline
value      & val & string  & \ding{52} & \\
\hline
\end{tabular}
\caption{Attributi impostabili per un campo costante} \label{tab:attr.val}
\end{table}
%--------1---------2---------3---------4---------5---------6---------7---------8
Nella tabella~\ref{tab:attr.val} sono mostrati tutti gli attributi previsti per 
un campo costante, i relativi nomi abbreviati, il corrispondente tipo-dati,
se l'\,attributo è obbligatorio o facoltativo, e l'\,eventuale valore di 
default.


\section{Campi multipli}
%--------1---------2---------3---------4---------5---------6---------7---------8
In alcuni casi è utile raggruppare alcuni campi all'\,interno di un elemento
contenitore di contesto. In questo modo è possibile usare lo stesso nome campo
in contesti diversi. Un campo multiplo non ha default globali.

\subsection{Campo Gruppo} \label{sub:yaml.grp}
%--------1---------2---------3---------4---------5---------6---------7---------8
La definizione yaml del campo gruppo riflette la struttura imposta dalla
service interface, vedi~\ref{lst:GrpModel}.
Un campo gruppo è specificato indicando il \textsl{tag} 
\fcolorbox{black}{yellow!75}{\texttt{!Grp}}\,, 
un esempio di definizione di campi gruppo è mostrato nel 
cod.~\ref{lst:xmplGrp}.

\ifesource
\begin{figure*}[!htb]
\begin{lstlisting}[language=yaml, 
caption={esempio definizione gruppo di campi}, 
label=lst:xmplGrp]
packages:
  - name: com.example.test
    classes:
      - name: Foo3118
        length: 12
        fields:
          - !Grp { name: startTime, at: 1, len: 6, fields: [
            !Num { name: hours  , at: 1, len: 2 }, 
            !Num { name: minutes, at: 3, len: 2 }, 
            !Num { name: seconds, at: 5, len: 2 }
            ] }
          - !Grp { name: stopTime , at: 7, len: 6, fields: [
            !Num { name: hours  , at:  7, len: 2 }, 
            !Num { name: minutes, at:  9, len: 2 }, 
            !Num { name: seconds, at: 11, len: 2 }
            ] }
\end{lstlisting}
\end{figure*}
\else
\begin{elisting}[!htb]
\begin{yamlcode}
packages:
  - name: com.example.test
    classes:
      - name: Foo3118
        length: 12
        fields:
          - !Grp { name: startTime, at: 1, len: 6, fields: [
            !Num { name: hours  , at: 1, len: 2 }, 
            !Num { name: minutes, at: 3, len: 2 }, 
            !Num { name: seconds, at: 5, len: 2 }
            ] }
          - !Grp { name: stopTime , at: 7, len: 6, fields: [
            !Num { name: hours  , at:  7, len: 2 }, 
            !Num { name: minutes, at:  9, len: 2 }, 
            !Num { name: seconds, at: 11, len: 2 }
            ] }
\end{yamlcode}
\caption{esempio definizione gruppo di campi}
\label{lst:xmplGrp}
\end{elisting}
\fi

\begin{table}[!htb]
\centering
\begin{tabular}{|>{\tt}l|>{\tt}c|>{\tt}c|c|l|}
\hline
\multicolumn{5}{|c|}{\texttt{!Grp}: \hyperref[lst:GrpModel]{GrpModel}}\\
\hline
\multicolumn{1}{|c|}{attributo} & \multicolumn{1}{c|}{alt} 
	& \multicolumn{1}{c|}{tipo} & \multicolumn{1}{c|}{O}
	& \multicolumn{1}{c|}{default} \\
\hline
offset     & at  & int     & {\color{lightgray}\ding{52}} & auto-calcolato \\
\hline
length     & len & int     & \ding{52} & \\
\hline
name       &     & String  & \ding{52} & \\
\hline
override   & ovr & boolean & & \texttt{false} \\
\hline
fields     &     & array  & \ding{52} & \\
\hline
\end{tabular}
\caption{Attributi impostabili per un gruppo} \label{tab:attr.grp}
\end{table}
%--------1---------2---------3---------4---------5---------6---------7---------8
Nella tabella~\ref{tab:attr.grp} sono mostrati tutti gli attributi previsti per 
un campo gruppo, i relativi nomi abbreviati, il corrispondente tipo-dati,
se l'\,attributo è obbligatorio o facoltativo, e l'\,eventuale valore di 
default.


\subsection{Campo Gruppo ripetuto} \label{sub:yaml.occ}
%--------1---------2---------3---------4---------5---------6---------7---------8
La definizione yaml del campo gruppo ripetuto riflette la struttura imposta 
dalla service interface, vedi~\ref{lst:OccModel}.
Un campo gruppo ripetuto è specificato indicando il \textsl{tag} 
\fcolorbox{black}{yellow!75}{\texttt{!Occ}}\,, 
un esempio di definizione di campo gruppo ripetuto è mostrato nel 
cod.~\ref{lst:xmplOcc}.

\ifesource
\begin{figure*}[!htb]
\begin{lstlisting}[language=yaml, 
caption={esempio definizione gruppo di campi ripetuto}, 
label=lst:xmplOcc]
packages:
  - name: com.example.test
    classes:
      - name: Foo3119
        length: 590
        fields:
          - !Num { name: nmErrors, at: 1, len: 2 }
          - !Occ { name: tabError, at: 3, len: 49, x: 12, fields: [
            !Abc { name: status  , at:  3, len:  5 },
            !Num { name: code    , at:  8, len:  4 },
            !Abc { name: message , at: 12, len: 40 }
          ] }
\end{lstlisting}
\end{figure*}
\else
\begin{elisting}[!htb]
\begin{yamlcode}
packages:
  - name: com.example.test
    classes:
      - name: Foo3119
        length: 590
        fields:
          - !Num { name: nmErrors, at: 1, len: 2 }
          - !Occ { name: tabError, at: 3, len: 49, x: 12, fields: [
            !Abc { name: status  , at:  3, len:  5 },
            !Num { name: code    , at:  8, len:  4 },
            !Abc { name: message , at: 12, len: 40 }
          ] }
\end{yamlcode}
\caption{esempio definizione gruppo di campi ripetuto}
\label{lst:xmplOcc}
\end{elisting}
\fi

\begin{table}[!htb]
\centering
\begin{tabular}{|>{\tt}l|>{\tt}c|>{\tt}c|c|l|}
\hline
\multicolumn{5}{|c|}{\texttt{!Occ}: \hyperref[lst:OccModel]{OccModel}}\\
\hline
\multicolumn{1}{|c|}{attributo} & \multicolumn{1}{c|}{alt} 
	& \multicolumn{1}{c|}{tipo} & \multicolumn{1}{c|}{O}
	& \multicolumn{1}{c|}{default} \\
\hline
offset     & at  & int     & {\color{lightgray}\ding{52}} & auto-calcolato \\
\hline
length     & len & int     & \ding{52} & \\
\hline
name       &     & String  & \ding{52} & \\
\hline
override   & ovr & boolean & & \texttt{false} \\
\hline
times      & x   & int     & \ding{52} & \\
\hline
fields     &     & array  & \ding{52} & \\
\hline
\end{tabular}
\caption{Attributi impostabili per un gruppo ripetuto} \label{tab:attr.occ}
\end{table}
%--------1---------2---------3---------4---------5---------6---------7---------8
Nella tabella~\ref{tab:attr.occ} sono mostrati tutti gli attributi previsti per 
un campo gruppo ripetuto, i relativi nomi abbreviati, il corrispondente 
tipo-dati, se l'\,attributo è obbligatorio o facoltativo, e l'\,eventuale valore 
di default.

\subsection{Campo gruppo incorporato da interfaccia} \label{sub:yaml.emb}
%--------1---------2---------3---------4---------5---------6---------7---------8
La definizione yaml del campo incorporato riflette la struttura imposta dalla
service interface, vedi~\ref{lst:EmbModel}.
Un campo incorporato è specificato indicando il \textsl{tag} 
\fcolorbox{black}{yellow!75}{\texttt{!Emb}}\,, 
un esempio di definizione di campo incorporato è mostrato nel 
cod.~\ref{lst:xmplEmb}.

\ifesource
\begin{figure*}[!htb]
\begin{lstlisting}[language=yaml, 
caption={esempio definizione di campi inclusi da interfaccia}, 
label=lst:xmplEmp]
packages:
  - name: com.example.test
    interfaces:
      - &Time
        name: ITime
        len: 6
        fields:
          - !Num { name: hours  , len: 2 }
          - !Num { name: minutes, len: 2 }
          - !Num { name: seconds, len: 2 }
    classes:
      - name: Foo311a
        length: 14
        fields:
          - !Num { name: year , at: 1, len: 4 }
          - !Num { name: month, at: 5, len: 2 }
          - !Num { name: mday , at: 7, len: 2 }
          - !Emb { src: *Time , at: 9, len: 6 }
\end{lstlisting}
\end{figure*}
\else
\begin{elisting}[!htb]
\begin{yamlcode}
packages:
  - name: com.example.test
    interfaces:
      - &Time
        name: ITime
        len: 6
        fields:
          - !Num { name: hours  , len: 2 }
          - !Num { name: minutes, len: 2 }
          - !Num { name: seconds, len: 2 }
    classes:
      - name: Foo311a
        length: 14
        fields:
          - !Num { name: year , at: 1, len: 4 }
          - !Num { name: month, at: 5, len: 2 }
          - !Num { name: mday , at: 7, len: 2 }
          - !Emb { src: *Time , at: 9, len: 6 }
\end{yamlcode}
\caption{esempio definizione gruppo di campi inclusi da interfaccia}
\label{lst:xmplEmb}
\end{elisting}
\fi

%--------1---------2---------3---------4---------5---------6---------7---------8
Un campo gruppo da interfaccia a differenza degli altri campi multipli non 
crea un elemento di contesto esplicito. I campi sono figli della struttura 
corrente, non di un elemento di contesto. Ma, l'\,elemento corrente
implementa l'\,interfaccia, e questo crea un contesto implicito.

\begin{table}[!htb]
\centering
\begin{tabular}{|>{\tt}l|>{\tt}c|>{\tt}c|c|l|}
\hline
\multicolumn{5}{|c|}{\texttt{!Emb}: \hyperref[lst:EmbModel]{EmbModel}}\\
\hline
\multicolumn{1}{|c|}{attributo} & \multicolumn{1}{c|}{alt} 
	& \multicolumn{1}{c|}{tipo} & \multicolumn{1}{c|}{O}
	& \multicolumn{1}{c|}{default} \\
\hline
offset     & at  & int     & {\color{lightgray}\ding{52}} & auto-calcolato\\
\hline
length     & len & int     & \ding{52} & \\
\hline
source     & src  & interface & \ding{52} & \\
\hline
\end{tabular}
\caption{Attributi impostabili per un elenco di campi importato da una interfaccia}
\label{tab:attr.emb}
\end{table}
%--------1---------2---------3---------4---------5---------6---------7---------8
Nella tabella~\ref{tab:attr.emb} sono mostrati tutti gli attributi previsti per 
un campo gruppo da interfaccia, i relativi nomi abbreviati, il corrispondente 
tipo-dati, se l'\,attributo è obbligatorio o facoltativo, e l'\,eventuale valore 
di default.


\subsection{Campo Gruppo da interfaccia} \label{sub:yaml.igrp}
%--------1---------2---------3---------4---------5---------6---------7---------8
La definizione yaml del campo gruppo/interfaccia riflette la struttura imposta 
dalla service interface, vedi~\ref{lst:GrpTraitModel}.
Un campo gruppo/interfaccia è specificato indicando il \textsl{tag} 
\fcolorbox{black}{yellow!75}{\texttt{!GRP}}\,, 
un esempio di definizione di campi gruppo/interfaccia è mostrato nel 
cod.~\ref{lst:xmplIGrp}.

\ifesource
\begin{figure*}[!htb]
\begin{lstlisting}[language=yaml, 
caption={esempio definizione gruppo di campi da interfaccia}, 
label=lst:xmplIGrp]
packages:
  - name: com.example.test
    interfaces:
      - &Time
        name: ITime
        len: 6
        fields:
          - !Num { name: hours  , len: 2 }
          - !Num { name: minutes, len: 2 }
          - !Num { name: seconds, len: 2 }
    classes:
      - name: Foo311b
        length: 12
        fields:
          - !GRP { name: startTime, at: 1, len: 6, as: *Time }
          - !GRP { name: stopTime , at: 7, len: 6, as: *Time }
\end{lstlisting}
\end{figure*}
\else
\begin{elisting}[!htb]
\begin{yamlcode}
packages:
  - name: com.example.test
    interfaces:
      - &Time
        name: ITime
        len: 6
        fields:
          - !Num { name: hours  , len: 2 }
          - !Num { name: minutes, len: 2 }
          - !Num { name: seconds, len: 2 }
    classes:
      - name: Foo311b
        length: 12
        fields:
          - !GRP { name: startTime, at: 1, len: 6, as: *Time }
          - !GRP { name: stopTime , at: 7, len: 6, as: *Time }
\end{yamlcode}
\caption{esempio definizione gruppo di campi da interfaccia}
\label{lst:xmplIGrp}
\end{elisting}
\fi
%--------1---------2---------3---------4---------5---------6---------7---------8
Un campo gruppo/interfaccia è simile al campo gruppo, la differenza è che i
campi del gruppo non sono definiti singolarmente, ma tutti insieme importandoli
dalla interfaccia. Il gruppo implementerà l'\,interfaccia.

\begin{table}[!htb]
\centering
\begin{tabular}{|>{\tt}l|>{\tt}c|>{\tt}c|c|l|}
\hline
\multicolumn{5}{|c|}{\texttt{!GRP}: \hyperref[lst:GrpTraitModel]{GrpTraitModel}}\\
\hline
\multicolumn{1}{|c|}{attributo} & \multicolumn{1}{c|}{alt} 
	& \multicolumn{1}{c|}{tipo} & \multicolumn{1}{c|}{O}
	& \multicolumn{1}{c|}{default} \\
\hline
offset     & at  & int     & {\color{lightgray}\ding{52}} & auto-calcolato \\
\hline
length     & len & int     & \ding{52} & \\
\hline
name       &     & String  & \ding{52} & \\
\hline
override   & ovr & boolean & & \texttt{false} \\
\hline
typedef    & as  & interface & \ding{52} & \\
\hline
\end{tabular}
\caption{Attributi impostabili per un gruppo da interfaccia} \label{tab:attr.igrp}
\end{table}
%--------1---------2---------3---------4---------5---------6---------7---------8
Nella tabella~\ref{tab:attr.igrp} sono mostrati tutti gli attributi previsti per 
un campo gruppo da interfaccia, i relativi nomi abbreviati, il corrispondente 
tipo-dati, se l'\,attributo è obbligatorio o facoltativo, e l'\,eventuale valore 
di default.

\subsection{Campo Gruppo ripetuto da interfaccia} \label{sub:yaml.iocc}
%--------1---------2---------3---------4---------5---------6---------7---------8
La definizione yaml del campo gruppo/interfaccia ripetuto riflette la struttura 
imposta dalla service interface, vedi~\ref{lst:OccTraitModel}.
Un campo gruppo/interfaccia ripetuto è specificato indicando il \textsl{tag} 
\fcolorbox{black}{yellow!75}{\texttt{!OCC}}\,, 
un esempio di definizione di campi gruppo/interfaccia ripetuto è mostrato nel 
cod.~\ref{lst:xmplIOcc}.

\ifesource
\begin{figure*}[!htb]
\begin{lstlisting}[language=yaml, 
caption={esempio definizione gruppo di campi ripetuto da interfaccia}, 
label=lst:xmplIOcc]
packages:
  - name: com.example.test
    interfaces:
      - &Error
        name: IError
        len: 49
        fields:
          - !Abc { name: status  , at:  1, len:  5 }
          - !Num { name: code    , at:  6, len:  4 }
          - !Abc { name: message , at: 10, len: 40 }
    classes:
      - name: Foo311c
        length: 590
        fields:
          - !Num { name: nmErrors, at: 1, len: 2}
          - !OCC { name: tabError, at: 3, len: 49, x: 12, as: *Error }
\end{lstlisting}
\end{figure*}
\else
\begin{elisting}[!htb]
\begin{yamlcode}
packages:
  - name: com.example.test
    interfaces:
      - &Error
        name: IError
        len: 49
        fileds:
          - !Abc { name: status  , at:  1, len:  5}
          - !Num { name: code    , at:  6, len:  4}
          - !Abc { name: message , at: 10, len: 40}
    classes:
      - name: Foo311c
        length: 590
        fields:
          - !Num { name: nmErrors, at: 1, len: 2}
          - !OCC { name: tabError, at: 3, len: 49, x: 12, as: *Error }
\end{yamlcode}
\caption{esempio definizione gruppo di campi ripetuto da interfaccia}
\label{lst:xmplIOcc}
\end{elisting}
\fi
%--------1---------2---------3---------4---------5---------6---------7---------8
Un campo gruppo/interfaccia ripetuto è simile al campo gruppo ripetuto, la 
differenza è che i campi del gruppo non sono definiti singolarmente, ma tutti 
insieme importandoli dalla interfaccia. Il gruppo implementerà l'\,interfaccia.

\begin{table}[!htb]
\centering
\begin{tabular}{|>{\tt}l|>{\tt}c|>{\tt}c|c|l|}
\hline
\multicolumn{5}{|c|}{\texttt{!OCC}: \hyperref[lst:OccTraitModel]{OccTraitModel}}\\
\hline
\multicolumn{1}{|c|}{attributo} & \multicolumn{1}{c|}{alt} 
	& \multicolumn{1}{c|}{tipo} & \multicolumn{1}{c|}{O}
	& \multicolumn{1}{c|}{default} \\
\hline
offset     & at  & int     & {\color{lightgray}\ding{52}} & auto-calcolato \\
\hline
length     & len & int     & \ding{52} & \\
\hline
name       &     & String  & \ding{52} & \\
\hline
override   & ovr & boolean & & \texttt{false} \\
\hline
times      & x   & int     & \ding{52} & \\
\hline
typedef    & as  & interface & \ding{52} & \\
\hline
\end{tabular}
\caption{Attributi impostabili per un gruppo ripetuto da interfaccia} 
\label{tab:attr.iocc}
\end{table}
%--------1---------2---------3---------4---------5---------6---------7---------8
Nella tabella~\ref{tab:attr.iocc} sono mostrati tutti gli attributi previsti per 
un campo gruppo ripetuto da interfaccia, i relativi nomi abbreviati, il 
corrispondente tipo-dati, se l'\,attributo è obbligatorio o facoltativo, e 
l'\,eventuale valore di default.

