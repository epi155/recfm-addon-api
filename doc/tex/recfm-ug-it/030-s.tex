%  ____                  _          
% / ___|  ___ _ ____   _(_) ___ ___ 
% \___ \ / _ \ '__\ \ / / |/ __/ _ \
%  ___) |  __/ |   \ V /| | (_|  __/
% |____/ \___|_|    \_/ |_|\___\___|
%                                  

\chapter*{Service}
%--------1---------2---------3---------4---------5---------6---------7---------8
La \textsl{Service Provider Interface} fissa semplicemente la struttura 
generale, ma contiene solo interfacce e java-bean.
Il \textsl{Service} è l'\,applicazione che permette all'\,utente di fornire
l'\,input previsto dalla \textsl{Service Provider Interface}.
Per produrre l'\,output, il \textsl{Service} utilizzerà il 
\textsl{ServiceLoader} per cercare nel classpath un \textsl{Service Provider} 
che implementi la \textsl{Service Provider Interface}, e sarà il 
\textsl{Service Provider} a generare l'\,output dall'\,input.

%--------1---------2---------3---------4---------5---------6---------7---------8
Il cuore del \textsl{Service Provider Interface} \verb!recfm-addon-api! è la
interfaccia \textsl{CodeProvider}. L'\,implementazione della interfaccia
viene cercata col meccanismo del ServiceLoader, cod.\ref{lst:getCodeProvider}.

\ifesource
\begin{figure*}[!htb]
\begin{lstlisting}[language=java, caption=recupero del CodeProvider, 
label=lst:getCodeProvider]
        ServiceLoader<CodeProvider> loader = ServiceLoader.load(CodeProvider.class);
        (*\hyperref[lst:CodeProvider]{CodeProvider}*) codeProvider = loader.iterator().next();
        (*\hyperref[lst:CodeFactory]{CodeFactory}*) factory = codeProvider.getInstance();
\end{lstlisting}
\end{figure*}
\else
\begin{elisting}[!htb]
\begin{javacode}
        ServiceLoader<CodeProvider> loader = ServiceLoader.load(CodeProvider.class);
        |\hyperref[lst:CodeProvider]{CodeProvider}| codeProvider = loader.iterator().next();
        |\hyperref[lst:CodeFactory]{CodeFactory}| factory = codeProvider.getInstance();
\end{javacode}
\caption{recupero del CodeProvider}
\label{lst:getCodeProvider}
\end{elisting}
\fi

%--------1---------2---------3---------4---------5---------6---------7---------8
Una volta recuperata una istanza di \textsl{CodeFactory} è possibile creare le
definizioni delle stringhe-dati e generare i sorgenti delle classi-dati.

%--------1---------2---------3---------4---------5---------6---------7---------8
Sono stati sviluppati due \textsl{client}, uno sotto forma di maven plugin 
\verb!recfm-maven-plugin!, e l'\,altro sotto forma di gradle plugin 
\verb!recfm-gradle-plugin!. Il codice in gran parte è identico, cambia solo il 
meccanismo di innesco.
