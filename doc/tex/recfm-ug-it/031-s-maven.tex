\chapter{Maven plugin}\label{sec:maven}
%--------1---------2---------3---------4---------5---------6---------7---------8
Il maven plugin \verb!recfm-maven-plugin! permette di creare più classi e 
interfacce usando uno più file di configurazione in formato yaml.
Le librerie esterne utilizzate richiedono il java 8, quindi per eseguire questo 
plugin è necessario almeno il java 8.

Il plugin si aspetta come parametri di configurazione
\ifesource
\begin{figure*}[!htb]
\begin{lstlisting}[language=java, caption=parametri impostabili del maven plugin, 
label=lst:maven.conf]
    @Parameter(defaultValue = "${project.build.directory}/generated-sources/recfm",
        property = "maven.recfm.generateDirectory", required = true)
    private File generateDirectory;

    @Parameter(defaultValue = "${project.build.resources[0].directory}",
        property = "maven.recfm.settingsDirectory", required = true)
    private File settingsDirectory;
    
    @Parameter(required = true)
    private String[] settings;

    @Parameter(defaultValue = "true", property = "maven.recfm.addCompileSourceRoot")
    private boolean addCompileSourceRoot = true;

    @Parameter(defaultValue = "false", property = "maven.recfm.addTestCompileSourceRoot")
    private boolean addTestCompileSourceRoot = false;
    
    @Parameter
    private String codeProviderClassName;
\end{lstlisting}
\end{figure*}
\else
\begin{elisting}[!htb]
\begin{javacode}
    @Parameter(defaultValue = "${project.build.directory}/generated-sources/recfm",
        property = "maven.recfm.generateDirectory", required = true)
    private File generateDirectory;

    @Parameter(defaultValue = "${project.build.resources[0].directory}",
        property = "maven.recfm.settingsDirectory", required = true)
    private File settingsDirectory;
    
    @Parameter(required = true)
    private String[] settings;

    @Parameter(defaultValue = "true", property = "maven.recfm.addCompileSourceRoot")
    private boolean addCompileSourceRoot = true;

    @Parameter(defaultValue = "false", property = "maven.recfm.addTestCompileSourceRoot")
    private boolean addTestCompileSourceRoot = false;
    
    @Parameter
    private String codeProviderClassName;
\end{javacode}
\caption{parametri impostabili del maven plugin}
\label{lst:maven.conf}
\end{elisting}
\fi

\begin{itemize}
%--------1---------2---------3---------4---------5---------6---------7---------8
\item \index{plugin!generateDirectory}
	Il campo \fcolorbox{black}{yellow!75}{\texttt{generateDirectory}} indica la 
	directory root da utilizzare per la generazione dei sorgenti, viene usato 
	per valorizzare il campo \verb!sourceDirectory! della classe 
	\hyperref[lst:GenerateArgs]{\texttt{GenerateArgs}}, come si vede dalla 
	definizione, se il parametro è omesso viene utilizzata la directory 
	\verb!target/generated-sources/recfm!, 	normalmente può essere lasciato il 
	valore di default.
	Gli altri tre parametri di \texttt{GenerateArgs} sono un identificativo del
	programma \textsl{service} e vengono valorizzati automaticamente.
%--------1---------2---------3---------4---------5---------6---------7---------8
\item \index{plugin!settingsDirectory}
	Il campo \fcolorbox{black}{yellow!75}{\texttt{settingsDirectory}} indica la 
	directory che contiene i file	di configurazione, se il parametro è omesso 
	viene usato il valore \verb!src/main/resources!, normalmente può essere 
	lasciato il valore di default.
%--------1---------2---------3---------4---------5---------6---------7---------8
\item \index{plugin!settings}
	Il campo \fcolorbox{black}{yellow!75}{\texttt{settings}} indica l'elenco dei 
	file di configurazione da utilizzare per generare le classi/interfacce; il 
	parametro deve essere fornito al plugin.
%--------1---------2---------3---------4---------5---------6---------7---------8
\item \index{plugin!addCompileSourceRoot}
	Il campo \fcolorbox{black}{yellow!75}{\texttt{addCompileSourceRoot}} è un 
	campo tecnico, indica a maven che la directory dove sono stati generati i 
	sorgenti deve essere inclusa tra quelle utilizzate per la compilazione 
	principale, se il parametro è omesso viene utilizzato il valore 
	\verb!true!; il valore \verb!true! è opportuno quando viene usata una 
	directory di generazione del codice diversa da \verb!src/main/java!, 
	altrimenti è necessario usare plugin aggiuntivi per aggiungere il nuovo path 
	a quello di compilazione di maven.
%--------1---------2---------3---------4---------5---------6---------7---------8
\item \index{plugin!addTestCompileSourceRoot}
	Il campo \fcolorbox{black}{yellow!75}{\texttt{addTestCompileSourceRoot}} è 
	analogo al campo precedente, ma aggiunge la directory di generazione al path 
	di compilazione dei test, se omesso viene utilizzato il valore \verb!false!, 
	tranne in casi particolari può essere lasciato il valore di default.
%--------1---------2---------3---------4---------5---------6---------7---------8
\item \index{plugin!codeProviderClassName}
	Il campo \fcolorbox{black}{yellow!75}{\texttt{codeProviderClassName}} indica 
	quale è la classe concreata che implementa la \textsl{Service Interface}, se 
	omesso viene utilizzata la ``prima'' implementazione recuperata del 
	\textsl{ServiceLoader}; se in classpath è presente una sola implementazione, 
	non è necessario valorizzare il parametro. 
\end{itemize}
%--------1---------2---------3---------4---------5---------6---------7---------8

%--------1---------2---------3---------4---------5---------6---------7---------8
\ifesource
\begin{figure*}[!htb]
\begin{lstlisting}[language=XML, caption=esempio minimale di esecuzione del 
plugin, label=lst:mvn-xmpl]
<plugin>
    <groupId>io.github.epi155</groupId>
    <artifactId>recfm-maven-plugin</artifactId>
    <version>0.7.0</version>
    <executions>
        <execution>
            <goals>
                <goal>generate</goal>
            </goals>
            <configuration>
                <settings>
                    <setting>foo.yaml</setting>
                    <setting>bar.yaml</setting>
                </settings>
            </configuration>
        </execution>
    </executions>
    <dependencies>
        <dependency>
            <groupId>io.github.epi155</groupId>
            <artifactId>recfm-java-addon</artifactId>
            <version>0.7.0</version>
        </dependency>
    </dependencies>
</plugin>
\end{lstlisting}
\end{figure*}
\else
\begin{elisting}[!htb]
\begin{xmlcode}
<plugin>
    <groupId>io.github.epi155</groupId>
    <artifactId>recfm-maven-plugin</artifactId>
    <version>0.7.0</version>
    <executions>
        <execution>
            <goals>
                <goal>generate</goal>
            </goals>
            <configuration>
                <settings>
                    <setting>foo.yaml</setting>
                    <setting>bar.yaml</setting>
                </settings>
            </configuration>
        </execution>
    </executions>
    <dependencies>
        <dependency>
            <groupId>io.github.epi155</groupId>
            <artifactId>recfm-java-addon</artifactId>
            <version>0.7.0</version>
        </dependency>
    </dependencies>
</plugin>
\end{xmlcode}
\caption{esempio minimale di esecuzione del plugin}
\label{lst:mvn-xmpl}
\end{elisting}
\fi
%--------1---------2---------3---------4---------5---------6---------7---------8
Un esempio di esecuzione del plugin è mostrato nel cod.~\ref{lst:mvn-xmpl},
il plugin per essere eseguito deve avere come dipendenza una libreria che 
fornisca l'\,implementazione dell'\,inter\-fac\-cia, altrimenti il 
\verb!ServiceLoader! non trova nulla ed il plugin termina in errore.

%--------1---------2---------3---------4---------5---------6---------7---------8
Tutti gli altri parametri sono forniti nei file di configurazione.
