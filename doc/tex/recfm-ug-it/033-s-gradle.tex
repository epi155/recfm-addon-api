\chapter{Gradle plugin}\label{sec:gradle}
%--------1---------2---------3---------4---------5---------6---------7---------8
Il gradle plugin \verb!recfm-gradle-plugin! permette di creare più classi e 
interfacce usando uno più file di configurazione in formato yaml.
Le librerie esterne utilizzate richiedono il java 8, quindi per eseguire questo 
plugin è necessario almeno il java 8.

%--------1---------2---------3---------4---------5---------6---------7---------8
Il gradle plugin \verb!recfm-gradle-plugin! è semplicemente l'\,adattamento del 
\verb!recfm-maven-plugin! per essere usato in un progetto \textsl{gradle}.
Se si esclude il modo in cui vengono passati i parametri di input, e come viene
attivato il plugin, il codice è la copia della versione per maven.

\ifesource
\begin{figure*}[!htb]
\begin{lstlisting}[language=java, caption=parametri impostabili del gradle plugin, 
label=lst:gradle.conf]
@Data
public class RecordFormatExtension {
    private String generateDirectory; // default: "${project.buildDir}/generated-sources/recfm"
    private String settingsDirectory; // default: "${project.projectDir}/src/main/resources"
    private String[] settings;
    private boolean addCompileSourceRoot = true;
    private boolean addTestCompileSourceRoot = false;
    private String codeProviderClassName;
}
\end{lstlisting}
\end{figure*}
\else
\begin{elisting}[!htb]
\begin{javacode}
@Data
public class RecordFormatExtension {
    private String generateDirectory; // default: "${project.buildDir}/generated-sources/recfm"
    private String settingsDirectory; // default: "${project.projectDir}/src/main/resources"
    private String[] settings;
    private boolean addCompileSourceRoot = true;
    private boolean addTestCompileSourceRoot = false;
    private String codeProviderClassName;
}
\end{javacode}
\caption{parametri impostabili del gradle plugin}
\label{lst:gradle.conf}
\end{elisting}
\fi
%--------1---------2---------3---------4---------5---------6---------7---------8
Come si vede col confronto del cod.~\ref{lst:maven.conf} i parametri utilizzati
sono gli stessi del maven-plugin, per la descrizione dei parametri rimando
al \S~\ref{sec:maven}.

\begin{elisting}[!htb]
\begin{minted}[frame=single,framesep=2mm,bgcolor=bg,fontsize=\footnotesize]{groovy}
buildscript {
    dependencies {
        // plugin
        classpath 'io.github.epi155:recfm-gradle-plugin:0.7.0'
        // addon for java code generation
        classpath 'io.github.epi155:recfm-java-addon:0.7.0'
    }
}
apply plugin: 'io.github.epi155.recfm-gradle-plugin'
recfm {
    settings = [ 'refcm-foo.yaml', 'refcm-bar.yaml' ]
}
\end{minted}
\caption{esempio minimale di esecuzione del plugin}
\label{lst:grd-xmpl}
\end{elisting}
