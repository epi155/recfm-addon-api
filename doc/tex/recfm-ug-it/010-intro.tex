\section*{Introduzione}
%--------1---------2---------3---------4---------5---------6---------7---------8
In alcune occasioni può capitare di avere a che fare con file (o aree di 
memoria) posizionali, vedi fig.~\ref{fig:str.data}, in questi casi è necessario 
perdere un sacco di tempo per fare una classe dedicata a ogni stringa-dati con 
i setter e getter per leggere e scrivere i valori\footnote{%
esiste \texttt{com.ancientprogramming.fixedformat4j:fixedformat4j} che 
fornisce alcune funzionalità base, ma in molte situazioni non è sufficientemente
flessibile.
}. 

\begin{figure}[!htb]
\centering\small
\scalebox{0.7}[1]{
\texttt{%
\fbox[lb]{S}%01
\fbox[lb]{C}%02
\fbox[lb]{A}%03
\fbox[lb]{R}%04
\fbox[lb]{L}%05
\fbox[lb]{E}%06
\fbox[lb]{T}%07
\fbox[lb]{T}%08
\fbox[lb]{ }%09
\fbox[lb]{ }%10
\fbox[lb]{ }%11
\fbox[lb]{ }%12
\fbox[lb]{ }%13
\fbox[lb]{ }%14
\fbox[lb]{ }%15
\fbox[lb]{J}%16
\fbox[lb]{O}%17
\fbox[lb]{H}%18
\fbox[lb]{A}%19
\fbox[lb]{N}%20
\fbox[lb]{S}%21
\fbox[lb]{S}%22
\fbox[lb]{O}%23
\fbox[lb]{N}%24
\fbox[lb]{ }%25
\fbox[lb]{ }%26
\fbox[lb]{ }%27
\fbox[lb]{ }%28
\fbox[lb]{ }%29
\fbox[lb]{ }%30
\fbox[lb]{1}%31
\fbox[lb]{9}%32
\fbox[lb]{8}%33
\fbox[lb]{4}%34
\fbox[lb]{1}%35
\fbox[lb]{1}%36
\fbox[lb]{2}%37
\fbox[lb]{2}%38
\fbox[lb]{N}%39
\fbox[lb]{E}%40
\fbox[lb]{W}%31
\fbox[lb]{ }%32
\fbox[lb]{Y}%33
\fbox[lb]{O}%34
\fbox[lb]{R}%35
\fbox[lb]{K}%36
\fbox[lb]{ }%37
\fbox[lb]{ }%38
\fbox[lb]{ }%39
\fbox[lb]{ }%40
\fbox[lb]{ }%41
\fbox[lb]{ }%42
\fbox[lb]{U}%43
\fbox[lb]{S}%44
\fbox[lbr]{A}%45
}}


\iffalse
\scalebox{0.7}[1]{
\texttt{%
\fbox[lb]{O}%01
\fbox[lb]{L}%02
\fbox[lb]{G}%03
\fbox[lb]{A}%04
\fbox[lb]{ }%05
\fbox[lb]{ }%06
\fbox[lb]{ }%07
\fbox[lb]{ }%08
\fbox[lb]{ }%09
\fbox[lb]{ }%10
\fbox[lb]{ }%11
\fbox[lb]{ }%12
\fbox[lb]{ }%13
\fbox[lb]{ }%14
\fbox[lb]{ }%15
\fbox[lb]{K}%16
\fbox[lb]{U}%17
\fbox[lb]{R}%18
\fbox[lb]{Y}%19
\fbox[lb]{L}%20
\fbox[lb]{E}%21
\fbox[lb]{N}%22
\fbox[lb]{K}%23
\fbox[lb]{O}%24
\fbox[lb]{ }%25
\fbox[lb]{ }%26
\fbox[lb]{ }%27
\fbox[lb]{ }%28
\fbox[lb]{ }%29
\fbox[lb]{ }%30
\fbox[lb]{1}%31
\fbox[lb]{9}%32
\fbox[lb]{7}%33
\fbox[lb]{9}%34
\fbox[lb]{1}%35
\fbox[lb]{1}%36
\fbox[lb]{1}%37
\fbox[lb]{4}%38
\fbox[lb]{B}%39
\fbox[lb]{E}%40
\fbox[lb]{R}%31
\fbox[lb]{D}%32
\fbox[lb]{J}%33
\fbox[lb]{A}%34
\fbox[lb]{N}%35
\fbox[lb]{S}%36
\fbox[lb]{'}%37
\fbox[lb]{K}%38
\fbox[lb]{ }%39
\fbox[lb]{ }%40
\fbox[lb]{ }%41
\fbox[lb]{ }%42
\fbox[lb]{U}%43
\fbox[lb]{K}%44
\fbox[lbr]{R}%45
}}
\fi

\scalebox{0.7}[1]{
\texttt{%
\fbox[lb]{A}%01
\fbox[lb]{N}%02
\fbox[lb]{A}%03
\fbox[lb]{ }%04
\fbox[lb]{ }%05
\fbox[lb]{ }%06
\fbox[lb]{ }%07
\fbox[lb]{ }%08
\fbox[lb]{ }%09
\fbox[lb]{ }%10
\fbox[lb]{ }%11
\fbox[lb]{ }%12
\fbox[lb]{ }%13
\fbox[lb]{ }%14
\fbox[lb]{ }%15
\fbox[lb]{D}%16
\fbox[lb]{E}%17
\fbox[lb]{ }%18
\fbox[lb]{A}%19
\fbox[lb]{R}%20
\fbox[lb]{M}%21
\fbox[lb]{A}%22
\fbox[lb]{S}%23
\fbox[lb]{ }%24
\fbox[lb]{ }%25
\fbox[lb]{ }%26
\fbox[lb]{ }%27
\fbox[lb]{ }%28
\fbox[lb]{ }%29
\fbox[lb]{ }%30
\fbox[lb]{1}%31
\fbox[lb]{9}%32
\fbox[lb]{8}%33
\fbox[lb]{8}%34
\fbox[lb]{0}%35
\fbox[lb]{4}%36
\fbox[lb]{3}%37
\fbox[lb]{0}%38
\fbox[lb]{H}%39
\fbox[lb]{A}%40
\fbox[lb]{V}%31
\fbox[lb]{A}%32
\fbox[lb]{N}%33
\fbox[lb]{A}%34
\fbox[lb]{ }%35
\fbox[lb]{ }%36
\fbox[lb]{ }%37
\fbox[lb]{ }%38
\fbox[lb]{ }%39
\fbox[lb]{ }%40
\fbox[lb]{ }%41
\fbox[lb]{ }%42
\fbox[lb]{C}%43
\fbox[lb]{U}%44
\fbox[lbr]{B}%45
}}

\caption{Esempio di file-dati posizionale} 
\label{fig:str.data}
\end{figure}



%--------1---------2---------3---------4---------5---------6---------7---------8
Questo gruppo di programmi si propone di minimizzare il tempo per creare queste
classi. In pratica viene definita la struttura della stringa-dati con un file di
configurazione, questo viene dato in pasto ad un plugin che genera la 
classe-dati corrispondente, che può essere utilizzata senza nessun ulteriore
intervento utente.

%--------1---------2---------3---------4---------5---------6---------7---------8
I programmi sono strutturati usando service provider interface, 
vedi fig.~\ref{fig:spi}, abbiamo un plugin, o un programma utente 
(\textsl{Service}), che vede direttamente le classi definite nella 
\textsl{Service Provider Interface} e recupera la implementazione usando il 
\textsl{ServiceLoader}, in questo modo non ha una dipendenza specifica con una
delle implementazioni usate. 
Il \textsl{Service Provider} deve implementare le classi definite nella 
\textsl{Service Provider Interface}.

\begin{figure}[!htb]
\centering
\begin{tikzpicture}[>=latex,font={\sf}]
\node(u1) at (0,1.5) [manual input,text width=2cm,fill=blue!10]{maven plugin};
\node(u2) at (3,1.5) [manual input,text width=2cm,fill=blue!10]{gradle plugin};
\node(u3) at (6,1.5) [manual input,text width=2cm,fill=blue!10]{custom service};
\node(si) at (3,0.0) [preparation,fill=yellow!20]{addon-api};
\node(a1) at (0,-1.5) [process,text width=1.7cm,fill=green!10]{java addon};
\node(a2) at (3,-1.5) [process,text width=1.7cm,fill=green!10]{scala addon};
\node(a3) at (6,-1.5) [process,text width=1.7cm,fill=green!10]{custom provider};

\node at (9,1.5) {Service};
\node at (9,0.0) {Service Provider Interface};
\node at (9,-1.5) {Service Provider};

\draw[arrow] (u1) -- (si.north);
\draw[arrow] (u2) -- (si.north);
\draw[arrow] (u3) -- (si.north);
\draw[arrow] (a1) -- (si.south);
\draw[arrow] (a2) -- (si.south);
\draw[arrow] (a3) -- (si.south);

\end{tikzpicture}
\caption{Struttura service, service-provider-interface, service-provider} 
\label{fig:spi}
\end{figure}

%--------1---------2---------3---------4---------5---------6---------7---------8
Se il \verb!maven-plugin! trova in esecuzione la libreria con 
l'\,implementazione \verb!java-addon! genererà i sorgenti in java, ma se trova
l'\,implementazione \verb!scala-addon! genererà i sorgenti in scala.

%--------1---------2---------3---------4---------5---------6---------7---------8
La documentazione è divisa in tre parti. Nella prima, \ref{vol:spi}, viene data 
una descrizione dettagliata delle classi definite nella 
\textsl{service provider interface}, questa parte è utile per chi volesse 
sviluppare un \textit{custom service} o un \textit{custom provider}. 
Se si è interessati solo a come generare il codice partendo dai file di 
configurazione può essere saltata.

%--------1---------2---------3---------4---------5---------6---------7---------8
Nella seconda parte, \ref{vol:srv}, viene data una descrizione di due plugin 
usati generare il codice.
In particolare come difinire i tracciati con i file di configurazione e come 
attivare il plugin.

%--------1---------2---------3---------4---------5---------6---------7---------8
Nella terza parte, \ref{vol:sp}, viene data una descrizione del \textsl{service 
provider} che genera il sorgente java mostrando anche alcune funzionalità 
aggiuntive delle classi generate oltre ai semplici setter e getter.
