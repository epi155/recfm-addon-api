\section{Generazione sorgente java --- \texttt{java-addon}} 
\label{sec:java.addon}
%--------1---------2---------3---------4---------5---------6---------7---------8
Le classi generate dal \textsl{CodeProvider} java oltre ai setter e getter
hanno una serie di metodi ausiliari, vedi cod.~\ref{lst:Foo312:java}.


\ifesource
\begin{figure*}[!htb]
\begin{lstlisting}[language=java, caption=esempio di classe generata (Foo312), 
label=lst:Foo312:java]
public class Foo312 extends FixRecord {
    public Foo312() { /* ... */ }
    public static Foo312 of(FixRecord r) { /* ... */ }
    public static Foo312 decode(String s) { /* ... */ }
    public Foo312 copy() { /* ... */ }
    // setter and getter ...
    public boolean validateFails((*\hyperref[lst:FieldValidateHandler:java]{FieldValidateHandler}*) handler) { /* ... */ }
    public boolean validateAllFails((*\hyperref[lst:FieldValidateHandler:java]{FieldValidateHandler}*) handler) { /* ... */ }
    public String toString() { /* ... */ }
    public String encode() { /* ... */ }    // from super class
    public int length() { return LRECL; }   // string-data length
}
\end{lstlisting}
\end{figure*}
\else
\begin{elisting}[!htb]
\begin{javacode}
public class Foo312 extends FixRecord {
    public Foo312() { /* ... */ }
    public static Foo312 of(FixRecord r) { /* ... */ }
    public static Foo312 decode(String s) { /* ... */ }
    public Foo312 copy() { /* ... */ }
    // setter and getter ...
    public boolean validateFails(|\hyperref[lst:FieldValidateHandler:java]{FieldValidateHandler}| handler) { /* ... */ }
    public boolean validateAllFails(|\hyperref[lst:FieldValidateHandler:java]{FieldValidateHandler}| handler) { /* ... */ }
    public String toString() { /* ... */ }
    public String encode() { /* ... */ }    // from super class
    public int length() { return LRECL; }   // string-data length
}
\end{javacode}
\caption{esempio di classe generata (Foo312)}
\label{lst:Foo312:java}
\end{elisting}
\fi

%--------1---------2---------3---------4---------5---------6---------7---------8
\begin{itemize}
\item viene fornito un costruttore senza argomenti, che crea la classe con i 
    valori a default
\item viene fornito un costruttore \textit{cast-like}, che prende come argomento
    qualunque altra classe che rappresenta una classe-dati
\item viene fornito un costruttore da stringa-dati (de-serializzatore)
\item viene fornito un metodo \textit{deep-copy} per duplicare la classe-dati
\item vengono forniti due metodo di validazione
\item viene fornito un metodo di \texttt{toString}
\item viene fornito un metodo per generare la stringa-dati (serializzatore)
\item viene fornito un metodo per recuperare la lunghezza della stringa-dati
\end{itemize}

%--------1---------2---------3---------4---------5---------6---------7---------8
Le classi generate dai file di configurazione ereditano delle classi generali
con metodi comuni per la gestione dei setter/getter dei controlli e le 
validazioni. Queste classi sono fornite come libreria esterna, 
vedi~\ref{lst:mvn-j-deps}.

\begin{elisting}[!htb]
\begin{xmlcode}
<dependencies>
  <dependency>
    <groupId>io.github.epi155</groupId>
    <artifactId>recfm-java-lib</artifactId>
    <version>0.7.0</version>
  </dependency>
</dependencies>
\end{xmlcode}
\caption{dipendenze runtime dell'\,addon-java}
\label{lst:mvn-j-deps}
\end{elisting}

%--------1---------2---------3---------4---------5---------6---------7---------8
Queste librerie sono compilate in compatibilità java-5, e contangono il
\textsl{module-info} per poter essere correttamente gestite anche con java-9 e
superiori.

\subsection{Validazione}
%--------1---------2---------3---------4---------5---------6---------7---------8
Come visto nel sorgente~\ref{lst:Foo312:java} per ogni classe vengono generati
due metodi di validazione. Per entrambi l'\,argo\-men\-to è una interfaccia 
dedicata, ma questo è per compatibilità pre-java-8.
L'\,argometo sarà una \textit{closure}, implementata con una classe anonima o 
interna o una $\lambda$-function.

\ifesource
\begin{figure*}[!htb]
\begin{lstlisting}[language=java, 
caption={gestore errori \texttt{FieldValidateHandler}}, 
label=lst:FieldValidateHandler:java]
public interface FieldValidateHandler {
    void error((|hyperref[lst:FieldValidateError:java]{FieldValidateError}| fieldValidateError);
}
\end{lstlisting}
\end{figure*}
\else
\begin{elisting}[!htb]
\begin{javacode}
public interface FieldValidateHandler {
    void error(|\hyperref[lst:FieldValidateError:java]{FieldValidateError}| fieldValidateError);
}
\end{javacode}
\caption{gestore errori \texttt{FieldValidateHandler}}
\label{lst:FieldValidateHandler:java}
\end{elisting}
\fi

%--------1---------2---------3---------4---------5---------6---------7---------8
I metodi di validazione indicano se la stringa-dati
acquisita con il costruttore statico \texttt{decode} ha superato la validazione
richiesta dalla definizione dei campi oppure no, ma ogni volta che viene 
rilevato un errore di validazione viene chiamato il metodo \texttt{error}
dell'\,interfaccia fornita come argomento con i dettagli dell'\,errore.
In questo modo è possibile accumulare tutti gli errori di validazione rilevati.
%--------1---------2---------3---------4---------5---------6---------7---------8
La differenza tra i due metodi è che il primo \texttt{validateFails} in caso di
più caratteri errati sullo stesso campo, segnala solo il primo, mentre il 
secondo \texttt{validateAllFails} segnala tutti i caratteri in errore.

\ifesource
\begin{figure*}[!htb]
\begin{lstlisting}[language=java, 
caption={dettaglio errore \texttt{FieldValidateError}}, 
label=lst:FieldValidateError:java]
public interface FieldValidateError {
    String name();          // field name in error
    int offset();           // field offset in error
    int length();           // field length in error
    String value();         // field value in error
    Integer column();       // column of the record with the wrong character
    (*\hyperref[lst:ValidateError:java]{ValidateError}*) code();   // error category
    Character wrong();      // wrong character
    String message();       // field message error
}
\end{lstlisting}
\end{figure*}
\else
\begin{elisting}[!htb]
\begin{javacode}
public interface FieldValidateError {
    String name();          // field name in error
    int offset();           // field offset in error
    int length();           // field length in error
    String value();         // field value in error
    Integer column();       // column of the record with the wrong character
    |\hyperref[lst:ValidateError:java]{ValidateError}| code();   // error category
    Character wrong();      // wrong character
    String message();       // field message error
}
\end{javacode}
\caption{dettaglio errore \texttt{FieldValidateError}}
\label{lst:FieldValidateError:java}
\end{elisting}
\fi

%--------1---------2---------3---------4---------5---------6---------7---------8
L'\,argomento del metodo \texttt{error} è l'\,interfaccia
\texttt{FieldValidateError}, che sostanzialmente è una java-bean che espone solo
i getter in formato \textit{fluente}.
%--------1---------2---------3---------4---------5---------6---------7---------8
Segnaliamo che alcuni valori possono essere \texttt{null}.
Un campo di tipo costante non ha un nome (\texttt{name}). 
Un campo di tipo costante o dominio o custom con un controllo impostato con 
espressione regolare non ha un carattere sbagliato (\texttt{wrong})
ad una ben precisa colonna (\texttt{column}).
Nel messaggio di errore, se è possibile identificare il carattere in
errore, viene mostrata la posizione del carattere relativa al campo (non alla 
stringa-dati), il carattere (se è un carattere di controllo viene mostrata la
codifica unicode), il \textit{nome} del carattere, e il tipo di errore; 
altrimenti viene mostrato il valore del campo e il tipo di errore.

\ifesource
\begin{figure*}[!htb]
\begin{lstlisting}[language=java, 
caption={categoria errore \texttt{ValidateError}}, 
label=lst:ValidateError:java]
public enum ValidateError {
    NotNumber, NotAscii, NotLatin, NotValid, NotDomain, NotBlank, NotEqual, NotMatch, NotDigitBlank
}
\end{lstlisting}
\end{figure*}
\else
\begin{elisting}[!htb]
\begin{javacode}
public enum ValidateError {
    NotNumber, NotAscii, NotLatin, NotValid, NotDomain, NotBlank, NotEqual, NotMatch, NotDigitBlank
}
\end{javacode}
\caption{categoria errore \texttt{ValidateError}}
\label{lst:ValidateError:java}
\end{elisting}
\fi

%--------1---------2---------3---------4---------5---------6---------7---------8
I possibili tipi di errore sono mostrati nel 
sorgente~\ref{lst:ValidateError:java}, il significato è evidente dal nome.

\subsection{Setter e getter}
%--------1---------2---------3---------4---------5---------6---------7---------8
L'\,implementazione della classe-dati usata da questa libreria, in realtà non 
genera dei campi de-serializzati. 
Quando la classe viene creata dalla stringa-dati, il costruttore statico si 
limita a salvare internamente la stringa-dati come vettore di caratteri. 
Il getter di un campo accede all'\,intervallo di caratteri corrispondenti al 
campo e li de-serializza. 
Analogamente il setter serializza il valore fornito e lo copia 
nell'\,intervallo di caratteri corrispondenti al campo. 
In questo modo è banale fare un \textsl{override} di un campo, e il 
costruttore \textit{cast-like} e \textit{deep-copy} sono quasi a costo zero.
Anche i metodi \textsl{encode} e \textsl{decode} sono sostanzialmente a costo 
zero perché le operazioni di serializzazione/deserializzazione sono in realtà 
eseguite dai setter/getter.

\subsection{Campi Singoli}
\subsubsection*{Gestione valore \texttt{null}}
%--------1---------2---------3---------4---------5---------6---------7---------8
In una stringa-dati non è rappresentabile un valore nullo, a meno che 
convenzionalmente si assegni ad una particolare stringa tale valore, come nei 
campi di tipo numerico-nullabile.
Quando un setter formalmente imposta il valore \texttt{null}, nella 
rappresentazione della stringa-dati in realtà verrà assegnato il valore di
default del campo: spazio per un tipo alfanumerico, zero per un tipo numerico,
lo \texttt{initChar} per un tipo custom, e il primo valore tra quelli definiti 
come possibili per un campo dominio.

\begin{elisting}[!htb]
\begin{javacode}
    String getValue() { /* ... */ }   // string getter
    int intValue() { /* ... */ }      // int getter
\end{javacode}
\caption{Accesso a volori numerici come stringe e numeri primitivi}
\label{lst:num.acc.both}
\end{elisting}

\subsubsection*{Accesso \texttt{Both} per campi numerici}
%--------1---------2---------3---------4---------5---------6---------7---------8
I campi numerici possono essere gestiti come stringhe di caratteri numerici o
come oggetti numerici primitivi. È possibile configurare i campi per avere 
getter/setter di tipo stringa o numerico primitivo o entrambi.
Nel caso venga scelto ``entrambi'' non è possibile definire il getter con il
nome canonico per entrambi i tipi.
In questi casi il nome canonico viene usato per il getter di tipo stringa,
il getter con il tipo primitivo ha come nome il tipo primitivo e il nome del 
campo, vedi~\ref{lst:num.acc.both}.

\begin{elisting}[!htb]
\begin{quote}
\begin{minted}[frame=single,framesep=2mm,bgcolor=bg,fontsize=\footnotesize,linenos,firstnumber=200]{java}
    @Test
    void testDomain() {
        BarDom dom = new BarDom();
        dom.setCur("AAA");
    }
\end{minted}
\end{quote}
\vspace*{-1cm}
\begin{javacode}
io.github.epi155.recfm.java.NotDomainException: com.example.BarDom.setCur, offending value "AAA"
	at com.example.test.TestBar.testDomain(TestBar.java:203)
	...
\end{javacode}
\caption{Eccezione sul setter}
\label{lst:set.throw}
\end{elisting}

\subsubsection*{Controlli sui setter e getter}
%--------1---------2---------3---------4---------5---------6---------7---------8
Se sono attivi i controlli sui setter e viene impostato un valore non permesso 
viene lanciata una eccezione che segnala la violazione del controllo.
L'\,eccezione posiziona lo stacktrace sulla istruzione del setter.

\begin{elisting}[!htb]
\begin{quote}
\begin{minted}[frame=single,framesep=2mm,bgcolor=bg,fontsize=\footnotesize,linenos,firstnumber=300]{java}
    @Test
    void testDomain() {
        BarDom d1 = BarDom.decode("AAA");
        String cur = d1.getCur();
    }
\end{minted}
\end{quote}
\vspace*{-1cm}
\begin{javacode}
io.github.epi155.recfm.java.NotDomainException: com.example.BarDom.getCur, offending value "AAA" @1+3
	at com.example.test.TestBar.testDomain(TestBar.java:303)
	...
\end{javacode}
\caption{Eccezione sul getter}
\label{lst:get.throw}
\end{elisting}
%--------1---------2---------3---------4---------5---------6---------7---------8
Analogamente sui getter. Se la stringa-dati contiene nella zona corrispondente 
a un campo un valore non valido per il campo, non viene fatta la validazione 
della struttura, che avrebbe segnalato il problema, e il codice continua fino 
al getter, viene lanciata una eccezione.
L'\,eccezione posiziona lo stacktrace sulla istruzione del getter.

\subsection{Campi Multipli}
%--------1---------2---------3---------4---------5---------6---------7---------8
In questo contesto considereremo campi multipli solo i campi di tipo gruppo o
gruppo ripetuto, definiti direttamente o tramite interfaccia.
Questo tipo di campi genera un elemento intermedio.
Come si vede dal sorgente~\ref{lst:grp.indef}, generato per un gruppo definito
da interfaccia, viene creata una classe interna per gestire l'\,elemento 
intermedio, un campo privato con una istanza dell'\,elemento intermedio, un
\textit{getter fluente} del campo, e un \textit{Consumer} del campo.

\begin{elisting}[!htb]
\begin{javacode}
    public class StopTime implements Validable, ITime {/* ... */}
    private final StopTime stopTime = this.new StopTime();
    public StopTime stopTime() { return this.stopTime; }
    public void withStopTime(WithAction<StopTime> action) { action.accept(this.stopTime); }
\end{javacode}
\caption{Definizione di un gruppo interno alla classe-dati}
\label{lst:grp.indef}
\end{elisting}

%--------1---------2---------3---------4---------5---------6---------7---------8
La classe interna implementerà l'\,interfaccia di validazione, e, se definito
tramite interfaccia, l'\,interfaccia con la definizione del dettaglio dei campi
della classe interna.
Ogni gruppo è validabile singolarmente come se fosse una classe-dati.
L'\,interfaccia di validazione, \texttt{Validable}, richiede il metodo
\texttt{validateFails} che abbiamo già incontrato nella validazione della
classe-dati. Anche tutte le classi-dati implementano l'\,interfaccia 
\texttt{Validable}.

\begin{elisting}[!htb]
\begin{javacode}
public interface Validable {
    boolean validateFails(FieldValidateHandler handler);
    boolean validateAllFails(FieldValidateHandler handler);
}
\end{javacode}
\caption{Interfaccia di validazione, a livello classe-dati e gruppo}
\label{lst:if.validable}
\end{elisting}

%--------1---------2---------3---------4---------5---------6---------7---------8
La definizione di un gruppo ripetuto è simile a qullo di un gruppo.
Viene creata una classe interna per gestire l'\,elemento 
intermedio ripetuto, un campo privato con $n$ istanze dell'\,elemento 
intermedio, un \textit{getter fluente} con un indice del campo, 
e un \textit{Consumer} con indice del campo.

\begin{elisting}[!htb]
\begin{javacode}
    public class TabError implements Validable, IError {/* ... */}
    private final TabError[] tabError = new TabError[] {
        this.new TabError(0),
        /* ... */
    };
    public TabError tabError(int k) { return this.tabError[k-1]; }
    public void withTabError(int k, WithAction<TabError> action) { action.accept(this.tabError[k-1]); }
\end{javacode}
\caption{Definizione di un gruppo ripetuto interno alla classe-dati}
\label{lst:occ.indef}
\end{elisting}

%--------1---------2---------3---------4---------5---------6---------7---------8
Anche in questo caso la classe interna che definisce l'\,elemento intermedio 
implementa l'\,interfaccia di validazione, e, se definito tramite interfaccia, 
l'\,interfaccia con la definizione del dettaglio dei campi della classe interna.



\iffalse
\begin{elisting}[!htb]
\begin{javacode}
    Foo311b foo = new Foo311b();
    foo.stopTime().setHours("01");
\end{javacode}
\caption{Setter di un campo all'\,interno di un gruppo}
\label{lst:grp.set}
\end{elisting}

\begin{elisting}[!htb]
\begin{javacode}
    Foo311b foo = new Foo311b();
    foo.withStopTime(stop -> {
        stop.setHours("01");
    });
\end{javacode}
\caption{Setter di un campo all'\,interno di un gruppo usando $\lambda$-function}
\label{lst:grp.withLambda}
\end{elisting}

\begin{elisting}[!htb]
\begin{javacode}
    Foo311b foo = new Foo311b();
    foo.withStopTime(new WithAction<Foo311b.StopTime>() {
        @Override
        public void call(Foo311b.StopTime stop) {
            stop.setHours("01");
        }
    });
\end{javacode}
\caption{Setter di un campo all'\,interno di un gruppo usando una funzione anonima}
\label{lst:grp.withAnonym}
\end{elisting}

\begin{elisting}[!htb]
\begin{javacode}
    Foo311b foo = new Foo311b();
    String hours = foo.stopTime().getHours();
\end{javacode}
\caption{Getter di un campo all'\,interno di un gruppo}
\label{lst:grp.get}
\end{elisting}

\begin{elisting}[!htb]
\begin{javacode}
    Foo311b foo = new Foo311b();
    boolean test = foo.validateFails(x -> System.out.println(x.message()));
\end{javacode}
\caption{Validazione di un gruppo}
\label{lst:grp.validate}
\end{elisting}
\fi
