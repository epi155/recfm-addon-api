\chapter{Service Provider}
%--------1---------2---------3---------4---------5---------6---------7---------8
Nei capitoli precedenti abbiamo visto la \textsl{Service Provider Interface}, 
che  definisce delle interfacce e delle classi che permettono di definire i 
traccciati, e indicare alcuni comportamenti che dovranno essere usati in fase di
utilizzazione dei tracciati; e alcuni esempi di \textsl{Service}, che
semplicemente valorizza gli oggetti messi a disposizione della 
\textsl{Service Provider Interface}, ma il vero lavoro di generazione del codice 
è fatto dal \textsl{Service Provider}.

%--------1---------2---------3---------4---------5---------6---------7---------8
La struttuta SPI consente di avere codice generato diverso, implementato in modo
diverso, o addirittura generare sorgente in un linguaggio diverso.

%--------1---------2---------3---------4---------5---------6---------7---------8
Qualunque sia il linguaggio generato e il dettaglio della implementazione il
\textsl{Service Provider} dovrà fornire alcune funzionalità generali.

\begin{itemize}
%--------1---------2---------3---------4---------5---------6---------7---------8
\item \textbf{decode}: partendo dalla stringa-dati, deve instanziare la 
    classe-dati;
\item \textbf{setter, getter}: la classe-dati generata deve fornire i metodi di 
    accesso ai singoli campi;
\item \textbf{costruttore vuoto}: la classe-dati può essere instanziata con i 
    valori di default dei campi;
\item \textbf{encode}: la classe-dati può essere serializzata nella 
    stringa-dati.
\end{itemize}
%--------1---------2---------3---------4---------5---------6---------7---------8
Sarebbe gradita anche qualche funzionalità accessoria:
\begin{itemize}
\item \textbf{validate}: validare la stringa-dati prima della 
    de-serializzazione, in modo da segnalare tutte le aree che non possono essere
    assegnate ai relativi campi, tipicamente caratteri non numerici in campi di
    tipo numerico;
\item \textbf{cast}: se due stringhe-dati hanno la stessa lunghezza, poter 
    passare da una classe-dati che le rappresenta all'\,altra;
\item \textbf{toString}: fornire un metodo che mostra tutti i valori dei campi 
    che compongono la classe-dati.
\item (deep) \textbf{copy}: genera una copia della classe-dati;
\end{itemize}


