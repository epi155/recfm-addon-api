\documentclass[a4paper,10pt]{report}

\usepackage[italian]{babel}     % sillabazione italiana
\usepackage[body={17.8cm,24.7cm}]{geometry}

\usepackage[utf8x]{inputenc}
\usepackage[T1]{fontenc}
\usepackage{pifont}
\usepackage{array,multirow}

\usepackage{makeidx}
\usepackage{tocloft}

\usepackage{eflowchart}	% custom style

\newif\ifesource
%\esourcetrue
\ifesource
\usepackage{esourcecode}	% custom style
\else
\usepackage{xcolor,colortbl}
%\usepackage[outputdir={/tmp}]{minted}
\usepackage{minted}
\definecolor{bg}{rgb}{0.95,0.95,0.92}
\usepackage{caption}
\newenvironment{elisting}[1][!htb]
%  {\captionsetup{aboveskip=-0.1\normalbaselineskip}\begin{listing}[#1]}
  {\captionsetup{aboveskip=0pt}\begin{listing}[#1]}
  {\end{listing}%
}
\newminted{java}{
  frame=single,
  framesep=2mm,
  bgcolor=bg,
  fontsize=\footnotesize,
  escapeinside=||,
    tabsize=4,
%    fontfamily=courier,
    autogobble,
%    xleftmargin=5pt,
%    xrightmargin=5pt,
%    frame=lines,
}
\newminted{yaml}{
  frame=single,
  framesep=2mm,
  bgcolor=bg,
  fontsize=\footnotesize,
    tabsize=4,
%    fontfamily=courier,
    autogobble,
%    xleftmargin=5pt,
%    xrightmargin=5pt,
%    frame=lines,
}
\fi

%\usepackage[pdftex]{graphicx}

\usepackage[pdftex,
%pdftitle={Graphics and Color with LaTeX},
%pdfauthor={Patrick W Daly},
%pdfsubject={Importing images and use of color in LaTeX},
%pdfkeywords={LaTeX, graphics, color},
pdfpagemode=UseOutlines,
bookmarks,bookmarksopen,
pdfstartview=FitH,
colorlinks,linkcolor=blue,citecolor=blue,
urlcolor=green,
]
{hyperref}

\renewcommand{\arraystretch}{1.2}

\usepackage{tikz}
%\usetikzlibrary{shapes,arrows}
%\usetikzlibrary{arrows,shapes,snakes,automata,backgrounds,petri}
\usetikzlibrary{arrows,shapes,decorations,automata,backgrounds,petri}

%\usepackage{bera}
%\usepackage{beraserif}
%\usepackage{berasans}
%\usepackage[scaled=0.9]{beramono}
\usepackage{newcent}
\usepackage[scaled=0.9]{beramono}
%\usepackage{bookman}
%\usepackage{dejavu}
%\usepackage{roboto}

\title{RECFM}
\author{E.~Pistolesi}

\makeindex
\setcounter{secnumdepth}{3}
\setcounter{tocdepth}{3}	% default is 2

\cftsetindents{section}{0.5in}{0.5in}
\cftsetindents{subsection}{0.5in}{0.6in}
\cftsetindents{subsubsection}{0.5in}{0.7in}
%\cftsetindents{paragraph}{0.5in}{0.5in}
\cftsetindents{table}{0.25in}{0.5in}

\ifesource
\cftsetindents{listings}{0.25in}{0.5in}
\renewcommand{\lstlistlistingname}{Elenco dei sorgenti}
\else
\renewcommand{\listingscaption}{Sorgente}
\renewcommand{\listoflistingscaption}{Elenco dei sorgenti}
%\captionsetup[listing]{aboveskip=-50pt,belowskip=-20pt}
\fi


%--- begin - document ---------------------------------------------------------
\begin{document}

\maketitle

\begin{abstract}
Spesso può capitare di avere a che fare con file (o aree di memoria)
posizionali, in questi casi è necessario perdere un sacco di tempo per fare una
classe dedicata a ogni tracciato con i setter e getter per leggere e scrivere
i valori. 
Questo gruppo di programmi si propone di minimizzare il tempo per creare queste
classi.

\begin{figure}[!htb]
\centering
\begin{tikzpicture}[>=latex,font={\sf}]
\node(u1) at (0,1.5) [manual input,text width=2cm,fill=blue!10]{maven plugin};
\node(u2) at (3,1.5) [manual input,text width=2cm,fill=blue!10]{gradle plugin};
\node(u3) at (6,1.5) [manual input,text width=2cm,fill=blue!10]{custom service};
\node(si) at (3,0.0) [preparation,fill=yellow!20]{addon-api};
\node(a1) at (0,-1.5) [process,text width=1.7cm,fill=green!10]{java addon};
\node(a2) at (3,-1.5) [process,text width=1.7cm,fill=green!10]{scala addon};
\node(a3) at (6,-1.5) [process,text width=1.7cm,fill=green!10]{custom provider};

\node at (9,1.5) {Service};
\node at (9,0.0) {Service Provider Interface};
\node at (9,-1.5) {Service Provider};

\draw[arrow] (u1) -- (si.north);
\draw[arrow] (u2) -- (si.north);
\draw[arrow] (u3) -- (si.north);
\draw[arrow] (a1) -- (si.south);
\draw[arrow] (a2) -- (si.south);
\draw[arrow] (a3) -- (si.south);

\end{tikzpicture}
\caption{Struttura service, service-provider-interface, service-provider} 
\label{fig:spi}
\end{figure}

%--------1---------2---------3---------4---------5---------6---------7---------8
Il programma è strutturato usando service provider interface, 
vedi fig.~\ref{fig:spi}, abbiamo un plugin, o un programma utente 
(\textsl{Service}), che vede direttamente le classi definite nella 
\textsl{Service Provider Interface} e recupera la implementazione usando il 
\textsl{ServiceLoader}, in questo modo non ha una dipendenza specifica con una
delle implementazioni usate. 
Il \textsl{Service Provider} deve implementare le classi definite nella 
\textsl{Service Provider Interface}.

%--------1---------2---------3---------4---------5---------6---------7---------8
Se il \verb!maven-plugin! trova in esecuzione la libreria con 
l'\,implementazione \verb!java-addon! genererà i sorgenti in java, ma se trova
l'\,implementazione \verb!scala-addon! genererà i sorgenti in scala.
\end{abstract}

%\input{cover.tex}

%\clearpage

\tableofcontents
\listoffigures
\listoftables

\ifesource
\lstlistoflistings
\else
\listoflistings
\fi

\clearpage

%  ___       _             __                
% |_ _|_ __ | |_ ___ _ __ / _| __ _  ___ ___ 
%  | || '_ \| __/ _ \ '__| |_ / _` |/ __/ _ \
%  | || | | | ||  __/ |  |  _| (_| | (_|  __/
% |___|_| |_|\__\___|_|  |_|  \__,_|\___\___|
%                                           

\chapter{Service Provider Interface}
%--------1---------2---------3---------4---------5---------6---------7---------8
L'\,artifatto \verb!recfm-addon-api! mette a disposizione una serie di 
interfacce, alcuni enum e java-bean per permettere al modulo client di definire
i tracciati. 
Il codice è compilato in moda da essere compatibile con il java 5, ma fornisce
il \verb!module-info! per essere utilizzabile propriamente anche con il java 9
e superiori.

\section{CodeProvider}\index{CodeProvider}
%--------1---------2---------3---------4---------5---------6---------7---------8
Il punto di partenza è l'\,interfaccia \textsl{CodeProvider}, vedi 
cod.~\ref{lst:CodeProvider}, questa interfaccia fornisce l'\,istanza 
dell'\,in\-ter\-fac\-cia \textsl{CodeFactory}.

\ifesource
\begin{figure*}[!htb]
\begin{lstlisting}[language=java, caption=interfaccia CodeProvider, 
label=lst:CodeProviderv]
public interface CodeProvider {
     (*\hyperref[lst:CodeFactory]{CodeFactory}*) getInstance();
}
\end{lstlisting}
\end{figure*}
\else
\begin{elisting}
\begin{javacode}
public interface CodeProvider {
     |\hyperref[lst:CodeFactory]{CodeFactory}| getInstance();
}
\end{javacode}
\caption{interfaccia CodeProvider}\label{lst:CodeProvider}
\end{elisting}
\fi
%--------1---------2---------3---------4---------5---------6---------7---------8


\section{CodeFactory}
%--------1---------2---------3---------4---------5---------6---------7---------8
L\,interfaccia \textsl{CodeFactory}, vedi cod.~\ref{lst:CodeFactory}, fornisce i 
metodi per definire tutti gli elementi della struttura.

\ifesource
\begin{figure*}[!htb]
\begin{lstlisting}[language=java, caption=interfaccia CodeFactory, 
label=lst:CodeFactory]
public interface CodeFactory {
    (*\hyperref[lst:ClassModel]{ClassModel}*) newClassModel();
    (*\hyperref[lst:TraitModel]{TraitModel}*) newTraitModel();

    (*\hyperref[lst:AbcModel]{AbcModel}*) newAbcModel();
    (*\hyperref[lst:NumModel]{NumModel}*) newNumModel();
    (*\hyperref[lst:NuxModel]{NuxModel}*) newNuxModel();
    (*\hyperref[lst:CusModel]{CusModel}*) newCusModel();
    (*\hyperref[lst:DomModel]{DomModel}*) newDomModel();
    (*\hyperref[lst:FilModel]{FilModel}*) newFilModel();
    (*\hyperref[lst:ValModel]{ValModel}*) newValModel();
    (*\hyperref[lst:GrpModel]{GrpModel}*) newGrpModel();
    (*\hyperref[lst:OccModel]{OccModel}*) newOccModel();
    (*\hyperref[lst:EmbModel]{EmbModel}*) newEmbModel();
    (*\hyperref[lst:GrpTraitModel]{GrpTraitModel}*) newGrpTraitModel();
    (*\hyperref[lst:OccTraitModel]{OccTraitModel}*) newOccTraitModel();
}
\end{lstlisting}\index{CodeFactory}
\end{figure*}
\else
\begin{elisting}
\begin{javacode}
public interface CodeFactory {
    |\hyperref[lst:ClassModel]{ClassModel}| newClassModel();
    |\hyperref[lst:TraitModel]{TraitModel}| newTraitModel();

    |\hyperref[lst:AbcModel]{AbcModel}| newAbcModel();
    |\hyperref[lst:NumModel]{NumModel}| newNumModel();
    |\hyperref[lst:NuxModel]{NuxModel}| newNuxModel();
    |\hyperref[lst:CusModel]{CusModel}| newCusModel();
    |\hyperref[lst:DomModel]{DomModel}| newDomModel();
    |\hyperref[lst:FilModel]{FilModel}| newFilModel();
    |\hyperref[lst:ValModel]{ValModel}| newValModel();
    |\hyperref[lst:GrpModel]{GrpModel}| newGrpModel();
    |\hyperref[lst:OccModel]{OccModel}| newOccModel();
    |\hyperref[lst:EmbModel]{EmbModel}| newEmbModel();
    |\hyperref[lst:GrpTraitModel]{GrpTraitModel}| newGrpTraitModel();
    |\hyperref[lst:OccTraitModel]{OccTraitModel}| newOccTraitModel();
}
\end{javacode}
\caption{interfaccia CodeFactory}
\label{lst:CodeFactory}\index{CodeFactory}
\end{elisting}
\fi
%--------1---------2---------3---------4---------5---------6---------7---------8

\section{Classi / Interfacce}
%--------1---------2---------3---------4---------5---------6---------7---------8
Il primo metodo dell\,interfaccia \textsl{CodeFactory} fornisce la definizione 
per una classe, vedi cod.~\ref{lst:ClassModel}, e il secondo metodo 
dell\,interfaccia \textsl{CodeFactory} fornisce la definizione per una 
interfaccia, vedi cod.~\ref{lst:TraitModel}.


\ifesource
\begin{figure*}[!htb]
\begin{lstlisting}[language=java, caption=interfaccia ClassModel, 
label=lst:ClassModel]
public interface ClassModel {
    void setName(String name);
    void setLength(int length);
    void setOnOverflow((*\hyperref[lst:LoadOverflowAction]{LoadOverflowAction}*) onOverflow);
    void setOnUnderflow((*\hyperref[lst:LoadUnderflowAction]{LoadUnderflowAction}*) onUnderflow);
    void setDoc(Boolean doc);
    void setFields(List<FieldModel> fields);

    void create(String namespace, (*\hyperref[lst:GenerateArgs]{GenerateArgs}*) ga, (*\hyperref[lst:FieldDefault]{FieldDefault}*) defaults);
}
\end{lstlisting}\index{ClassModel}
\end{figure*}
\else
\begin{elisting}
\begin{javacode}
public interface ClassModel {
    void setName(String name);
    void setLength(int length);
    void setOnOverflow(|\hyperref[lst:LoadOverflowAction]{LoadOverflowAction}| onOverflow);
    void setOnUnderflow(|\hyperref[lst:LoadUnderflowAction]{LoadUnderflowAction}| onUnderflow);
    void setDoc(Boolean doc);
    void setFields(List<FieldModel> fields);

    void create(String namespace, |\hyperref[lst:GenerateArgs]{GenerateArgs}| ga, |\hyperref[lst:FieldDefault]{FieldDefault}| defaults);
}
\end{javacode}
\caption{interfaccia ClassModel}
\label{lst:ClassModel}\index{ClassModel}
\end{elisting}
\fi

%--------1---------2---------3---------4---------5---------6---------7---------8
Entrambe le definizioni richiedono il nome della struttura, la sua
lunghezza, indicano se generare o meno la documentazione automatica per la 
classe, l'\,elenco dei campi che la compongono e mettono a disposizione un 
metodo per generare il codice sorgente.

\ifesource
\begin{figure*}[!htb]
\begin{lstlisting}[language=java, caption=interfaccia TraitModel, 
label=lst:TraitModel]
public interface TraitModel {
    void setName(String name);
    void setLength(int length);
    void setDoc(Boolean doc);
    void setFields(List<FieldModel> fields);

    void create(String namespace, (*\hyperref[lst:GenerateArgs]{GenerateArgs}*) ga, (*\hyperref[lst:FieldDefault]{FieldDefault}*) defaults);
}
\end{lstlisting}\index{TraitModel}
\end{figure*}
\else
\begin{elisting}
\begin{javacode}
public interface TraitModel {
    void setName(String name);
    void setLength(int length);
    void setDoc(Boolean doc);
    void setFields(List<FieldModel> fields);

    void create(String namespace, |\hyperref[lst:GenerateArgs]{GenerateArgs}| ga, |\hyperref[lst:FieldDefault]{FieldDefault}| defaults);
}
\end{javacode}
\caption{interfaccia TraitModel}
\label{lst:TraitModel}\index{TraitModel}
\end{elisting}
\fi
%--------1---------2---------3---------4---------5---------6---------7---------8
La definizione della classe richiede anche due parametri aggiuntivi per indicare
come comportarsi se la dimensione dei dati forniti fosse superiore o inferiore a
quella attesa.

%--------1---------2---------3---------4---------5---------6---------7---------8
Prima di vedere il dettaglio della definizione dei vari campi, vediamo il 
contenuto degli altri due parametri richiesti per la generazione del codice
sorgente.

\subsection{Argomenti globali --- GenerateArgs}
%--------1---------2---------3---------4---------5---------6---------7---------8
La classe \textsl{GenerateArgs}, vedi cod.~\ref{lst:GenerateArgs}, permette di
definire alcuni parametri generali, comuni per tutte le classi generate.
%--------1---------2---------3---------4---------5---------6---------7---------8
Il parametro \texttt{sourceDirectory} indica la directory sorgente radice dove 
generare i sorgenti, i tre parametri successivi identificano il programma 
(o plugin) che ha fornito la definizione del tracciato, questi parametri sono 
mostrati come commento all'\,inizio dei file generati.

\ifesource
\begin{figure*}[!htb]
\begin{lstlisting}[language=java, caption=classe GenerateArgs, 
label=lst:GenerateArgs]
@Builder
public class GenerateArgs {
    @NonNull public final String sourceDirectory;
    public final String group;
    public final String artifact;
    public final String version;
}
\end{lstlisting}\index{GenerateArgs}
\end{figure*}
\else
\begin{elisting}
\begin{javacode}
@Builder
public class GenerateArgs {
    @NonNull public final String sourceDirectory;
    public final String group;
    public final String artifact;
    public final String version;
}
\end{javacode}
\caption{classe GenerateArgs}
\label{lst:GenerateArgs}\index{GenerateArgs}
\end{elisting}
\fi
%--------1---------2---------3---------4---------5---------6---------7---------8

\subsection{Default dei campi --- FieldDefault}
%--------1---------2---------3---------4---------5---------6---------7---------8
Nella definizione delle classi e dei campi, alcuni parametri disponibili nelle
relative definizione cambiano necessariamente (il nome del campo), altri sono
quasi sempre uguali per la stessa tipologia di campo (quali sono i caratteri
validi di un campo alfanumerico?).
Per semplificare la definizione delle classi, e relativi campi, è possibile
omettere nella definizione i parametri ``poco variabili'', è però necessario
indicare quale valore usare per questi parametri quando vengono omessi.
La classe \textsl{FieldDefault}, cod.~\ref{lst:FieldDefault}, mette a 
disposizione alcune classi dedicate per impostare il default dei parametri 
``poco variabili''.

\ifesource
\begin{figure*}[!htb]
\begin{lstlisting}[language=java, caption=classe FieldDefault, 
label=lst:FieldDefault]
@Data
public class FieldDefault {
    private (*\hyperref[lst:ClsDefault]{ClsDefault}*) cls = new ClsDefault();
    private (*\hyperref[lst:AbcDefault]{AbcDefault}*) abc = new AbcDefault();
    private (*\hyperref[lst:NumDefault]{NumDefault}*) num = new NumDefault();
    private (*\hyperref[lst:NuxDefault]{NuxDefault}*) nux = new NuxDefault();
    private (*\hyperref[lst:FilDefault]{FilDefault}*) fil = new FilDefault();
    private (*\hyperref[lst:CusDefault]{CusDefault}*) cus = new CusDefault();
}
\end{lstlisting}\index{FieldDefault}
\end{figure*}
\else
\begin{elisting}
\begin{javacode}
@Data
public class FieldDefault {
    private |\hyperref[lst:ClsDefault]{ClsDefault}| cls = new ClsDefault();
    private |\hyperref[lst:AbcDefault]{AbcDefault}| abc = new AbcDefault();
    private |\hyperref[lst:NumDefault]{NumDefault}| num = new NumDefault();
    private |\hyperref[lst:NuxDefault]{NuxDefault}| nux = new NuxDefault();
    private |\hyperref[lst:FilDefault]{FilDefault}| fil = new FilDefault();
    private |\hyperref[lst:CusDefault]{CusDefault}| cus = new CusDefault();
}
\end{javacode}
\caption{classe FieldDefault}
\label{lst:FieldDefault}\index{FieldDefault}
\end{elisting}
\fi
%--------1---------2---------3---------4---------5---------6---------7---------8
Il primo default riguarda il comportamento di default della classe quando viene 
creata partendo da una struttura (stringa), e questa ha una dimensione diversa 
da quella attesa

\ifesource
\begin{figure*}[!htb]
\begin{lstlisting}[language=java, caption=classe ClsDefault, 
label=lst:ClsDefault]
@Data
public static class ClsDefault {
    private (*\hyperref[lst:LoadOverflowAction]{LoadOverflowAction}*) onOverflow = LoadOverflowAction.Trunc;   // enum {Error, Trunc}
    private (*\hyperref[lst:LoadUnderflowAction]{LoadUnderflowAction}*) onUnderflow = LoadUnderflowAction.Pad;  // enum {Error, Pad}
    private boolean doc = true;
}
\end{lstlisting}
\end{figure*}
\else
\begin{elisting}
\begin{javacode}
@Data
public static class ClsDefault {
    private |\hyperref[lst:LoadOverflowAction]{LoadOverflowAction}| onOverflow = LoadOverflowAction.Trunc;   // enum {Error, Trunc}
    private |\hyperref[lst:LoadUnderflowAction]{LoadUnderflowAction}| onUnderflow = LoadUnderflowAction.Pad;  // enum {Error, Pad}
    private boolean doc = true;
}
\end{javacode}
\caption{classe ClsDefault}\index{ClsDefault}
\label{lst:ClsDefault}
\end{elisting}
\fi
%--------1---------2---------3---------4---------5---------6---------7---------8
nel caso che la lunghezza della struttura fornita sia maggiore di quella attesa
è possibile lanciare una eccezione e ignorare il contenuto in eccesso,
nel caso che la lunghezza della struttura fornita sia minore di quella atteso è
possibile lanciare una eccezione o completare la parte mancante con i valori di
default della parte mancante.

%--------1---------2---------3---------4---------5---------6---------7---------8
Gli altri default permettono di impostare i valori di default di alcuni 
parametri per cinque tipologie di campi. Non avendo mostrato il dettaglio delle 
definizione delle varie tipologie di campo, non è opportuno introdurre in 
questo punto il contenuto delle classi dei default, saranno mostrate insieme al 
campo corrispondente.

\section{Definizione di campi}
%--------1---------2---------3---------4---------5---------6---------7---------8
Nella definizione della classe, e dell'\,interfaccia l'\,elenco dei campi è
impostato come \verb!List<FieldModel>!, ma l'\,interfaccia \verb!FieldModel! è
una scatola vuota, serve solo per collegare a essa tutte le definizioni dei
campi. In generale tutti i campi hanno una posizione iniziale (offset) e una
dimensione (length); molti campi sono referenziabili tramite un nome, 
ma non tutti hanno necessariamente il nome; quando i campi hanno un nome
possono essere primari o sovra-definire (override) campi primari, in fase di 
inizializzazione dei campi di una classe vengono considerati solo le definizioni
dei campi primari.

\subsection{Campo Alfanumerico}
%--------1---------2---------3---------4---------5---------6---------7---------8
Un campo alfanumerico ha i 4 parametri base: \verb!offset!, \verb!length!,
\verb!name! e \verb!override!.

%--------1---------2---------3---------4---------5---------6---------7---------8
I parametri \verb!onOverflow! e \verb!onUnderflow! indica come deve comportarsi
il setter quando viene fornito un valore con dimensione maggiore o minore di
quella prevista per quel campo.
Il parametro \hypertarget{abc:ovf}{\texttt{onOverflow}} può assumere i valori 
\verb!Error! e \verb!Trunc!, nel primo caso è atteso che il codice generi una 
eccezione, nel secondo caso il valore viene trocato (a destra) ignorando i 
caratteri in eccesso rispetto alla dimensione attesa.
Il parametro \hypertarget{abc:unf}{\texttt{onUnderflow}} può assumere i valori 
\verb!Error! e \verb!Pad!, nel primo caso è atteso che il codice generi una 
eccezione, nel secondo caso vengono affiunti degli spazi (a destra) per 
raggiungere la dimensione attesa.

%--------1---------2---------3---------4---------5---------6---------7---------8
Il parametro \hypertarget{abc:chk}{\texttt{check}} permette di indicare dei 
controlli per restringere l'\,insieme di caratteri permessi per il valore. 
Questo controllo si attiva in fase di validazione della strina-dati, chiamando 
i setter e i getter.
I possibili valori sono \verb!None!, \verb!Ascii!, \verb!Latin1!, \verb!Valid!,
nel primo caso non viene fatto nessun controllo, nel secondo caso vengono 
accettati solo i caratteri ASCII, nel terzo caso vengono accettati solo i 
caratteri ISO-8859-1\footnote{più precisamente i caratteri unicode da 
\texttt{\textbackslash u0020} a \texttt{\textbackslash u007e} e 
da \texttt{\textbackslash u00a0} a \texttt{\textbackslash u00ff}}, e 
nell'\,ultimo vengono accettati i caratteri UTF-8 validi.

\ifesource
\begin{figure*}[!htb]
\begin{lstlisting}[language=java, 
caption=interfaccia AbcModel (campo alfanumerico), 
label=lst:AbcModel]
public interface AbcModel extends FieldModel {
    void setOffset(int offset);
    void setLength(int length);
    void setName(String name);

    void setOverride(boolean override);

    void setOnOverflow((*\hyperref[lst:OverflowAction]{OverflowAction}*) onOverflow);
    void setOnUnderflow((*\hyperref[lst:UnderflowAction]{UnderflowAction}*) onUnderflow);
    void setCheck((*\hyperref[lst:CheckAbc]{CheckAbc}*) check);
    void setNormalize((*\hyperref[lst:NormalizeAbcMode]{NormalizeAbcMode}*) normalize);
    void setCheckGetter(Boolean checkGetter);
    void setCheckSetter(Boolean checkSetter);
}
\end{lstlisting}\index{AbcModel}
\end{figure*}
\else
\begin{elisting}
\begin{javacode}
public interface AbcModel extends FieldModel {
    void setOffset(int offset);
    void setLength(int length);
    void setName(String name);

    void setOverride(boolean override);

    void setOnOverflow(|\hyperref[lst:OverflowAction]{OverflowAction}| onOverflow);
    void setOnUnderflow(|\hyperref[lst:UnderflowAction]{UnderflowAction}| onUnderflow);
    void setCheck(|\hyperref[lst:CheckAbc]{CheckAbc}| check);
    void setNormalize(|\hyperref[lst:NormalizeAbcMode]{NormalizeAbcMode}| normalize);
    void setCheckGetter(Boolean checkGetter);
    void setCheckSetter(Boolean checkSetter);
}
\end{javacode}
\caption{interfaccia AbcModel (campo alfanumerico)}
\label{lst:AbcModel}\index{AbcModel}
\end{elisting}
\fi

%--------1---------2---------3---------4---------5---------6---------7---------8
Il campo \hypertarget{abc:nrm}{\texttt{normalize}} permette di indicare come 
normalizzare il valore del campo in fase di getter.
Il parametro \verb!normalize! può assumere 3 valori \verb!None!, \verb!Trim! e
\verb!Trim1!. Il primo valore indica di non eseguire nessuna modifica del dato,
il secondo valore indica di rimuovere tutti gli spazi a destra fino a trovare
un carattere diverso da spazio, nel caso che il valore sia composto solo da
spazi viene prodotta una stringa vuota, l'\,ultimo valore, analogamente al 
valore precedente fa rimuovere gli spazi a destra fino a trovare un carattere
diverso da spazio, ma nel caso che il valore sia composto solo da spazi 
restituisce una stringa composta da uno spazio.

%--------1---------2---------3---------4---------5---------6---------7---------8
Il campo \hypertarget{abc:get}{\texttt{checkGetter}} indica se attivare o meno 
il controllo indicato col parametro \verb!check! quando viene chiamato il 
getter; se la stringa-dati viene validata preventivamente, questo controllo può 
essere disattivato.
Il campo \hypertarget{abc:set}{\texttt{checkSetter}} indica se attivare o meno 
il controllo indicato col parametro \verb!check! quando viene chiamato il 
setter.

\ifesource
\begin{figure*}[!htb]
\begin{lstlisting}[language=java, 
caption=class AbcDefault (default campo alfanumerico), 
label=lst:AbcDefault]
@Data
public class AbcDefault {
    private (*\hyperref[lst:OverflowAction]{OverflowAction}*) onOverflow = OverflowAction.Trunc;
    private (*\hyperref[lst:UnderflowAction]{UnderflowAction}*) onUnderflow = UnderflowAction.Pad;
    private (*\hyperref[lst:CheckAbc]{CheckAbc}*) check = CheckAbc.Ascii;
    private (*\hyperref[lst:NormalizeAbcMode]{NormalizeAbcMode}*) normalize = NormalizeAbcMode.None;
    private boolean checkGetter = true;
    private boolean checkSetter = true;
}
\end{lstlisting}\index{AbcDefault}
\end{figure*}
\else
\begin{elisting}
\begin{javacode}
@Data
public class AbcDefault {
    private |\hyperref[lst:OverflowAction]{OverflowAction}| onOverflow = OverflowAction.Trunc;
    private |\hyperref[lst:UnderflowAction]{UnderflowAction}| onUnderflow = UnderflowAction.Pad;
    private |\hyperref[lst:CheckAbc]{CheckAbc}| check = CheckAbc.Ascii;
    private |\hyperref[lst:NormalizeAbcMode]{NormalizeAbcMode}| normalize = NormalizeAbcMode.None;
    private boolean checkGetter = true;
    private boolean checkSetter = true;
}
\end{javacode}
\caption{class AbcDefault (default campo alfanumerico)}
\label{lst:AbcDefault}\index{AbcDefault}
\end{elisting}
\fi

%--------1---------2---------3---------4---------5---------6---------7---------8
La classe \textsl{AbcDefault}, vedi cod.~\ref{lst:AbcDefault}, imposta i valori
di default per i parametri \verb!onOverflow!, \verb!onUnderflow!, \verb!check!, 
\verb!normalize!, \verb!checkGetter!, \verb!checkSetter!, nel caso non siano 
impostati dal client.

\subsection{Campo Numerico}
%--------1---------2---------3---------4---------5---------6---------7---------8
Anche un campo numerico ha i 4 parametri base: \verb!offset!, \verb!length!,
\verb!name! e \verb!override!.

%--------1---------2---------3---------4---------5---------6---------7---------8
Il parametro \verb!access! indica come generare i setter e getter.
Nella stringa-dati il campo numerico ha una rappresentazione in formato stringa,
nel codice generato è possibile scegliere se i setter e getter gestiscano un
formato stringa (con caratteri numerici) o convertire il frammento di 
stringa-dati corrispondente al campo in una rappresentazione numerica nativa 
(\verb!byte!, \verb!short!, \verb!int!, \verb!!\verb!long!) o generare entrambi 
i formati.
Il parametro \verb!access! può assumere i valori \verb!String!, \verb!Number! e 
\verb!Both!. Nel primo caso vengono generati setter e getter che gestiscono
stringhe (numeriche), nel secondo caso formati numerici nativi, e nell'\,ultimo
caso vengono generati entrambi i formati (andrà indicato dal provider come
distinguere il getter stringa da quello numerico).
Nel caso venga usato un accesso che prevede setter/getter di tipo stringa,
viene controllato in fase di setter che la stringa fornita sia numerica,
e in fase di getter che la stringa restituita sia numerica.

\ifesource
\begin{figure*}[!htb]
\begin{lstlisting}[language=java, 
caption=interfaccia NumModel (campo numerico), 
label=lst:NumModel]
public interface NumModel extends FieldModel {
    void setOffset(int offset);
    void setLength(int length);
    void setName(String name);

    void setOverride(boolean override);

    void setAccess((*\hyperref[lst:AccesMode]{AccesMode}*) access);
    void setWordWidth((*\hyperref[lst:WordWidth]{WordWidth}*) width);
    void setOnOverflow((*\hyperref[lst:OverflowAction]{OverflowAction}*) onOverflow);
    void setOnUnderflow((*\hyperref[lst:UnderflowAction]{UnderflowAction}*) onUnderflow);
    void setNormalize((*\hyperref[lst:NormalizeNumMode]{NormalizeNumMode}*) normalize);
}
\end{lstlisting}\index{NumModel}
\end{figure*}
\else
\begin{elisting}
\begin{javacode}
public interface NumModel extends FieldModel {
    void setOffset(int offset);
    void setLength(int length);
    void setName(String name);

    void setOverride(boolean override);

    void setAccess(|\hyperref[lst:AccesMode]{AccesMode}| access);
    void setWordWidth(|\hyperref[lst:WordWidth]{WordWidth}| width);
    void setOnOverflow(|\hyperref[lst:OverflowAction]{OverflowAction}| onOverflow);
    void setOnUnderflow(|\hyperref[lst:UnderflowAction]{UnderflowAction}| onUnderflow);
    void setNormalize(|\hyperref[lst:NormalizeNumMode]{NormalizeNumMode}| normalize);
}
\end{javacode}
\caption{interfaccia NumModel (campo numerico)}
\label{lst:NumModel}\index{NumModel}
\end{elisting}
\fi

%--------1---------2---------3---------4---------5---------6---------7---------8
Il parametro \verb!wordWidth! ha senso solo se si è scelta una modalità di
accesso che genera i setter/getter numerici, sostanzialmente indica la 
dimensione minima da usare nelle rappresentazioni numeriche.
Il parametro \verb!wordWidth! può assumere i valori \verb!Byte!, \verb!Short!, 
\verb!Int! e \verb!Long!, i valori corrispondono all'\,utilizzo dei tipi nativi
corrispondenti.
Per fare un esempio, se un campo numerico è rappresentato da una stringa di 4 
caratteri, può essere convertito in formato numerico in formato \verb!short!,
se il parametro \verb!wordWidth! è impostato a \verb!Int! vengono generati
setter/getter di tipo \verb!int!; se il valore del parametro fosse stato 
\verb!Byte! o  \verb!Short! sarebbero stati generati setter/getter di tipo 
\verb!short!.

%--------1---------2---------3---------4---------5---------6---------7---------8
I parametri \verb!onOverflow! e \verb!onUnderflow! indica come deve comportarsi
il setter quando viene fornito un valore con dimensione maggiore o minore di
quella prevista per quel campo.
Il parametro \verb!onOverflow! può assumere i valori \verb!Error! e 
\verb!Trunc!, nel primo caso è atteso che il codice generi una eccezione,
nel secondo caso il valore viene troncato (a sinistra) ignorando le cifre in
eccesso rispetto alla dimensione attesa.
Il parametro \verb!onUnderflow! può assumere i valori \verb!Error! e \verb!Pad!,
nel primo caso è atteso che il codice generi una eccezione, nel secondo caso
vengono aggiunti degli zero (a sinistra) per raggiungere la dimensione attesa.

%--------1---------2---------3---------4---------5---------6---------7---------8
Il campo \verb!normalize! ha senso solo se si è scelta una modalità di
accesso che genera i setter/getter stringa, permette di indicare come 
normalizzare il valore del campo in fase di getter.
Il parametro \verb!normalize! può assumere 2 valori \verb!None! e \verb!Trim!. 
Il primo valore indica di non eseguire nessuna modifica del dato,
l'\,altro indica di rimuovere tutti gli zero a sinistra fino a trovare
una cifra diversa da zero, nel caso che il valore sia composto solo da
zeri viene prodotta una stringa composta da uno zero.

\ifesource
\begin{figure*}[!htb]
\begin{lstlisting}[language=java, 
caption=class NumDefault (default campo numerico), 
label=lst:NumDefault]
@Data
public class NumDefault {
    private (*\hyperref[lst:AccesMode]{AccesMode}*) access = AccesMode.String;
    private (*\hyperref[lst:WordWidth]{WordWidth}*) wordWidth = WordWidth.Int;
    private (*\hyperref[lst:OverflowAction]{OverflowAction}*) onOverflow = OverflowAction.Trunc;
    private (*\hyperref[lst:UnderflowAction]{UnderflowAction}*) onUnderflow = UnderflowAction.Pad;
    private (*\hyperref[lst:NormalizeNumMode]{NormalizeNumMode}*) normalize = NormalizeNumMode.None;
}
\end{lstlisting}\index{NumDefault}
\end{figure*}
\else
\begin{elisting}
\begin{javacode}
@Data
public class NumDefault {
    private |\hyperref[lst:AccesMode]{AccesMode}| access = AccesMode.String;
    private |\hyperref[lst:WordWidth]{WordWidth}| wordWidth = WordWidth.Int;
    private |\hyperref[lst:OverflowAction]{OverflowAction}| onOverflow = OverflowAction.Trunc;
    private |\hyperref[lst:UnderflowAction]{UnderflowAction}| onUnderflow = UnderflowAction.Pad;
    private |\hyperref[lst:NormalizeNumMode]{NormalizeNumMode}| normalize = NormalizeNumMode.None;
}
\end{javacode}
\caption{class NumDefault (default campo numerico)}
\label{lst:NumDefault}\index{NumDefault}
\end{elisting}
\fi

%--------1---------2---------3---------4---------5---------6---------7---------8
La classe \textsl{NumDefault}, vedi cod.~\ref{lst:NumDefault}, imposta i valori
di default per i parametri \verb!access!, \verb!wordWidth!, \texttt{onOver\-flow}, 
\verb!onUnderflow! e \verb!normalize! nel caso non siano  impostati dal client.


\subsection{Campo Custom (alfanumerico)}
%--------1---------2---------3---------4---------5---------6---------7---------8
Anche un campo custom ha i 4 parametri base: \verb!offset!, \verb!length!,
\verb!name! e \verb!override!.

%--------1---------2---------3---------4---------5---------6---------7---------8
Un campo custom è una generalizzazione di un campo alfanumerico, e può essere 
configurato per emulare un campo numerico o numerico nullabile.
Il primo parametro sensibile da considerare è \verb!align!, il parametro indica
come deve essere allineato il campo \texttt{onUnderflow = Pad}.
Il parametro può assumere 2 valori \verb!LFT! e \verb!RGT!, il primo valore 
indica che il campo deve essere allineato a sinistra, il secondo valore che 
deve essere allineato a destra.
Il valore di questo parametro non impatta solo sul parametro \verb!onUnderflow!
(indicando da quale direzione devono essere aggiunti i caratteri di padding),
ma anche su \verb!onOverflow! (indicando da quale direzione devono essere tolti
i caratteri in eccesso) e \verb!normalize! (indicando da quale direzione devono
essere rimossi i caratteri di padding).

%--------1---------2---------3---------4---------5---------6---------7---------8
Il parametro \verb!padChar! indica il carattere di riempimento da aggiungere
(in caso di \texttt{onUnderflow = Pad}) o togliere (in caso di 
\texttt{normalize = Trim}).
%--------1---------2---------3---------4---------5---------6---------7---------8
Il parametro \verb!initChar! indica il carattere da usare per inizializzare il
campo.

\ifesource
\begin{figure*}[!htb]
\begin{lstlisting}[language=java, 
caption=interfaccia CusModel (campo custom), 
label=lst:CusModel]
public interface CusModel extends FieldModel {
    void setOffset(int offset);
    void setLength(int length);
    void setName(String name);
    
    void setOverride(boolean override);
    
    void setAlign((*\hyperref[lst:AlignMode]{AlignMode}*) align);
    void setPadChar(Character padChar);
    void setInitChar(Character initChar);
    void setCheck((*\hyperref[lst:CheckCus]{CheckCus}*) check);
    void setRegex(String regex);
    void setOnOverflow((*\hyperref[lst:OverflowAction]{OverflowAction}*) onOverflow);
    void setOnUnderflow((*\hyperref[lst:UnderflowAction]{UnderflowAction}*) onUnderflow);
    void setNormalize((*\hyperref[lst:NormalizeAbcMode]{NormalizeAbcMode}*) normalize);
    void setCheckGetter(Boolean checkGetter);
    void setCheckSetter(Boolean checkSetter);
}
\end{lstlisting}\index{CusModel}
\end{figure*}
\else
\begin{elisting}
\begin{javacode}
public interface CusModel extends FieldModel {
    void setOffset(int offset);
    void setLength(int length);
    void setName(String name);
    
    void setOverride(boolean override);
    
    void setAlign(|\hyperref[lst:AlignMode]{AlignMode}| align);
    void setPadChar(Character padChar);
    void setInitChar(Character initChar);
    void setCheck(|\hyperref[lst:CheckCus]{CheckCus}| check);
    void setRegex(String regex);
    void setOnOverflow(|\hyperref[lst:OverflowAction]{OverflowAction}| onOverflow);
    void setOnUnderflow(|\hyperref[lst:UnderflowAction]{UnderflowAction}| onUnderflow);
    void setNormalize(|\hyperref[lst:NormalizeAbcMode]{NormalizeAbcMode}| normalize);
    void setCheckGetter(Boolean checkGetter);
    void setCheckSetter(Boolean checkSetter);
}
\end{javacode}
\caption{interfaccia CusModel (campo custom)}
\label{lst:CusModel}\index{CusModel}
\end{elisting}
\fi

%--------1---------2---------3---------4---------5---------6---------7---------8
Per il parametro \verb!check! valgono le stesse considerazioni del 
corrispondente parametro nel \hyperlink{abc:chk}{caso alfanumerico},
In questo caso i possibili valori sono \verb!None!, \verb!Ascii!, \verb!Latin1!, 
\verb!Valid!, \verb!Digit! e \verb!DigitOrBlank!.
I primi quattro valori sono identici al caso alfanumerico, il valore 
\verb!Digit! limita i caratteri accettati a quelli numerici (da 0 a 9), come
per un campo numerico; il valore \verb!DigitOrBlank! richiede che i caratteri
siano numerici o tutti spazi, come per un campo numerico nullabile.

%--------1---------2---------3---------4---------5---------6---------7---------8
Il parametro \verb!regex! può essere valorizzato con una espressione regolare
che deve essere soddisfatta da valore del campo. Se questo parametro è presente,
viene ignorato il parametro \verb!check!.

\ifesource
\begin{figure*}[!htb]
\begin{lstlisting}[language=java, 
caption=class CusDefault (default campo custom), 
label=lst:CusDefault]
@Data
public class CusDefault {
    private (*\hyperref[lst:AlignMode]{AlignMode}*) align = AlignMode.LFT;
    private char padChar = ' ';
    private char initChar = ' ';
    private (*\hyperref[lst:CheckCus]{CheckCus}*) check = CheckCus.Ascii;
    private (*\hyperref[lst:OverflowAction]{OverflowAction}*) onOverflow = OverflowAction.Trunc;
    private (*\hyperref[lst:UnderflowAction]{UnderflowAction}*) onUnderflow = UnderflowAction.Pad;
    private (*\hyperref[lst:NormalizeAbcMode]{NormalizeAbcMode}*) normalize = NormalizeAbcMode.None;
    private boolean checkGetter = true;
    private boolean checkSetter = true;
}
\end{lstlisting}\index{CusDefault}
\end{figure*}
\else
\begin{elisting}
\begin{javacode}
@Data
public class CusDefault {
    private |\hyperref[lst:AlignMode]{AlignMode}| align = AlignMode.LFT;
    private char padChar = ' ';
    private char initChar = ' ';
    private |\hyperref[lst:CheckCus]{CheckCus}| check = CheckCus.Ascii;
    private |\hyperref[lst:OverflowAction]{OverflowAction}| onOverflow = OverflowAction.Trunc;
    private |\hyperref[lst:UnderflowAction]{UnderflowAction}| onUnderflow = UnderflowAction.Pad;
    private |\hyperref[lst:NormalizeAbcMode]{NormalizeAbcMode}| normalize = NormalizeAbcMode.None;
    private boolean checkGetter = true;
    private boolean checkSetter = true;
}
\end{javacode}
\caption{class CusDefault (default campo custom)}
\label{lst:CusDefault}\index{CusDefault}
\end{elisting}
\fi

%--------1---------2---------3---------4---------5---------6---------7---------8
Per i parametri \hyperlink{abc:ovf}{\texttt{onOverflow}}, 
\hyperlink{abc:unf}{\texttt{onUnderflow}}, 
\hyperlink{abc:nrm}{\texttt{normalize}},
\hyperlink{abc:get}{\texttt{checkGetter}},
\hyperlink{abc:get}{\texttt{checkSetter}} valgono le stessa considerazioni dei 
corrispondenti campi nel caso alfanumerico. Attenzione perché l'\,azione 
dei parametri \texttt{onOver\-flow}, \verb!onUnderflow! e \verb!normalize! 
dipende anche dal valore dei parametri \verb!align! e \verb!initChar!.

%--------1---------2---------3---------4---------5---------6---------7---------8
La classe \textsl{CusDefault}, vedi cod.~\ref{lst:CusDefault}, imposta i valori
di default per i parametri \verb!align!, \verb!padChar!, \verb!initChar!, 
\verb!check!, \verb!onOverflow!, \verb!onUnderflow!, \verb!normalize!, 
\verb!checkGetter! e \verb!checkSetter! nel caso non siano  impostati dal 
client.


\subsection{Campo Numerico nullabile}
%--------1---------2---------3---------4---------5---------6---------7---------8
Un campo numerico nullabile è una estensione di un campo numerico ordinario.
La differenza è che nella stringa-dati è permesso il valore spazio (tutti i
caratteri a spazio), a questo valore corrisponde il valore \verb!null! nel 
campo della classe-dati.

\ifesource
\begin{figure*}[!htb]
\begin{lstlisting}[language=java, 
caption=interfaccia NuxModel (campo numerico nullabile), 
label=lst:NuxModel]
public interface NuxModel extends FieldModel {
    void setOffset(int offset);
    void setLength(int length);
    void setName(String name);

    void setOverride(boolean override);

    void setAccess((*\hyperref[lst:AccesMode]{AccesMode}*) access);
    void setWordWidth((*\hyperref[lst:WordWidth]{WordWidth}*) width);
    void setOnOverflow((*\hyperref[lst:OverflowAction]{OverflowAction}*) onOverflow);
    void setOnUnderflow((*\hyperref[lst:UnderflowAction]{UnderflowAction}*) onUnderflow);
    void setNormalize((*\hyperref[lst:NormalizeNumMode]{NormalizeNumMode}*) normalize);
    void setInitialize((*\hyperref[lst:InitializeNuxMode]{InitializeNuxMode}*) initialize);
}
\end{lstlisting}\index{NuxModel}
\end{figure*}
\else
\begin{elisting}
\begin{javacode}
public interface NuxModel extends FieldModel {
    void setOffset(int offset);
    void setLength(int length);
    void setName(String name);

    void setOverride(boolean override);

    void setAccess(|\hyperref[lst:AccesMode]{AccesMode}| access);
    void setWordWidth(|\hyperref[lst:WordWidth]{WordWidth}| width);
    void setOnOverflow(|\hyperref[lst:OverflowAction]{OverflowAction}| onOverflow);
    void setOnUnderflow(|\hyperref[lst:UnderflowAction]{UnderflowAction}| onUnderflow);
    void setNormalize(|\hyperref[lst:NormalizeNumMode]{NormalizeNumMode}| normalize);
    void setInitialize(|\hyperref[lst:InitializeNuxMode]{InitializeNuxMode}| initialize);
}
\end{javacode}
\caption{interfaccia NuxModel (campo numerico nullabile)}
\label{lst:NuxModel}\index{NuxModel}
\end{elisting}
\fi

%--------1---------2---------3---------4---------5---------6---------7---------8
Come si vede dalla definizione, cod.~\ref{lst:NuxModel} ci sono gli stessi
parametri di un campo numerico (cod.~\ref{lst:NumModel}), con gli stessi 
significati, più uno: il parametro \verb!initialize!.
Questo parametro indica come inizializzare il campo quando viene creata la
classe con il costruttore vuoto, a spazio (cioè a \verb!null!), o a zero.

\ifesource
\begin{figure*}[!htb]
\begin{lstlisting}[language=java, 
caption=class NuxDefault (default campo numerico nullabile), 
label=lst:NuxDefault]
@Data
public class NuxDefault {
    private (*\hyperref[lst:AccesMode]{AccesMode}*) access = AccesMode.String;
    private (*\hyperref[lst:WordWidth]{WordWidth}*) wordWidth = WordWidth.Int;
    private (*\hyperref[lst:OverflowAction]{OverflowAction}*) onOverflow = OverflowAction.Trunc;
    private (*\hyperref[lst:UnderflowAction]{UnderflowAction}*) onUnderflow = UnderflowAction.Pad;
    private (*\hyperref[lst:NormalizeNumMode]{NormalizeNumMode}*) normalize = NormalizeNumMode.None;
    private (*\hyperref[lst:InitializeNuxMode]{InitializeNuxMode}*) initialize = InitializeNuxMode.Spaces;
}
\end{lstlisting}\index{NuxDefault}
\end{figure*}
\else
\begin{elisting}
\begin{javacode}
@Data
public class NuxDefault {
    private |\hyperref[lst:AccesMode]{AccesMode}| access = AccesMode.String;
    private |\hyperref[lst:WordWidth]{WordWidth}| wordWidth = WordWidth.Int;
    private |\hyperref[lst:OverflowAction]{OverflowAction}| onOverflow = OverflowAction.Trunc;
    private |\hyperref[lst:UnderflowAction]{UnderflowAction}| onUnderflow = UnderflowAction.Pad;
    private |\hyperref[lst:NormalizeNumMode]{NormalizeNumMode}| normalize = NormalizeNumMode.None;
    private |\hyperref[lst:InitializeNuxMode]{InitializeNuxMode}| initialize = InitializeNuxMode.Spaces;
}
\end{javacode}
\caption{class NuxDefault (default campo numerico nullabile)}
\label{lst:NuxDefault}
\end{elisting}
\fi

%--------1---------2---------3---------4---------5---------6---------7---------8
Per gestire il default a livello generale per questo tipo di campi viene usata
una classe default dedicata, che è la copia della corrispondente per il caso
numerico ordinario, con in più il default di inizializzazione.

\subsection{Campo Dominio}
%--------1---------2---------3---------4---------5---------6---------7---------8
Un campo di tipo dominio è un campo alfanumerico che può assumere solo dei 
valori costanti predefiniti.
Il campo di tipo dominio, cod.~\ref{lst:DomModel}, ha i 4 parametri base: 
\verb!offset!, \verb!length!, \verb!name! e \verb!override!, e in più il 
parametro \verb!items! che dovrà fornire l'\,elenco dei valori costanti ammessi.

\ifesource
\begin{figure*}[!htb]
\begin{lstlisting}[language=java, 
caption=interfaccia DomModel (campo dominio), 
label=lst:DomModel]
public interface DomModel extends FieldModel {
    void setOffset(int offset);
    void setLength(int length);
    void setName(String name);
    
    void setOverride(boolean override);
    
    void setItems(String[] items);
}
\end{lstlisting}\index{DomModel}
\end{figure*}
\else
\begin{elisting}
\begin{javacode}
public interface DomModel extends FieldModel {
    void setOffset(int offset);
    void setLength(int length);
    void setName(String name);
    
    void setOverride(boolean override);
    
    void setItems(String[] items);
}
\end{javacode}
\caption{interfaccia DomModel (campo dominio)}
\label{lst:DomModel}
\end{elisting}
\fi

%--------1---------2---------3---------4---------5---------6---------7---------8
Per questo tipo di campo non ha senso nessun default globale come visto per i
campi alfanumerici e numerici. Il valori permessi sono quelli forniti nel 
parametro \verb!items!, qualunque altro valore causerà una eccezione.

\subsection{Campo Filler}
%--------1---------2---------3---------4---------5---------6---------7---------8
Un campo di tipo \textsl{Filler}, non è un campo vero e proprio, non ha un nome 
associato, non genera nessun setter o getter nella classe-dati, e nessun metodo
di controllo. È un modo per indicare che nella stringa-dati è presente un'\,area
a cui non è associato nessun valore, o non siamo interessati a quella area della
stringa-dati.

\ifesource
\begin{figure*}[!htb]
\begin{lstlisting}[language=java, 
caption=interfaccia FilModel (campo filler), 
label=lst:FilModel]
public interface FilModel extends FieldModel {
    void setOffset(int offset);
    void setLength(int length);
    
    void setFill(Character fill);
}
\end{lstlisting}\index{FilModel}
\end{figure*}
\else
\begin{elisting}
\begin{javacode}
public interface FilModel extends FieldModel {
    void setOffset(int offset);
    void setLength(int length);
    
    void setFill(Character fill);
}
\end{javacode}
\caption{interfaccia FilModel (campo filler)}
\label{lst:FilModel}
\end{elisting}
\fi

%--------1---------2---------3---------4---------5---------6---------7---------8
Il campo filler ha soltanto i 2 parametri base: \verb!offset! e \verb!length!.
Ha poi il parametro \verb!fill! per indicare con quale carattere valorizzare il
campo quando la classe-dati viene creata con il costruttore vuoto.

\ifesource
\begin{figure*}[!htb]
\begin{lstlisting}[language=java, 
caption=class FilDefault (default campo filler), 
label=lst:FilDefault]
@Data
public class FilDefault {
    private char fill = 0;
}
\end{lstlisting}\index{FilDefault}
\end{figure*}
\else
\begin{elisting}
\begin{javacode}
@Data
public class FilDefault {
    private char fill = 0;
}
\end{javacode}
\caption{class FilDefault (default campo filler)}
\label{lst:FilDefault}
\end{elisting}
\fi

%--------1---------2---------3---------4---------5---------6---------7---------8
Il campo filler prevede una classe default dedicata per indicare il carattere di
riempimento a livello generale.

\subsection{Campo Valore costante}
%--------1---------2---------3---------4---------5---------6---------7---------8
Un campo di tipo \textsl{Costante} è simile al campo filler. Come per il campo
filler non vengono generati setter e getter nella classe-dati, ma nella 
validazione della stringa-dati viene verificato che l'\,area corrispondente al 
campo abbia il valore costante indicato.

\ifesource
\begin{figure*}[!htb]
\begin{lstlisting}[language=java, 
caption=interfaccia ValModel (campo costante), 
label=lst:ValModel]
public interface ValModel extends FieldModel {
    void setOffset(int offset);
    void setLength(int length);
    
    void setValue(String value);
}
\end{lstlisting}\index{ValModel}
\end{figure*}
\else
\begin{elisting}
\begin{javacode}
public interface ValModel extends FieldModel {
    void setOffset(int offset);
    void setLength(int length);
    
    void setValue(String value);
}
\end{javacode}
\caption{interfaccia ValModel (campo costante)}
\label{lst:ValModel}
\end{elisting}
\fi

%--------1---------2---------3---------4---------5---------6---------7---------8
Anche il campo costante ha soltanto i 2 parametri base: \verb!offset! e 
\verb!length!.
Ha poi il parametro \verb!value! per indicare il valore costante atteso.
Il valore fornito viene usato per \textit{valorizzare} il campo quando la 
classe-dati viene istanziata con il costruttore vuoto.

\subsection{Campo Gruppo di campi}
%--------1---------2---------3---------4---------5---------6---------7---------8
Un campo di tipo \textsl{Gruppo} non è un campo vero e proprio, è in realtà un
contenitore di altri campi.

\ifesource
\begin{figure*}[!htb]
\begin{lstlisting}[language=java, 
caption=interfaccia GrpModel (campo gruppo), 
label=lst:GrpModel]
public interface GrpModel extends FieldModel {
    void setOffset(int offset);
    void setLength(int length);
    void setName(String name);
    
    void setOverride(boolean override);
    
    void setFields(List<FieldModel> fields);
}
\end{lstlisting}\index{GrpModel}
\end{figure*}
\else
\begin{elisting}
\begin{javacode}
public interface GrpModel extends FieldModel {
    void setOffset(int offset);
    void setLength(int length);
    void setName(String name);
    
    void setOverride(boolean override);
    
    void setFields(List<FieldModel> fields);
}
\end{javacode}
\caption{interfaccia GrpModel (campo gruppo)}
\label{lst:GrpModel}
\end{elisting}
\fi

%--------1---------2---------3---------4---------5---------6---------7---------8
Un campo gruppo ha i 4 parametri base: \verb!offset!, \verb!length!, 
\verb!name! e \verb!override!.
Oltre a questo prevede il parametro \verb!fields! per fornire l'\,elenco dei 
campi figli.
L'\,area di stringa-data selezionata da \verb!offset! e \verb!length! dovrà
essere completamente definita dai campi figli.
È possibile per un campo figlio, usare lo stesso nome di un campo definito
allo stesso livello del campo padre. 

\subsection{Campo Gruppo di campi ripetuto}
%--------1---------2---------3---------4---------5---------6---------7---------8
Un campo di tipo \textsl{Gruppo Ripetuto} è simile al campo gruppo, con la 
differenza che sono presenti $n$ occorrenze del gruppo.
Oltre ai parametri usati dal campo gruppo è presente il parametro \verb!times!,
che indica il numero di volte che il gruppo è ripetuto.

\ifesource
\begin{figure*}[!htb]
\begin{lstlisting}[language=java, 
caption=interfaccia OccModel (campo gruppo ripetuto), 
label=lst:OccModel]
public interface OccModel extends  FieldModel {
    void setOffset(int offset);
    void setLength(int length);
    void setName(String name);
    
    void setOverride(boolean override);
    
    void setFields(List<FieldModel> fields);
    void setTimes(int times);
}
\end{lstlisting}\index{OccModel}
\end{figure*}
\else
\begin{elisting}
\begin{javacode}
public interface OccModel extends  FieldModel {
    void setOffset(int offset);
    void setLength(int length);
    void setName(String name);
    
    void setOverride(boolean override);
    
    void setFields(List<FieldModel> fields);
    void setTimes(int times);
}
\end{javacode}
\caption{interfaccia OccModel (campo gruppo ripetuto)}
\label{lst:OccModel}
\end{elisting}
\fi

%--------1---------2---------3---------4---------5---------6---------7---------8
In questo caso l'\,area della stringa-dati definita da \verb!offset! e 
\verb!length! è quella della prima occorrenza del gruppo. La dimensione realmente
usata è \verb!length! $\times$ \verb!times!.

\subsection{Campi incorporati mediante interfaccia}
%--------1---------2---------3---------4---------5---------6---------7---------8
Una interfaccia (cod.~\ref{lst:TraitModel}) può essere definita allo stesso modo
di una classe (cod.~\ref{lst:ClassModel}).
Il campo \textsl{Incorporato} non è un campo vero e proprio, non ha un nome.
Permette di indicare che una certa area della stringa-dati deve essere 
interpretata con i campi definiti dalla interfaccia indicata.

\ifesource
\begin{figure*}[!htb]
\begin{lstlisting}[language=java, 
caption=interfaccia EmbModel (campo incorporato), 
label=lst:EmbModel]
public interface EmbModel extends FieldModel {
    void setOffset(int offset);
    void setLength(int length);
    
    void setSource((*\hyperref[lst:TraitModel]{TraitModel}*) source);
}
\end{lstlisting}\index{EmbModel}
\end{figure*}
\else
\begin{elisting}
\begin{javacode}
public interface EmbModel extends FieldModel {
    void setOffset(int offset);
    void setLength(int length);
    
    void setSource(|\hyperref[lst:TraitModel]{TraitModel}| source);
}
\end{javacode}
\caption{interfaccia EmbModel (campo incorporato)}
\label{lst:EmbModel}
\end{elisting}
\fi
%--------1---------2---------3---------4---------5---------6---------7---------8
Il campo incorporato ha soltanto i 2 parametri base: \verb!offset! e 
\verb!length!.
Ha poi il parametro \verb!source! per indicare quale interfaccia utilizzare per
definire i campi. L'\,interfaccia dovrà avere la stessa dimensione 
(\verb!length!) del campo incorporato, la posizione iniziale della interfaccia
verrà traslata per adattarla al valore indicato dal campo incorporato.
%--------1---------2---------3---------4---------5---------6---------7---------8
La classe generata, ovviamente, implementerà l'\,interfaccia indicata.

\subsection{Campo Gruppo di campi definito mediante interfaccia} 
%--------1---------2---------3---------4---------5---------6---------7---------8
Il campo \textsl{Gruppo/Interfaccia} è simile al campo incorporato.
In questo caso i campi definiti dall'\,interfaccia non vengono incorporati al
livello corrente, ma viene definito un gruppo che li contiene.

\ifesource
\begin{figure*}[!htb]
\begin{lstlisting}[language=java, 
caption=interfaccia GrpTraitModel (campo gruppo/interfaccia), 
label=lst:GrpTraitModel]
public interface GrpTraitModel extends FieldModel {
    void setOffset(int offset);
    void setLength(int length);
    void setName(String name);
    
    void setOverride(boolean override);
    
    void setTypedef((*\hyperref[lst:TraitModel]{TraitModel}*) typedef);
}
\end{lstlisting}\index{GrpTraitModel}
\end{figure*}
\else
\begin{elisting}
\begin{javacode}
public interface GrpTraitModel extends FieldModel {
    void setOffset(int offset);
    void setLength(int length);
    void setName(String name);
    
    void setOverride(boolean override);
    
    void setTypedef(|\hyperref[lst:TraitModel]{TraitModel}| typedef);
}
\end{javacode}
\caption{interfaccia GrpTraitModel (campo gruppo/interfaccia)}
\label{lst:GrpTraitModel}
\end{elisting}
\fi

%--------1---------2---------3---------4---------5---------6---------7---------8
Un campo gruppo/interfaccia ha i 4 parametri base: \verb!offset!, \verb!length!, 
\verb!name! e \verb!override!.
Ha poi il parametro \verb!typedef! per indicare quale interfaccia utilizzare per
definire i campi del gruppo. L'\,interfaccia dovrà avere la stessa dimensione 
(\verb!length!) del campo gruppo/interfaccia, la posizione iniziale della 
interfaccia verrà traslata per adattarla al valore indicato dal campo 
gruppo/interfaccia.
%--------1---------2---------3---------4---------5---------6---------7---------8
In questo caso è il gruppo che implementa l'\,interfaccia indicata.


\subsection{Campo Gruppo di campi ripetuto definito mediante interfaccia}
%--------1---------2---------3---------4---------5---------6---------7---------8
Il campo \textsl{Gruppo/Interfaccia Ripetuto} è simile al campo 
gruppo/interfaccia, con la differenza che sono presenti $n$ occorrenze del 
gruppo. Oltre ai parametri usati dal campo gruppo/interfaccia è presente il 
parametro \verb!times!, che indica il numero di volte che il gruppo è ripetuto.

\ifesource
\begin{figure*}[!htb]
\begin{lstlisting}[language=java, 
caption=interfaccia OccTraitModel (campo gruppo/interfaccia ripetuto), 
label=lst:OccTraitModel]
public interface OccTraitModel extends FieldModel {
    void setOffset(int offset);
    void setLength(int length);
    void setName(String name);
    
    void setOverride(boolean override);
    
    void setTypedef((*\hyperref[lst:TraitModel]{TraitModel}*) typedef);
    void setTimes(int times);
}
\end{lstlisting}\index{OccTraitModel}
\end{figure*}
\else
\begin{elisting}
\begin{javacode}
public interface OccTraitModel extends FieldModel {
    void setOffset(int offset);
    void setLength(int length);
    void setName(String name);
    
    void setOverride(boolean override);
    
    void setTypedef(|\hyperref[lst:TraitModel]{TraitModel}| typedef);
    void setTimes(int times);
}
\end{javacode}
\caption{interfaccia OccTraitModel (campo gruppo/interfaccia ripetuto)}
\label{lst:OccTraitModel}
\end{elisting}
\fi
%--------1---------2---------3---------4---------5---------6---------7---------8
Come per il caso del gruppo ripetuto l'\,area della stringa-dati definita 
da \verb!offset! e \verb!length! è quella della prima occorrenza del gruppo. 
La dimensione realmente usata è \verb!length! $\times$ \verb!times!.

\subsection{enum utilizzati}
%--------1---------2---------3---------4---------5---------6---------7---------8
Molte delle classi o interfacce di configurazione hanno dei campi con valori
limitati ad alcuni valori espresse mediante \texttt{enum}.
Vediamoli uno per uno.

\ifesource
\begin{lstlisting}[language=java, 
caption=enum LoadOverflowAction, 
label=lst:LoadOverflowAction]
public enum LoadOverflowAction { Error, Trunc }
\end{lstlisting}\index{LoadOverflowAction}
\else
\begin{elisting}
\begin{javacode}
public enum LoadOverflowAction { Error, Trunc }
\end{javacode}
\caption{enum LoadOverflowAction}\label{lst:LoadOverflowAction}
\end{elisting}
\fi
%--------1---------2---------3---------4---------5---------6---------7---------8
L'\,enum \hyperref[lst:LoadOverflowAction]{\texttt{LoadOverflowAction}} è usato 
dalla classe \hyperref[lst:ClassModel]{\texttt{ClassModel}} per indicare come 
comportarsi quando la classe viene deserializzata e la stringa-dati ha una 
dimensione maggiore di quella attesa dalla classe-dati.

\ifesource
\begin{lstlisting}[language=java, 
caption=enum LoadUnderflowAction, 
label=lst:LoadUnderflowAction]
public enum LoadUnderflowAction { Error, Pad }
\end{lstlisting}\index{LoadUnderflowAction}
\else
\begin{elisting}
\begin{javacode}
public enum LoadUnderflowAction { Error, Pad }
\end{javacode}
\caption{enum LoadUnderflowAction}\label{lst:LoadUnderflowAction}
\end{elisting}
\fi
%--------1---------2---------3---------4---------5---------6---------7---------8
L'\,enum \hyperref[lst:LoadUnderflowAction]{\texttt{LoadUnderflowAction}} è 
usato dalla classe \hyperref[lst:ClassModel]{\texttt{ClassModel}} per indicare 
come comportarsi quando la classe viene deserializzata e la stringa-dati ha una 
dimensione inferiore a quella attesa dalla classe-dati.

\ifesource
\begin{lstlisting}[language=java, 
caption=enum CheckAbc, 
label=lst:CheckAbc]
public enum CheckAbc { None, Ascii, Latin1, Valid }
\end{lstlisting}\index{CheckAbc}
\else
\begin{elisting}
\begin{javacode}
public enum CheckAbc { None, Ascii, Latin1, Valid }
\end{javacode}
\caption{enum CheckAbc}\label{lst:CheckAbc}
\end{elisting}
\fi
%--------1---------2---------3---------4---------5---------6---------7---------8
L'\,enum \hyperref[lst:CheckAbc]{\texttt{CheckAbc}} è usato dalla classe
\hyperref[lst:AbcModel]{\texttt{AbcModel}} per indicare quali caratteri sono
considerati validi.

\ifesource
\begin{lstlisting}[language=java, 
caption=enum OverflowAction, 
label=lst:OverflowAction]
public enum OverflowAction { Error, Trunc }
\end{lstlisting}\index{OverflowAction}
\else
\begin{elisting}
\begin{javacode}
public enum OverflowAction { Error, Trunc }
\end{javacode}
\caption{enum OverflowAction}\label{lst:OverflowAction}
\end{elisting}
\fi
%--------1---------2---------3---------4---------5---------6---------7---------8
L'\,enum \hyperref[lst:OverflowAction]{\texttt{OverflowAction}} è usato dalla 
classi per gestire i campi numerici e alfanumerici 
(\hyperref[lst:AbcModel]{\texttt{AbcModel}},\break
\hyperref[lst:NumModel]{\texttt{NumModel}},
\hyperref[lst:CusModel]{\texttt{CusModel}},
\hyperref[lst:NuxModel]{\texttt{NuxModel}}), per indicare come comportarsi
quando il setter propone un valore che ha una dimensione maggiore di quella
attesa per quel campo.

\ifesource
\begin{lstlisting}[language=java, 
caption=enum UnderflowAction, 
label=lst:UnderflowAction]
public enum UnderflowAction { Error, Pad }
\end{lstlisting}\index{UnderflowAction}
\else
\begin{elisting}
\begin{javacode}
public enum UnderflowAction { Error, Pad }
\end{javacode}
\caption{enum UnderflowAction}\label{lst:UnderflowAction}
\end{elisting}
\fi
%--------1---------2---------3---------4---------5---------6---------7---------8
L'\,enum \hyperref[lst:UnderflowAction]{\texttt{UnderflowAction}} è usato dalla 
classi per gestire i campi numerici e alfanumerici 
(\hyperref[lst:AbcModel]{\texttt{AbcModel}},\break
\hyperref[lst:NumModel]{\texttt{NumModel}},
\hyperref[lst:CusModel]{\texttt{CusModel}},
\hyperref[lst:NuxModel]{\texttt{NuxModel}}), per indicare come comportarsi
quando il setter propone un valore che ha una dimensione inferiore a quella
attesa per quel campo.

\ifesource
\begin{lstlisting}[language=java, 
caption=enum NormalizeAbcMode, 
label=lst:NormalizeAbcMode]
public enum NormalizeAbcMode { None, Trim, Trim1 }
\end{lstlisting}\index{NormalizeAbcMode}
\else
\begin{elisting}
\begin{javacode}
public enum NormalizeAbcMode { None, Trim, Trim1 }
\end{javacode}
\caption{enum NormalizeAbcMode}\label{lst:NormalizeAbcMode}
\end{elisting}
\fi
%--------1---------2---------3---------4---------5---------6---------7---------8
L'\,enum \hyperref[lst:NormalizeAbcMode]{\texttt{NormalizeAbcMode}} è usato 
dalla classi per gestire i campi alfanumerici 
(\hyperref[lst:AbcModel]{\texttt{AbcModel}},
\hyperref[lst:CusModel]{\texttt{CusModel}}), per indicare come normalizzare
il valore restituito dal getter.

\ifesource
\begin{lstlisting}[language=java, 
caption=enum NormalizeNumMode, 
label=lst:NormalizeNumMode]
public enum NormalizeNumMode { None, Trim }
\end{lstlisting}\index{NormalizeNumMode}
\else
\begin{elisting}
\begin{javacode}
public enum NormalizeNumMode { None, Trim }
\end{javacode}
\caption{enum NormalizeNumMode}\label{lst:NormalizeNumMode}
\end{elisting}
\fi
%--------1---------2---------3---------4---------5---------6---------7---------8
L'\,enum \hyperref[lst:NormalizeNumMode]{\texttt{NormalizeNumMode}} è usato 
dalla classi per gestire i campi numerici 
(\hyperref[lst:NumModel]{\texttt{NumModel}},
\hyperref[lst:NuxModel]{\texttt{NuxModel}}), per indicare come normalizzare
il valore restituito dal getter.

\ifesource
\begin{lstlisting}[language=java, 
caption=enum WordWidth, 
label=lst:WordWidth]
public enum WordWidth { Byte, Short, Int, Long }
\end{lstlisting}\index{WordWidth}
\else
\begin{elisting}
\begin{javacode}
public enum WordWidth { Byte, Short, Int, Long }
\end{javacode}
\caption{enum WordWidth}\label{lst:WordWidth}
\end{elisting}
\fi
%--------1---------2---------3---------4---------5---------6---------7---------8
L'\,enum \hyperref[lst:WordWidth]{\texttt{WordWidth}} è usato 
dalla classi per gestire i campi numerici 
(\hyperref[lst:NumModel]{\texttt{NumModel}},
\hyperref[lst:NuxModel]{\texttt{NuxModel}}), per indicare il tipo dato numerico
primitivo di dimensione minima da usare quando vengono creati setter e getter
numerici.

\ifesource
\begin{lstlisting}[language=java, 
caption=enum AccesMode, 
label=lst:AccesMode]
public enum AccesMode { String, Number, Both }
\end{lstlisting}\index{AccesMode}
\else
\begin{elisting}
\begin{javacode}
public enum AccesMode { String, Number, Both }
\end{javacode}
\caption{enum AccesMode}\label{lst:AccesMode}
\end{elisting}
\fi
%--------1---------2---------3---------4---------5---------6---------7---------8
L'\,enum \hyperref[lst:AccesMode]{\texttt{AccesMode}} è usato 
dalla classi per gestire i campi numerici 
(\hyperref[lst:NumModel]{\texttt{NumModel}},
\hyperref[lst:NuxModel]{\texttt{NuxModel}}), per indicare se creare setter e
getter alfanumerici, numerici o entrambi.


\ifesource
\begin{lstlisting}[language=java, 
caption=enum InitializeNuxMode, 
label=lst:InitializeNuxMode]
public enum InitializeNuxMode { Spaces, Zeroes }
\end{lstlisting}\index{InitializeNuxMode}
\else
\begin{elisting}
\begin{javacode}
public enum InitializeNuxMode { Spaces, Zeroes }
\end{javacode}
\caption{enum InitializeNuxMode}\label{lst:InitializeNuxMode}
\end{elisting}
\fi
%--------1---------2---------3---------4---------5---------6---------7---------8
L'\,enum \hyperref[lst:InitializeNuxMode]{\texttt{InitializeNuxMode}} è usato 
dalla classe \hyperref[lst:NuxModel]{\texttt{NuxModel}}, per indicare come 
inizializzare il campo quando la classe-dati viene creata col costruttore senza
argomenti.

\ifesource
\begin{lstlisting}[language=java, 
caption=enum CheckCus, 
label=lst:CheckCus]
public enum CheckCus { None, Ascii, Latin1, Valid, Digit, DigitOrBlank }
\end{lstlisting}\index{CheckCus}
\else
\begin{elisting}
\begin{javacode}
public enum CheckCus { None, Ascii, Latin1, Valid, Digit, DigitOrBlank }
\end{javacode}
\caption{enum CheckCus}\label{lst:CheckCus}
\end{elisting}
\fi
%--------1---------2---------3---------4---------5---------6---------7---------8
L'\,enum \hyperref[lst:CheckCus]{\texttt{CheckCus}} è usato dalla classe
\hyperref[lst:CusModel]{\texttt{CusModel}} per indicare quali caratteri sono
considerati validi.



\ifesource
\begin{lstlisting}[language=java, 
caption=enum AlignMode, 
label=lst:AlignMode]
public enum AlignMode { LFT, RGT }
\end{lstlisting}\index{AlignMode}
\else
\begin{elisting}
\begin{javacode}
public enum AlignMode { LFT, RGT }
\end{javacode}
\caption{enum AlignMode}\label{lst:AlignMode}
\end{elisting}
\fi
%--------1---------2---------3---------4---------5---------6---------7---------8
L'\,enum \hyperref[lst:CheckCus]{\texttt{CheckCus}} è usato dalla classe
\hyperref[lst:CusModel]{\texttt{CusModel}} per indicare come allineare il campo.


\vfill


%  ____                  _          
% / ___|  ___ _ ____   _(_) ___ ___ 
% \___ \ / _ \ '__\ \ / / |/ __/ _ \
%  ___) |  __/ |   \ V /| | (_|  __/
% |____/ \___|_|    \_/ |_|\___\___|
%                                  

\chapter{Service}
%--------1---------2---------3---------4---------5---------6---------7---------8
L'\,interfaccia fissa semplicemente la struttura generale, ma contiene solo
interfacce e java-bean.
Il \textsl{Service} è l'\,applicazione e eseguirà quanti richiesto.
Per questo utilizzerà il \textsl{ServiceLoader} per cercare per cercare nel 
classpath un \textsl{Service Provider} che implementi la \textsl{Service 
Provider Interface}.

%--------1---------2---------3---------4---------5---------6---------7---------8
Il cuore del \textsl{Service Provider Interface} \verb!recfm-addon-api! è la
interfaccia \textsl{CodeProvider}. L'\,implementazione della interfaccia
viene cercata col meccanismo del ServiceLoader, cod.\ref{lst:getCodeProvider}.

\ifesource
\begin{figure*}[!htb]
\begin{lstlisting}[language=java, caption=recupero del CodeProvider, 
label=lst:getCodeProvider]
        ServiceLoader<CodeProvider> loader = ServiceLoader.load(CodeProvider.class);
        (*\hyperref[lst:CodeProvider]{CodeProvider}*) codeProvider = loader.iterator().next();
\end{lstlisting}
\end{figure*}
\else
\begin{elisting}
\begin{javacode}
        ServiceLoader<CodeProvider> loader = ServiceLoader.load(CodeProvider.class);
        |\hyperref[lst:CodeProvider]{CodeProvider}| codeProvider = loader.iterator().next();
\end{javacode}
\caption{recupero del CodeProvider}
\label{lst:getCodeProvider}
\end{elisting}
\fi

%--------1---------2---------3---------4---------5---------6---------7---------8
Una volta recuperata una istanza di \textsl{CodeFactory} è possibile creare le
definizioni delle stringhe-dati e generare i sorgenti delle classi-dati.

%--------1---------2---------3---------4---------5---------6---------7---------8
Sono stati sviluppati due \textsl{client}, uno sotto forma di maven plugin 
\verb!recfm-maven-plugin!, e l'\,altro sotto forma di gradle plugin 
\verb!recfm-gradle-plugin!. Il codice in gran parte è identico, cambia solo il 
meccanismo di innesco.

\section{Maven plugin}
%--------1---------2---------3---------4---------5---------6---------7---------8
Il maven plugin \verb!recfm-maven-plugin! permette di creare più classi e 
interfacce usando uno più file di configurazione in formato yaml.
Le librerie esterne utilizzate richiedono il java 8, quindi per eseguire questo 
plugin è necessario almeno il java 8.

Il plugin si aspetta come parametri di configurazione
\ifesource
\begin{figure*}[!htb]
\begin{lstlisting}[language=java, caption=parametri impostabili del maven plugin, 
label=lst:spring-conf]
    @Parameter(defaultValue = "${project.build.directory}/generated-sources/recfm",
        property = "maven.recfm.generateDirectory", required = true)
    private File generateDirectory;

    @Parameter(defaultValue = "${project.build.resources[0].directory}",
        property = "maven.recfm.settingsDirectory", required = true)
    private File settingsDirectory;
    
    @Parameter(required = true)
    private String[] settings;

    @Parameter(defaultValue = "true", property = "maven.recfm.addCompileSourceRoot")
    private boolean addCompileSourceRoot = true;

    @Parameter(defaultValue = "false", property = "maven.recfm.addTestCompileSourceRoot")
    private boolean addTestCompileSourceRoot = false;
    
    @Parameter
    private String codeProviderClassName;
\end{lstlisting}
\end{figure*}
\else
\begin{elisting}
\begin{javacode}
    @Parameter(defaultValue = "${project.build.directory}/generated-sources/recfm",
        property = "maven.recfm.generateDirectory", required = true)
    private String generateDirectory;

    @Parameter(defaultValue = "${project.build.resources[0].directory}",
        property = "maven.recfm.settingsDirectory", required = true)
    private String settingsDirectory;
    
    @Parameter(required = true)
    private String[] settings;

    @Parameter(defaultValue = "true", property = "maven.recfm.addCompileSourceRoot")
    private boolean addCompileSourceRoot = true;

    @Parameter(defaultValue = "false", property = "maven.recfm.addTestCompileSourceRoot")
    private boolean addTestCompileSourceRoot = false;
    
    @Parameter
    private String codeProviderClassName;
\end{javacode}
\caption{parametri impostabili del maven plugin}
\label{lst:spring-conf}
\end{elisting}
\fi

\begin{itemize}
%--------1---------2---------3---------4---------5---------6---------7---------8
\item \index{plugin!generateDirectory}
	Il campo \fcolorbox{black}{yellow!75}{\texttt{generateDirectory}} indica la 
	directory root da utilizzare per la generazione dei sorgenti, viene usato 
	per valorizzare il campo \verb!sourceDirectory! della classe 
	\hyperref[lst:GenerateArgs]{\texttt{GenerateArgs}}, come si vede dalla 
	definizione, se il parametro è omesso viene utilizzata la directory 
	\verb!target/generated-sources/recfm!, 	normalmente può essere lasciato il 
	valore di default.
	Gli altri tre parametri di \texttt{GenerateArgs} sono un identificativo del
	programma \textsl{service} e vengono valorizzati automaticamente.
%--------1---------2---------3---------4---------5---------6---------7---------8
\item \index{plugin!settingsDirectory}
	Il campo \fcolorbox{black}{yellow!75}{\texttt{settingsDirectory}} indica la 
	directory che contiene i file	di configurazione, se il parametro è omesso 
	viene usato il valore \verb!src/main/resources!, normalmente può essere 
	lasciato il valore di default.
%--------1---------2---------3---------4---------5---------6---------7---------8
\item \index{plugin!settings}
	Il campo \fcolorbox{black}{yellow!75}{\texttt{settings}} indica l'elenco dei 
	file di configurazione da utilizzare per generare le classi/interfacce; il 
	parametro deve essere fornito al plugin.
%--------1---------2---------3---------4---------5---------6---------7---------8
\item \index{plugin!addCompileSourceRoot}
	Il campo \fcolorbox{black}{yellow!75}{\texttt{addCompileSourceRoot}} è un 
	campo tecnico, indica a maven che la directory dove sono stati generati i 
	sorgenti deve essere inclusa tra quelle utilizzate per la compilazione 
	principale, se il parametro è omesso viene utilizzato il valore 
	\verb!true!; il valore \verb!true! è opportuno quando viene usata una 
	directory di generazione del codice diversa da \verb!src/main/java!, 
	altrimenti è necessario usare plugin aggiuntivi per aggiungere il nuovo path 
	a quello di compilazione di maven.
%--------1---------2---------3---------4---------5---------6---------7---------8
\item \index{plugin!addTestCompileSourceRoot}
	Il campo \fcolorbox{black}{yellow!75}{\texttt{addTestCompileSourceRoot}} è 
	analogo al campo precedente, ma aggiunge la directory di generazione al path 
	di compilazione dei test, se omesso viene utilizzato il valore \verb!false!, 
	tranne in casi particolari può essere lasciato il valore di default.
%--------1---------2---------3---------4---------5---------6---------7---------8
\item \index{plugin!codeProviderClassName}
	Il campo \fcolorbox{black}{yellow!75}{\texttt{codeProviderClassName}} indica 
	quale è la classe concreata che implementa la \textsl{Service Interface}, se 
	omesso viene utilizzata la ``prima'' implementazione recuperata del 
	\textsl{ServiceLoader}; se in classpath è presente una sola implementazione, 
	non è necessario valorizzare il parametro. 
\end{itemize}
%--------1---------2---------3---------4---------5---------6---------7---------8

%--------1---------2---------3---------4---------5---------6---------7---------8
\ifesource
\begin{figure*}[!htb]
\begin{lstlisting}[language=XML, caption=esempio minimale di esecuzione del 
plugin, label=lst:mvn-xmpl]
<plugin>
    <groupId>io.github.epi155</groupId>
    <artifactId>recfm-maven-plugin</artifactId>
    <version>0.7.0</version>
    <executions>
        <execution>
            <goals>
                <goal>generate</goal>
            </goals>
            <configuration>
                <settings>
                    <setting>foo.yaml</setting>
                    <setting>bar.yaml</setting>
                </settings>
            </configuration>
        </execution>
    </executions>
    <dependencies>
        <dependency>
            <groupId>io.github.epi155</groupId>
            <artifactId>recfm-java-addon</artifactId>
            <version>0.7.0</version>
        </dependency>
    </dependencies>
</plugin>
\end{lstlisting}
\end{figure*}
\else
\begin{elisting}
\begin{minted}[frame=single,framesep=2mm,bgcolor=bg,fontsize=\footnotesize]{xml}
<plugin>
    <groupId>io.github.epi155</groupId>
    <artifactId>recfm-maven-plugin</artifactId>
    <version>0.7.0</version>
    <executions>
        <execution>
            <goals>
                <goal>generate</goal>
            </goals>
            <configuration>
                <settings>
                    <setting>foo.yaml</setting>
                    <setting>bar.yaml</setting>
                </settings>
            </configuration>
        </execution>
    </executions>
    <dependencies>
        <dependency>
            <groupId>io.github.epi155</groupId>
            <artifactId>recfm-java-addon</artifactId>
            <version>0.7.0</version>
        </dependency>
    </dependencies>
</plugin>
\end{minted}
\caption{esempio minimale di esecuzione del plugin}
\label{lst:mvn-xmpl}
\end{elisting}
\fi
%--------1---------2---------3---------4---------5---------6---------7---------8
Un esempio di esecuzione del plugin è mostrato nel cod.~\ref{lst:mvn-xmpl},
il plugin per essere eseguito deve avere come dipendenza una libreria che 
fornisca l'\,implementazione dell'\,inter\-fac\-cia, altrimenti il 
\verb!ServiceLoader! non trova nulla ed il plugin termina in errore.

%--------1---------2---------3---------4---------5---------6---------7---------8
Tutti gli altri parametri sono forniti nei file di configurazione.

\subsection{Struttura del file di configurazione}
%--------1---------2---------3---------4---------5---------6---------7---------8
Per gestire la configurazione dei tracciati il plugin definisce la classe
\textsl{MasterBook}, vedi cod.~\ref{lst:MasterBook}, è divisa in due componenti,
la prima \texttt{defaults} è semplicemente il java-bean \textsl{FieldDefault}
(vedi~\ref{lst:FieldDefault}) messo a disposizione dalla 
\textsl{Service Provider Interface} per fornire i valori di default dei 
parametri ``poco variabili'' delle classi e dei campi.

\ifesource
\begin{figure*}[!htb]
\begin{lstlisting}[language=java, caption=classe di configurazione MasterBook, 
label=lst:MasterBook]
@Data
public class MasterBook {
    private (*\hyperref[lst:FieldDefault]{FieldDefault}*) defaults = new FieldDefault();
    private List<(*\hyperref[lst:ClassPackage]{ClassPackage}*)> packages = new ArrayList<>();
}
\end{lstlisting}
\end{figure*}
\else
\begin{elisting}
\begin{javacode}
@Data
public class MasterBook {
    private |\hyperref[lst:FieldDefault]{FieldDefault}| defaults = new FieldDefault();
    private List<|\hyperref[lst:ClassPackage]{ClassPackage}|> packages = new ArrayList<>();
}
\end{javacode}
\caption{classe di configurazione MasterBook}
\label{lst:MasterBook}
\end{elisting}
\fi

%--------1---------2---------3---------4---------5---------6---------7---------8
Per semplificare la valorizzazione del file di configurazione yaml, viene usata
un funzionalità delle librerie yaml, che permette di definire nomi abbreviati o
alternativi dei parametri e dei valori dei campi di tipo enum.
I dettagli della componenti del campo \texttt{defaults} sarà mostrato insieme al
campo a fornisce il default del valore dei parametri.

\iffalse
\subsubsection{Default per classi (interfacce)}
\begin{table}[!htb]
\centering
\begin{tabular}{|>{\small\tt}l|>{\small\tt}c|>{\small\tt}c|>{\small\tt}l|}
\hline
\multicolumn{4}{|c|}{cls: ClsDefault }\\
\hline
\multicolumn{1}{|c|}{attributo} & \multicolumn{1}{c|}{alt} 
	& \multicolumn{1}{c|}{tipo} & \multicolumn{1}{c|}{default} \\
\hline
\hline
onOverflow & ovf & enum    & Trunc \\
\hline
onUnderflow& unf & enum    & Pad   \\
\hline
doc        &     & boolean & true\\
\hline
\end{tabular}
%--------1---------2---------3---------4---------5---------6---------7---------8
\caption{default per le classi/interfacce} 
\label{tab:dflt.cls}
\end{table}

\subsubsection{Default per campi alfanumerici}

\subsubsection{Default per campi numerici}
\begin{table}[!htb]
\centering
\begin{tabular}{|>{\small\tt}l|>{\small\tt}c|>{\small\tt}c|>{\small\tt}l|}
\hline
\multicolumn{4}{|c|}{num: NumDefault }\\
\hline
\multicolumn{1}{|c|}{attributo} & \multicolumn{1}{c|}{alt} 
	& \multicolumn{1}{c|}{tipo} & \multicolumn{1}{c|}{default} \\
\hline
\hline
onOverflow & ovf & enum & Trunc \\
\hline
onUnderflow& unf & enum & Pad   \\
\hline
normalize  & nrm & enum & None \\
\hline
wordWidth  & wid & enum & W4 \\
\hline
access     & acc & enum & String \\
\hline
\end{tabular}
%--------1---------2---------3---------4---------5---------6---------7---------8
\caption{default per campi numerici} 
\label{tab:dflt.num}
\end{table}
\fi

%--------1---------2---------3---------4---------5---------6---------7---------8

\ifesource
\begin{figure*}[!htb]
\begin{lstlisting}[language=java, caption=classe di configurazione ClassPackage, 
label=lst:ClassPackage]
@Data
public class ClassPackage {
    private String name;     // package name
    private List<(*\hyperref[lst:TraitModel]{TraitModel}*)> interfaces = new ArrayList<>();
    private List<(*\hyperref[lst:ClassModel]{ClassModel}*)> classes = new ArrayList<>();
}
\end{lstlisting}
\end{figure*}
\else
\begin{elisting}
\begin{javacode}
@Data
public class ClassPackage {
    private String name;     // package name
    private List<|\hyperref[lst:TraitModel]{TraitModel}|> interfaces = new ArrayList<>();
    private List<|\hyperref[lst:ClassModel]{ClassModel}|> classes = new ArrayList<>();
}
\end{javacode}
\caption{classe di configurazione ClassPackage}
\label{lst:ClassPackage}
\end{elisting}
\fi
%--------1---------2---------3---------4---------5---------6---------7---------8
La seconda componente di \textsl{MasterBook}, \texttt{packages}, è una lista di
\textsl{ClassPackage} (\ref{lst:ClassPackage}), cioè di package all'\,in\-ter\-no
dei quali vengono definiti un elenco di interfacce e classi.
Espandendo un esempio di questo oggetto in formato yaml (con il default 
relativo) abbiamo:

\ifesource
\begin{figure*}[!htb]
\begin{lstlisting}[language=yaml, 
caption={configurazione, area packages / interfaces / classes}, 
label=lst:pakg-conf]
defaults:
  cls:
    onOverflow: Trunc   # :ovf: Error, Trunc
    onUnderflow: Pad    # :unf: Error, Pad
    doc: true
packages:
  - name: com.example.test  # package name
    interfaces:
      - &IFoo         (*\color{purple}{\# interface reference}*)
        name: IFoo    # interface name
        length: 12    # :len: interface length
        fields:
          - ...
    classes:
      - name: Foo           # class name
        length: 10          # :len: class length
        onOverflow: Trunc   # :ovf: Error, Trunc
        onUnderflow: Pad    # :unf: Error, Pad
        fields:
          - ...
\end{lstlisting}
\end{figure*}
\else
\begin{elisting}
\begin{yamlcode}
defaults:
  cls:
    onOverflow: Trunc   # :ovf: Error, Trunc
    onUnderflow: Pad    # :unf: Error, Pad
    doc: true
packages:
  - name: com.example.test  # package name
    interfaces:
      - &IFoo         # interface reference
        name: IFoo    # interface name
        length: 12    # :len: interface length
        fields:
          - ...
    classes:
      - name: Foo           # class name
        length: 10          # :len: class length
        onOverflow: Trunc   # :ovf: Error, Trunc
        onUnderflow: Pad    # :unf: Error, Pad
        fields:
          - ...
\end{yamlcode}
\caption{configurazione, area packages / interfaces / classes}
\label{lst:pakg-conf}
\end{elisting}
\fi
%--------1---------2---------3---------4---------5---------6---------7---------8
Nei commenti vengono mostrati gli eventuali nomi alternativi dei campi e 
l'\,elenco dei valori \textsl{enum} permessi.
Se non vengono usate interfacce, il nodo \texttt{interfaces} può essere omesso,
sia per le classi che le interfacce il nome e la lunghezza del tracciato da 
associare devono essere impostate dall'\,utente, nella definizione della classe
può anche essere impostato il comportamento nel caso che venga fornita in fase
di de-serializzazione una struttura con una dimensione maggiore o minore di 
quella attesa.

\begin{table}[!htb]
\centering
\begin{tabular}{|>{\tt}l|>{\tt}c|>{\tt}c|l|}
\hline
\multicolumn{4}{|c|}{TraitModel --- interfaces}\\
\hline
\multicolumn{1}{|c|}{attributo} & \multicolumn{1}{c|}{alt} 
	& \multicolumn{1}{c|}{tipo} & \multicolumn{1}{c|}{note} \\
\hline
\hline
name       &     & String  & obbligatorio \\
\hline
length     & len & int     & obbligatorio \\
\hline
doc        &     & boolean & default \texttt{\$\{defaults.cls.doc:true\}}\\
\hline
fields     &     & array & obbligatorio \\
\hline
\end{tabular}
%--------1---------2---------3---------4---------5---------6---------7---------8
\caption{Attributi impostabili per la definizione di una interfaccia} 
\label{tab:attr.trait}
\end{table}


%--------1---------2---------3---------4---------5---------6---------7---------8
Anche se tutti i tipi di campo hanno una posizione di inizio e una lunghezza,
il dettaglio dei parametri di configurazione varia da campo a campo ed è 
necessario introdurre i parametri di configurazione campo per campo.

\begin{table}[!htb]
\centering
\begin{tabular}{|>{\tt}l|>{\tt}c|>{\tt}c|l|}
\hline
\multicolumn{4}{|c|}{ClassModel --- classes}\\
\hline
\multicolumn{1}{|c|}{attributo} & \multicolumn{1}{c|}{alt} 
	& \multicolumn{1}{c|}{tipo} & \multicolumn{1}{c|}{note} \\
\hline
\hline
name       &     & String  & obbligatorio \\
\hline
length     & len & int     & obbligatorio \\
\hline
onOverflow & ovf & enum    & default \texttt{\$\{defaults.cls.onOverflow:Trunc\}}\\
\hline
onUnderlow & unf & enum    & default \texttt{\$\{defaults.cls.onUnderflow:Pad\}}\\
\hline
doc        &     & boolean & default \texttt{\$\{defaults.cls.doc:true\}}\\
\hline
fields     &     & array & obbligatorio \\
\hline
\end{tabular}
%--------1---------2---------3---------4---------5---------6---------7---------8
\caption{Attributi impostabili per la definizione di una classe} 
\label{tab:attr.class}
\end{table}

%--------1---------2---------3---------4---------5---------6---------7---------8
Per indicare esplicitamente il tipo di campo utilizzato vengono introdotti dei
\textsl{tag} da associare ad ogni campo

\begin{table}[!htb]
\centering
\begin{tabular}{|>{\tt}l|>{\tt}l|l|}
\hline
\multicolumn{3}{|c|}{Tag definizione campo}\\
\hline
\multicolumn{1}{|c|}{tag} & \multicolumn{1}{c|}{classe} 
	& \multicolumn{1}{c|}{note} \\
\hline
\hline
\hyperref[sub:yaml.abc]{!Abc} & \hyperref[lst:AbcModel]{AbcModel} & campo alfanumerico \\
\hline
\hyperref[sub:yaml.num]{!Num} & \hyperref[lst:NumModel]{NumModel} & campo numerico \\
\hline
\hyperref[sub:yaml.cus]{!Cus} & \hyperref[lst:CusModel]{CusModel} & campo custom \\
\hline
\hyperref[sub:yaml.nux]{!Nux} & \hyperref[lst:NuxModel]{NuxModel} & campo numerico nullabile \\
\hline
\hyperref[sub:yaml.dom]{!Dom} & \hyperref[lst:DomModel]{DomModel} & campo dominio \\
\hline
\hyperref[sub:yaml.fil]{!Fil} & \hyperref[lst:FilModel]{FilModel} & campo filler \\
\hline
\hyperref[sub:yaml.val]{!Val} & \hyperref[lst:ValModel]{ValModel} & campo costante \\
\hline
\hyperref[sub:yaml.grp]{!Grp} & \hyperref[lst:GrpModel]{GrpModel} & gruppo di campi \\
\hline
\hyperref[sub:yaml.occ]{!Occ} & \hyperref[lst:OccModel]{OccModel} & gruppo di campi ripetuti \\
\hline
\hyperref[sub:yaml.emb]{!Emb} & \hyperref[lst:EmbModel]{EmbModel} & campi inclusi da interfaccia \\
\hline
\hyperref[sub:yaml.igrp]{!GRP} & \hyperref[lst:GrpTraitModel]{GrpTraitModel} & gruppo di campi inclusi da interfaccia \\
\hline
\hyperref[sub:yaml.iocc]{!OCC} & \hyperref[lst:OccTraitModel]{OccTraitModel} & gruppo di campi ripetuti inclusi da interfaccia\\
\hline
\end{tabular}
%--------1---------2---------3---------4---------5---------6---------7---------8
\caption{Tag yaml per la identificazione del campo} 
\label{tab:tag.class}
\end{table}



\subsubsection{Campo Alfanumerico} \label{sub:yaml.abc}
%--------1---------2---------3---------4---------5---------6---------7---------8
La definizione yaml del campo alfanumerico riflette la struttura imposta dalla
service provider interface, vedi~\ref{lst:AbcModel}.
Un esmpio di definizione di campi numerici è mostrato nel 
cod.~\ref{lst:xmplAbc}, nell'\,esempio è mostrato anche il nodo del default
globale per i campi alfanumerici, i valori impostati sono quelli di default
della \textsl{service provider interface}, quindi non è necessario impostare
esplicitamente i parametri se si vuole impostare questi valori.

\ifesource
\begin{figure*}[!htb]
\begin{lstlisting}[language=yaml, caption={esempio definizione campi alfanumerici}, 
label=lst:xmplAbc]
defaults:
  abc:
    check: Ascii        # :chk: None, Ascii, Latin1, Valid
    onOverflow: Trunc   # :ovf: Error, Trunc
    onUnderflow: Pad    # :unf: Error, Pad
    normalize: None     # :nrm: None, Trim, Trim1
    checkGetter: true   # :get:
    checkSetter: true   # :set:
packages:
  - name: com.example.test
    classes:
      - name: Foo3111
        length: 50
        fields:
          - !Abc { name: cognome    , at:   1, len:    25 }
          - !Abc { name: nome       , at:  26, len:    20 }
          - !Abc { name: stCivile   , at:  46, len:     1 }
          - !Abc { name: nazionalita, at:  47, len:     3 }
          - !Abc { name: sesso      , at:  50, len:     1 }
\end{lstlisting}
\end{figure*}
\else
\begin{elisting}
\begin{yamlcode}
defaults:
  abc:
    check: Ascii        # :chk: None, Ascii, Latin1, Valid
    onOverflow: Trunc   # :ovf: Error, Trunc
    onUnderflow: Pad    # :unf: Error, Pad
    normalize: None     # :nrm: None, Trim, Trim1
    checkGetter: true   # :get:
    checkSetter: true   # :set:
packages:
  - name: com.example.test
    classes:
      - name: Foo3111
        length: 50
        fields:
          - !Abc { name: cognome    , at:   1, len:    25 }
          - !Abc { name: nome       , at:  26, len:    20 }
          - !Abc { name: stCivile   , at:  46, len:     1 }
          - !Abc { name: nazionalita, at:  47, len:     3 }
          - !Abc { name: sesso      , at:  50, len:     1 }
\end{yamlcode}
\caption{esempio definizione campi alfanumerici}
\label{lst:xmplAbc}
\end{elisting}
\fi

%--------1---------2---------3---------4---------5---------6---------7---------8
Nell'\,esempio, il nodo di default dei campi alfanumerici, è impostato usando
i nomi canonici dei parametri. Il \textit{plugin} usa una funzionalià 
disponibile della libreria per leggere il file yaml, e definisce dei nomi
abbreviati dei parametri, che possono essere utilizzati come alternativa ai
nomi canonici.

\begin{table}[!htb]
\centering
\begin{tabular}{|c|>{\tt}l|>{\tt}c|>{\tt}c|c|l|}
\cline{2-6} \multicolumn{1}{c|}{}
&\multicolumn{5}{c|}{\texttt{!Abc}: \hyperref[lst:AbcModel]{AbcModel}}\\
\cline{2-6} \multicolumn{1}{c|}{}
&\multicolumn{1}{c|}{attributo} & \multicolumn{1}{c|}{alt} 
	& \multicolumn{1}{c|}{tipo} & \multicolumn{1}{c|}{O}
	& \multicolumn{1}{c|}{default} \\
\cline{2-6} \multicolumn{1}{c|}{}
&offset     & at  & int     & \ding{52} & \\
\cline{2-6} \multicolumn{1}{c|}{}
&length     & len & int     & \ding{52} & \\
\cline{2-6} \multicolumn{1}{c|}{}
&name       &     & String  & \ding{52} & \\
\cline{2-6} \multicolumn{1}{c|}{}
&override   & ovr & boolean & & \texttt{false} \\
\hline
\parbox[t]{2mm}{\multirow{6}{*}{\rotatebox[origin=c]{90}{\hyperref[lst:AbcDefault]{AbcDefault}}}}
&onOverflow & ovf & \hyperref[lst:OverflowAction]{enum} & & \texttt{\$\{defaults.abc.onOverflow:Trunc\}}\\
\cline{2-6}
&onUnderlow & unf & \hyperref[lst:UnderflowAction]{enum} & & \texttt{\$\{defaults.abc.onUnderflow:Pad\}}\\
\cline{2-6}
&check      & chk & \hyperref[lst:CheckAbc]{enum} & & \texttt{\$\{defaults.abc.check:Ascii\}}\\
\cline{2-6}
&normalize  & nrm & \hyperref[lst:NormalizeAbcMode]{enum} & & \texttt{\$\{defaults.abc.normalize:None\}}\\
\cline{2-6}
&checkGetter & get & boolean & & \texttt{\$\{defaults.abc.checkGetter:true\}}\\
\cline{2-6}
&checkSetter & set & boolean & & \texttt{\$\{defaults.abc.checkSetter:true\}}\\
\hline
\end{tabular}
\caption{Attributi impostabili per un campo alfanumerico} \label{tab:attr.abc}
\end{table}


\subsubsection{Campo Numerico} \label{sub:yaml.num}
%--------1---------2---------3---------4---------5---------6---------7---------8
La definizione yaml del campo numerico riflette la struttura imposta dalla
service interface, vedi~\ref{lst:NumModel}

\ifesource
\begin{figure*}[!htb]
\begin{lstlisting}[language=yaml, caption={esempio definizione campi numerici}, 
label=lst:xmplNum]
defaults:
  num:
    onOverflow: Trunc   # :ovf: Error, Trunc
    onUnderflow: Pad    # :unf: Error, Pad
    normalize: None     # :nrm: None, Trim
    wordWidth: Int      # :wid: Byte(1,byte), Short(2,short), Int(4,int), Long(8,long)
    access: String      # :acc: String(Str), Numeric(Num), Both(All)
packages:
  - name: com.example.test
    classes:
      - name: Foo3112
        length: 8
        doc: No
        fields:
          - !Num { name: year , at: 1, len: 4 }
          - !Num { name: month, at: 5, len: 2 }
          - !Num { name: mday , at: 7, len: 2 }
\end{lstlisting}
\end{figure*}
\else
\begin{elisting}
\begin{yamlcode}
defaults:
  num:
    onOverflow: Trunc   # :ovf: Error, Trunc
    onUnderflow: Pad    # :unf: Error, Pad
    normalize: None     # :nrm: None, Trim
    wordWidth: Int      # :wid: Byte(1,byte), Short(2,short), Int(4,int), Long(8,long)
    access: String      # :acc: String(Str), Numeric(Num), Both(All)
packages:
  - name: com.example.test
    classes:
      - name: Foo3112
        length: 8
        doc: No
        fields:
          - !Num { name: year , at: 1, len: 4 }
          - !Num { name: month, at: 5, len: 2 }
          - !Num { name: mday , at: 7, len: 2 }
\end{yamlcode}
\caption{esempio definizione campi numerici}
\label{lst:xmplNum}
\end{elisting}
\fi


\begin{table}[!htb]
\centering
\begin{tabular}{|c|>{\tt}l|>{\tt}c|>{\tt}c|c|l|}
\cline{2-6} \multicolumn{1}{c|}{}
&\multicolumn{5}{c|}{\texttt{!Num}: \hyperref[lst:NumModel]{NumModel}}\\
\cline{2-6} \multicolumn{1}{c|}{}
&\multicolumn{1}{c|}{attributo} & \multicolumn{1}{c|}{alt} 
	& \multicolumn{1}{c|}{tipo} & \multicolumn{1}{c|}{O}
	& \multicolumn{1}{c|}{default} \\
\cline{2-6} \multicolumn{1}{c|}{}
&offset     & at  & int     & \ding{52} & \\
\cline{2-6} \multicolumn{1}{c|}{}
&length     & len & int     & \ding{52} & \\
\cline{2-6} \multicolumn{1}{c|}{}
&name       &     & String  & \ding{52} & \\
\cline{2-6} \multicolumn{1}{c|}{}
&override   & ovr & boolean & & \texttt{false} \\
\hline
\parbox[t]{2mm}{\multirow{5}{*}{\rotatebox[origin=c]{90}{\hyperref[lst:NumDefault]{NumDefault}}}}
&onOverflow & ovf & \hyperref[lst:OverflowAction]{enum} & & \texttt{\$\{defaults.num.onOverflow:Trunc\}}\\
\cline{2-6}
&onUnderlow & unf & \hyperref[lst:UnderflowAction]{enum} & & \texttt{\$\{defaults.num.onUnderflow:Pad\}}\\
\cline{2-6}
&access     & acc & \hyperref[lst:AccesMode]{enum} & & \texttt{\$\{defaults.num.access:String\}}\\
\cline{2-6}
&wordWidth  & wid & \hyperref[lst:WordWidth]{enum} & & \texttt{\$\{defaults.num.wirdWidth:Int\}}\\
\cline{2-6}
&normalize  & nrm & \hyperref[lst:NormalizeNumMode]{enum} & & \texttt{\$\{defaults.num.normalize:None\}}\\
\hline
\end{tabular}
\caption{Attributi impostabili per un campo numerico} \label{tab:attr.num}
\end{table}



\subsubsection{Campo Custom (alfanumerico)} \label{sub:yaml.cus}
%--------1---------2---------3---------4---------5---------6---------7---------8
La definizione yaml del campo numerico riflette la struttura imposta dalla
service interface, vedi~\ref{lst:CusModel}

\ifesource
\begin{figure*}[!htb]
\begin{lstlisting}[language=yaml, caption={esempio definizione campi custom}, 
label=lst:xmplCus]
defaults:
  cus:
    padChar: ' '        # :pad:
    initChar: ' '       # :ini:
    check: Ascii        # :chk: None, Ascii, Latin1, Valid, Digit, DigitOrBlank
    align: LFT          # LFT, RGT
    onOverflow: Trunc   # :ovf: Error, Trunc
    onUnderflow: Pad    # :unf: Error, Pad
    normalize: None     # :nrm: None, Trim, Trim1
    checkGetter: true   # :get:
    checkSetter: true   # :set:
packages:
  - name: com.example.test
    classes:
      - name: Foo3113
        length: 8
        doc: No
        fields:
          - !Cus { name: year , at: 1, len: 4 }
          - !Cus { name: month, at: 5, len: 2 }
          - !Cus { name: mday , at: 7, len: 2 }
\end{lstlisting}
\end{figure*}
\else
\begin{elisting}
\begin{yamlcode}
defaults:
  cus:
    padChar: ' '        # :pad:
    initChar: ' '       # :ini:
    check: Ascii        # :chk: None, Ascii, Latin1, Valid, Digit, DigitOrBlank
    align: LFT          # LFT, RGT
    onOverflow: Trunc   # :ovf: Error, Trunc
    onUnderflow: Pad    # :unf: Error, Pad
    normalize: None     # :nrm: None, Trim, Trim1
    checkGetter: true   # :get:
    checkSetter: true   # :set:
packages:
  - name: com.example.test
    classes:
      - name: Foo3113
        length: 8
        doc: No
        fields:
          - !Cus { name: year , at: 1, len: 4 }
          - !Cus { name: month, at: 5, len: 2 }
          - !Cus { name: mday , at: 7, len: 2 }
\end{yamlcode}
\caption{esempio definizione campi custom}
\label{lst:xmplCus}
\end{elisting}
\fi

\begin{table}[!htb]
\centering
\begin{tabular}{|c|>{\tt}l|>{\tt}c|>{\tt}c|c|l|}
\cline{2-6} \multicolumn{1}{c|}{}
&\multicolumn{5}{c|}{\texttt{!Cus}: \hyperref[lst:CusModel]{CusModel}}\\
\cline{2-6} \multicolumn{1}{c|}{}
&\multicolumn{1}{c|}{attributo} & \multicolumn{1}{c|}{alt} 
	& \multicolumn{1}{c|}{tipo} & \multicolumn{1}{c|}{O}
	& \multicolumn{1}{c|}{default} \\
\cline{2-6} \multicolumn{1}{c|}{}
&offset     & at  & int     & \ding{52} & \\
\cline{2-6} \multicolumn{1}{c|}{}
&length     & len & int     & \ding{52} & \\
\cline{2-6} \multicolumn{1}{c|}{}
&name       &     & String  & \ding{52} & \\
\cline{2-6} \multicolumn{1}{c|}{}
&override   & ovr & boolean & & \texttt{false} \\
\hline
\parbox[t]{2mm}{\multirow{9}{*}{\rotatebox[origin=c]{90}{\hyperref[lst:CusDefault]{CusDefault}}}}
&onOverflow & ovf & \hyperref[lst:OverflowAction]{enum} & & \texttt{\$\{defaults.cus.onOverflow:Trunc\}}\\
\cline{2-6}
&onUnderlow & unf & \hyperref[lst:UnderflowAction]{enum} & & \texttt{\$\{defaults.cus.onUnderflow:Pad\}}\\
\cline{2-6}
&padChar    & pad & char    & & \texttt{\$\{defaults.cus.pad:' '\}}\\
\cline{2-6}
&initChar   & ini & char    & & \texttt{\$\{defaults.cus.init:' '\}}\\
\cline{2-6}
&check      & chk & \hyperref[lst:CheckCus]{enum} & & \texttt{\$\{defaults.cus.check:Ascii\}}\\
\cline{2-6}
&align      &     & \hyperref[lst:AlignMode]{enum} & & \texttt{\$\{defaults.cus.align:LFT\}}\\
\cline{2-6}
&normalize  & nrm & \hyperref[lst:NormalizeAbcMode]{enum} & & \texttt{\$\{defaults.cus.normalize:None\}}\\
\cline{2-6}
&checkGetter & get & boolean & & \texttt{\$\{defaults.cus.checkGetter:true\}}\\
\cline{2-6}
&checkSetter & set & boolean & & \texttt{\$\{defaults.cus.checkSetter:true\}}\\
\hline \multicolumn{1}{c|}{}
&regex      &     & String  & & \texttt{null} \\
\cline{2-6}
\end{tabular}
\caption{Attributi impostabili per un campo custom} \label{tab:attr.cus}
\end{table}



\subsubsection{Campo Numerico nullabile} \label{sub:yaml.nux}
%--------1---------2---------3---------4---------5---------6---------7---------8
La definizione yaml del campo numerico riflette la struttura imposta dalla
service interface, vedi~\ref{lst:NuxModel}

\begin{table}[!htb]
\centering
\begin{tabular}{|>{\tt}l|>{\tt}c|>{\tt}c|l|}
\hline
\multicolumn{4}{|c|}{NuxModel --- \texttt{!Nux}}\\
\hline
\multicolumn{1}{|c|}{attributo} & \multicolumn{1}{c|}{alt} 
	& \multicolumn{1}{c|}{tipo} & \multicolumn{1}{c|}{note} \\
\hline
\hline
offset     & at  & int     & obbligatorio \\
\hline
length     & len & int     & obbligatorio \\
\hline
name       &     & String  & obbligatorio \\
\hline
override   & ovr & boolean & default \texttt{false} \\
\hline
onOverflow & ovf & enum    & default \texttt{\$\{defaults.nux.onOverflow:Trunc\}}\\
\hline
onUnderlow & unf & enum    & default \texttt{\$\{defaults.nux.onUnderflow:Pad\}}\\
\hline
access     & acc & enum    & default \texttt{\$\{defaults.nux.access:String\}}\\
\hline
wordWidth  & wid & enum    & default \texttt{\$\{defaults.nux.wirdWidth:Int\}}\\
\hline
normalize  & nrm & enum    & default \texttt{\$\{defaults.nux.normalize:None\}}\\
\hline
initialize & ini & enum    & default \texttt{\$\{defaults.nux.initialize:Space\}}\\
\hline
\end{tabular}
\caption{Attributi impostabili per un campo numerico nullabile} \label{tab:attr.nux}
\end{table}

\ifesource
\begin{figure*}[!htb]
\begin{lstlisting}[language=yaml, 
caption={esempio definizione campi numerici nullabili}, 
label=lst:xmplNux]
defaults:
  nux:
    onOverflow: Trunc   # :ovf: Error, Trunc
    onUnderflow: Pad    # :unf: Error, Pad
    normalize: None     # :nrm: None, Trim
    wordWidth: Int      # :wid: Byte(1,byte), Short(2,short), Int(4,int), Long(8,long)
    access: String      # :acc: String(Str), Numeric(Num), Both(All)
    initialize: Spaces  # :ini: Spaces(Space), Zeroes(Zero)
packages:
  - name: com.example.test
    classes:
      - name: Foo3114
        length: 8
        doc: No
        fields:
          - !Nux { name: year , at: 1, len: 4 }
          - !Nux { name: month, at: 5, len: 2 }
          - !Nux { name: mday , at: 7, len: 2 }
\end{lstlisting}
\end{figure*}
\else
\begin{elisting}
\begin{yamlcode}
defaults:
  nux:
    onOverflow: Trunc   # :ovf: Error, Trunc
    onUnderflow: Pad    # :unf: Error, Pad
    normalize: None     # :nrm: None, Trim
    wordWidth: Int      # :wid: Byte(1,byte), Short(2,short), Int(4,int), Long(8,long)
    access: String      # :acc: String(Str), Numeric(Num), Both(All)
    initialize: Spaces  # :ini: Spaces(Space), Zeroes(Zero)
packages:
  - name: com.example.test
    classes:
      - name: Foo3114
        length: 8
        doc: No
        fields:
          - !Nux { name: year , at: 1, len: 4 }
          - !Nux { name: month, at: 5, len: 2 }
          - !Nux { name: mday , at: 7, len: 2 }
\end{yamlcode}
\caption{esempio definizione campi numerici nullabili}
\label{lst:xmplNux}
\end{elisting}
\fi

\subsubsection{Campo Dominio} \label{sub:yaml.dom}
%--------1---------2---------3---------4---------5---------6---------7---------8
La definizione yaml del campo numerico riflette la struttura imposta dalla
service interface, vedi~\ref{lst:DomModel}

\begin{table}[!htb]
\centering
\begin{tabular}{|>{\tt}l|>{\tt}c|>{\tt}c|l|}
\hline
\multicolumn{4}{|c|}{DomModel --- \texttt{!Dom}}\\
\hline
\multicolumn{1}{|c|}{attributo} & \multicolumn{1}{c|}{alt} 
	& \multicolumn{1}{c|}{tipo} & \multicolumn{1}{c|}{note} \\
\hline
\hline
offset     & at  & int     & obbligatorio \\
\hline
length     & len & int     & obbligatorio \\
\hline
name       &     & String  & obbligatorio \\
\hline
override   & ovr & boolean & default \texttt{false} \\
\hline
items      &     & array  & obbligatorio (valori permessi)\\
\hline
\end{tabular}
\caption{Attributi impostabili per un campo dominio} \label{tab:attr.dom}
\end{table}

\ifesource
\begin{figure*}[!htb]
\begin{lstlisting}[language=yaml, caption={esempio definizione campi dominio}, 
label=lst:xmplDom]
packages:
  - name: com.example.test
    classes:
      - name: Foo3115
        length: 12
        doc: No
        fields:
          - !Num { name: year , at: 1, len: 4 }
          - !Dom { name: month, at: 5, len: 3, 
                   items: [ Jan, Feb, Mar, Apr, May, Jun, Jul, Aug, Sep, Oct, Nov, Dec ] }
          - !Num { name: mday , at: 8, len: 2 }
          - !Dom { name: wday , at: 10, len: 3, 
                   items: [ Sun, Mon, Tue, Wed, Thu, Fri, Sat ] }
\end{lstlisting}
\end{figure*}
\else
\begin{elisting}
\begin{yamlcode}
packages:
  - name: com.example.test
    classes:
      - name: Foo3115
        length: 12
        doc: No
        fields:
          - !Num { name: year , at: 1, len: 4 }
          - !Dom { name: month, at: 5, len: 3, 
                   items: [ Jan, Feb, Mar, Apr, May, Jun, Jul, Aug, Sep, Oct, Nov, Dec ] }
          - !Num { name: mday , at: 8, len: 2 }
          - !Dom { name: wday , at: 10, len: 3, 
                   items: [ Sun, Mon, Tue, Wed, Thu, Fri, Sat ] }
\end{yamlcode}
\caption{esempio definizione campi dominio}
\label{lst:xmplDom}
\end{elisting}
\fi


\subsubsection{Campo Filler} \label{sub:yaml.fil}
%--------1---------2---------3---------4---------5---------6---------7---------8
La definizione yaml del campo numerico riflette la struttura imposta dalla
service interface, vedi~\ref{lst:FilModel}

\begin{table}[!htb]
\centering
\begin{tabular}{|>{\tt}l|>{\tt}c|>{\tt}c|l|}
\hline
\multicolumn{4}{|c|}{FilModel --- \texttt{!Fil}}\\
\hline
\multicolumn{1}{|c|}{attributo} & \multicolumn{1}{c|}{alt} 
	& \multicolumn{1}{c|}{tipo} & \multicolumn{1}{c|}{note} \\
\hline
\hline
offset     & at  & int     & obbligatorio \\
\hline
length     & len & int     & obbligatorio \\
\hline
fill       &     & char    & default \texttt{\$\{defaults.fil.fill:0\}}\\
\hline
\end{tabular}
\caption{Attributi impostabili per un campo filler} \label{tab:attr.fill}
\end{table}

\ifesource
\begin{figure*}[!htb]
\begin{lstlisting}[language=yaml, 
caption={esempio definizione campi filler}, 
label=lst:xmplFil]
defaults:
  fil:
    fill: 0             # \u0000
packages:
  - name: com.example.test
    classes:
      - name: Foo3116
        length: 10
        doc: No
        fields:
          - !Num { at: 1, len: 4, name: year }
          - !Fil { at: 5, len: 1, fill: '-' }
          - !Num { at: 6, len: 2, name: month }
          - !Fil { at: 8, len: 1, fill: '-' }
          - !Num { at: 9, len: 2, name: mday }
\end{lstlisting}
\end{figure*}
\else
\begin{elisting}
\begin{yamlcode}
defaults:
  fil:
    fill: 0             # \u0000
packages:
  - name: com.example.test
    classes:
      - name: Foo3116
        length: 10
        doc: No
        fields:
          - !Num { at: 1, len: 4, name: year }
          - !Fil { at: 5, len: 1, fill: '-' }
          - !Num { at: 6, len: 2, name: month }
          - !Fil { at: 8, len: 1, fill: '-' }
          - !Num { at: 9, len: 2, name: mday }
\end{yamlcode}
\caption{esempio definizione campi filler}
\label{lst:xmplFil}
\end{elisting}
\fi

\subsubsection{Campo Costante} \label{sub:yaml.val}
%--------1---------2---------3---------4---------5---------6---------7---------8
La definizione yaml del campo numerico riflette la struttura imposta dalla
service interface, vedi~\ref{lst:ValModel}

\begin{table}[!htb]
\centering
\begin{tabular}{|>{\tt}l|>{\tt}c|>{\tt}c|l|}
\hline
\multicolumn{4}{|c|}{ValModel --- \texttt{!Val}}\\
\hline
\multicolumn{1}{|c|}{attributo} & \multicolumn{1}{c|}{alt} 
	& \multicolumn{1}{c|}{tipo} & \multicolumn{1}{c|}{note} \\
\hline
\hline
offset     & at  & int     & obbligatorio \\
\hline
length     & len & int     & obbligatorio \\
\hline
value      & val & string  & obbligatorio \\
\hline
\end{tabular}
\caption{Attributi impostabili per un campo costante} \label{tab:attr.val}
\end{table}

\ifesource
\begin{figure*}[!htb]
\begin{lstlisting}[language=yaml, 
caption={esempio definizione campi costanti}, 
label=lst:xmplVal]
packages:
  - name: com.example.test
    classes:
      - name: Foo3117
        length: 10
        fields:
          - !Num { at: 1, len: 4, name: year }
          - !Val { at: 5, len: 1, val: "-" }
          - !Num { at: 6, len: 2, name: month }
          - !Val { at: 8, len: 1, val: "-" }
          - !Num { at: 9, len: 2, name: mday }
\end{lstlisting}
\end{figure*}
\else
\begin{elisting}
\begin{yamlcode}
packages:
  - name: com.example.test
    classes:
      - name: Foo3117
        length: 10
        fields:
          - !Num { at: 1, len: 4, name: year }
          - !Val { at: 5, len: 1, val: "-" }
          - !Num { at: 6, len: 2, name: month }
          - !Val { at: 8, len: 1, val: "-" }
          - !Num { at: 9, len: 2, name: mday }
\end{yamlcode}
\caption{esempio definizione campi costanti}
\label{lst:xmplVal}
\end{elisting}
\fi

\subsubsection{Campo Gruppo} \label{sub:yaml.grp}
%--------1---------2---------3---------4---------5---------6---------7---------8
La definizione yaml del campo numerico riflette la struttura imposta dalla
service interface, vedi~\ref{lst:GrpModel}

\begin{table}[!htb]
\centering
\begin{tabular}{|>{\tt}l|>{\tt}c|>{\tt}c|l|}
\hline
\multicolumn{4}{|c|}{GrpModel --- \texttt{!Grp}}\\
\hline
\multicolumn{1}{|c|}{attributo} & \multicolumn{1}{c|}{alt} 
	& \multicolumn{1}{c|}{tipo} & \multicolumn{1}{c|}{note} \\
\hline
\hline
offset     & at  & int     & obbligatorio \\
\hline
length     & len & int     & obbligatorio \\
\hline
name       &     & String  & obbligatorio \\
\hline
override   & ovr & boolean & default \texttt{false} \\
\hline
fields     &     & array  & obbligatorio (elenco campi)\\
\hline
\end{tabular}
\caption{Attributi impostabili per un gruppo} \label{tab:attr.grp}
\end{table}

\ifesource
\begin{figure*}[!htb]
\begin{lstlisting}[language=yaml, 
caption={esempio definizione gruppo di campi}, 
label=lst:xmplGrp]
packages:
  - name: com.example.test
    classes:
      - name: Foo3118
        length: 12
        fields:
          - !Grp { name: startTime, at: 1, len: 6, fields: [
            !Num { name: hours  , at: 1, len: 2 }, 
            !Num { name: minutes, at: 3, len: 2 }, 
            !Num { name: seconds, at: 5, len: 2 }
            ] }
          - !Grp { name: stopTime , at: 7, len: 6, fields: [
            !Num { name: hours  , at:  7, len: 2 }, 
            !Num { name: minutes, at:  9, len: 2 }, 
            !Num { name: seconds, at: 11, len: 2 }
            ] }
\end{lstlisting}
\end{figure*}
\else
\begin{elisting}
\begin{yamlcode}
packages:
  - name: com.example.test
    classes:
      - name: Foo3118
        length: 12
        fields:
          - !Grp { name: startTime, at: 1, len: 6, fields: [
            !Num { name: hours  , at: 1, len: 2 }, 
            !Num { name: minutes, at: 3, len: 2 }, 
            !Num { name: seconds, at: 5, len: 2 }
            ] }
          - !Grp { name: stopTime , at: 7, len: 6, fields: [
            !Num { name: hours  , at:  7, len: 2 }, 
            !Num { name: minutes, at:  9, len: 2 }, 
            !Num { name: seconds, at: 11, len: 2 }
            ] }
\end{yamlcode}
\caption{esempio definizione gruppo di campi}
\label{lst:xmplGrp}
\end{elisting}
\fi

\subsubsection{Campo Gruppo ripetuto} \label{sub:yaml.occ}
%--------1---------2---------3---------4---------5---------6---------7---------8
La definizione yaml del campo numerico riflette la struttura imposta dalla
service interface, vedi~\ref{lst:OccModel}

\begin{table}[!htb]
\centering
\begin{tabular}{|>{\tt}l|>{\tt}c|>{\tt}c|l|}
\hline
\multicolumn{4}{|c|}{OccModel --- \texttt{!Occ}}\\
\hline
\multicolumn{1}{|c|}{attributo} & \multicolumn{1}{c|}{alt} 
	& \multicolumn{1}{c|}{tipo} & \multicolumn{1}{c|}{note} \\
\hline
\hline
offset     & at  & int     & obbligatorio \\
\hline
length     & len & int     & obbligatorio \\
\hline
name       &     & String  & obbligatorio \\
\hline
override   & ovr & boolean & default \texttt{false} \\
\hline
times      & x   & int     & obbligatorio \\
\hline
fields     &     & array  & obbligatorio (elenco campi)\\
\hline
\end{tabular}
\caption{Attributi impostabili per un gruppo ripetuto} \label{tab:attr.occ}
\end{table}

\ifesource
\begin{figure*}[!htb]
\begin{lstlisting}[language=yaml, 
caption={esempio definizione gruppo di campi ripetuto}, 
label=lst:xmplOcc]
packages:
  - name: com.example.test
    classes:
      - name: Foo3119
        length: 590
        fields:
          - !Num { name: nmErrors, at: 1, len: 2 }
          - !Occ { name: tabError, at: 3, len: 49, x: 12, fields: [
            !Abc { name: status  , at:  3, len:  5 },
            !Num { name: code    , at:  8, len:  4 },
            !Abc { name: message , at: 12, len: 40 }
          ] }
\end{lstlisting}
\end{figure*}
\else
\begin{elisting}
\begin{yamlcode}
packages:
  - name: com.example.test
    classes:
      - name: Foo3119
        length: 590
        fields:
          - !Num { name: nmErrors, at: 1, len: 2 }
          - !Occ { name: tabError, at: 3, len: 49, x: 12, fields: [
            !Abc { name: status  , at:  3, len:  5 },
            !Num { name: code    , at:  8, len:  4 },
            !Abc { name: message , at: 12, len: 40 }
          ] }
\end{yamlcode}
\caption{esempio definizione gruppo di campi ripetuto}
\label{lst:xmplOcc}
\end{elisting}
\fi

\subsubsection{Campo incorporato da interfaccia} \label{sub:yaml.emb}
%--------1---------2---------3---------4---------5---------6---------7---------8
La definizione yaml del campo numerico riflette la struttura imposta dalla
service interface, vedi~\ref{lst:EmbModel}

\begin{table}[!htb]
\centering
\begin{tabular}{|>{\tt}l|>{\tt}c|>{\tt}c|l|}
\hline
\multicolumn{4}{|c|}{EmbModel --- \texttt{!Emb}}\\
\hline
\multicolumn{1}{|c|}{attributo} & \multicolumn{1}{c|}{alt} 
	& \multicolumn{1}{c|}{tipo} & \multicolumn{1}{c|}{note} \\
\hline
\hline
offset     & at  & int     & obbligatorio \\
\hline
length     & len & int     & obbligatorio \\
\hline
source     & src  & Trait  & obbligatorio (interfaccia)\\
\hline
\end{tabular}
\caption{Attributi impostabili per un elenco di campi importato da una interfaccia}
\label{tab:attr.emb}
\end{table}


\subsubsection{Campo Gruppo da interfaccia} \label{sub:yaml.igrp}
%--------1---------2---------3---------4---------5---------6---------7---------8
La definizione yaml del campo numerico riflette la struttura imposta dalla
service interface, vedi~\ref{lst:GrpTraitModel}

\begin{table}[!htb]
\centering
\begin{tabular}{|>{\tt}l|>{\tt}c|>{\tt}c|l|}
\hline
\multicolumn{4}{|c|}{GrpTraitModel --- \texttt{!GRP}}\\
\hline
\multicolumn{1}{|c|}{attributo} & \multicolumn{1}{c|}{alt} 
	& \multicolumn{1}{c|}{tipo} & \multicolumn{1}{c|}{note} \\
\hline
\hline
offset     & at  & int     & obbligatorio \\
\hline
length     & len & int     & obbligatorio \\
\hline
name       &     & String  & obbligatorio \\
\hline
override   & ovr & boolean & default \texttt{false} \\
\hline
typedef    & as  & Trait   & obbligatorio (interfaccia)\\
\hline
\end{tabular}
\caption{Attributi impostabili per un gruppo da interfaccia} \label{tab:attr.igrp}
\end{table}

\ifesource
\begin{figure*}[!htb]
\begin{lstlisting}[language=yaml, 
caption={esempio definizione gruppo di campi da interfaccia}, 
label=lst:xmplIGrp]
packages:
  - name: com.example.test
    interfaces:
      - &Time
        name: ITime
        len: 6
        fields:
          - !Num { name: hours  , at: 1, len: 2 }
          - !Num { name: minutes, at: 3, len: 2 }
          - !Num { name: seconds, at: 5, len: 2 }
    classes:
      - name: Foo311b
        length: 12
        fields:
          - !GRP { name: startTime, at: 1, len: 6, as: *Time }
          - !GRP { name: stopTime , at: 7, len: 6, as: *Time }
\end{lstlisting}
\end{figure*}
\else
\begin{elisting}
\begin{yamlcode}
packages:
  - name: com.example.test
    interfaces:
      - &Time
        name: ITime
        len: 6
        fields:
          - !Num { name: hours  , at: 1, len: 2 }
          - !Num { name: minutes, at: 3, len: 2 }
          - !Num { name: seconds, at: 5, len: 2 }
    classes:
      - name: Foo311b
        length: 12
        fields:
          - !GRP { name: startTime, at: 1, len: 6, as: *Time }
          - !GRP { name: stopTime , at: 7, len: 6, as: *Time }
\end{yamlcode}
\caption{esempio definizione gruppo di campi da interfaccia}
\label{lst:xmplIGrp}
\end{elisting}
\fi

\subsubsection{Campo Gruppo ripetuto da interfaccia} \label{sub:yaml.iocc}
%--------1---------2---------3---------4---------5---------6---------7---------8
La definizione yaml del campo numerico riflette la struttura imposta dalla
service interface, vedi~\ref{lst:OccTraitModel}

\begin{table}[!htb]
\centering
\begin{tabular}{|>{\tt}l|>{\tt}c|>{\tt}c|l|}
\hline
\multicolumn{4}{|c|}{OccTraitModel --- \texttt{!OCC}}\\
\hline
\multicolumn{1}{|c|}{attributo} & \multicolumn{1}{c|}{alt} 
	& \multicolumn{1}{c|}{tipo} & \multicolumn{1}{c|}{note} \\
\hline
\hline
offset     & at  & int     & obbligatorio \\
\hline
length     & len & int     & obbligatorio \\
\hline
name       &     & String  & obbligatorio \\
\hline
override   & ovr & boolean & default \texttt{false} \\
\hline
times      & x   & int     & obbligatorio \\
\hline
typedef    & as  & Trait   & obbligatorio (interfaccia)\\
\hline
\end{tabular}
\caption{Attributi impostabili per un gruppo ripetuto da interfaccia} \label{tab:attr.iocc}
\end{table}

\ifesource
\begin{figure*}[!htb]
\begin{lstlisting}[language=yaml, 
caption={esempio definizione gruppo di campi ripetuto da interfaccia}, 
label=lst:xmplIOcc]
packages:
  - name: com.example.test
    interfaces:
      - &Error
        name: IError
        len: 49
        fields:
          - !Abc { name: status  , at:  1, len:  5 }
          - !Num { name: code    , at:  6, len:  4 }
          - !Abc { name: message , at: 10, len: 40 }
    classes:
      - name: Foo311c
        length: 590
        fields:
          - !Num { name: nmErrors, at: 1, len: 2}
          - !OCC { name: tabError, at: 3, len: 49, x: 12, as: *Error }
\end{lstlisting}
\end{figure*}
\else
\begin{elisting}
\begin{yamlcode}
packages:
  - name: com.example.test
    interfaces:
      - &Error
        name: IError
        len: 49
        fileds:
          - !Abc { name: status  , at:  1, len:  5}
          - !Num { name: code    , at:  6, len:  4}
          - !Abc { name: message , at: 10, len: 40}
    classes:
      - name: Foo311c
        length: 590
        fields:
          - !Num { name: nmErrors, at: 1, len: 2}
          - !Occ { name: tabError, at: 3, len: 49, x: 12, as: *Error }
\end{yamlcode}
\caption{esempio definizione gruppo di campi ripetuto da interfaccia}
\label{lst:xmplIOcc}
\end{elisting}
\fi


\chapter{Service Provider}
%--------1---------2---------3---------4---------5---------6---------7---------8
Nei capitoli precedenti abbiamo visto la \textsl{Service Provider Interface}, 
che  definisce delle interfacce e delle classi che permettono di definire i 
traccciati, e indicare alcuni comportamenti che dovranno essere usati in fase di
utilizzazione dei tracciati; e alcuni esempi di \textsl{Service}, che
semplicemente valorizza gli oggetti messi a disposizione della 
\textsl{Service Provider Interface}, ma il vero lavoro di generazione del codice 
è fatto dal \textsl{Service Provider}.

%--------1---------2---------3---------4---------5---------6---------7---------8
La struttuta SPI consente di avere codice generato diverso, implementato in modo
diverso, o addirittura generare sorgente in un linguaggio diverso.

%--------1---------2---------3---------4---------5---------6---------7---------8
Qualunque sia il linguaggio generato e il dettaglio della implementazione il
\textsl{Service Provider} dovrà fornire alcune funzionalità generali.

\begin{itemize}
%--------1---------2---------3---------4---------5---------6---------7---------8
\item \textbf{decode}: partendo dalla stringa-dati, deve instanziare la 
    classe-dati;
\item \textbf{setter, getter}: la classe-dati generata deve fornire i metodi di 
    accesso ai singoli campi;
\item \textbf{costruttore vuoto}: la classe-dati può essere instanziata con i 
    valori di default dei campi;
\item \textbf{encode}: la classe-dati può essere serializzata nella 
    stringa-dati.
\end{itemize}
%--------1---------2---------3---------4---------5---------6---------7---------8
Sarebbe gradita anche qualche funzionalià accessoria:
\begin{itemize}
\item \textbf{validate}: validare la stringa-dati prima della 
    deserializzaione, in modo da segnalare tutte le aree che non possono essere
    assegnate ai relativi campi, tipicamente caratteri non numerici in campi di
    tipo numerico;
\item \textbf{cast}: se due stringhe-dati hanno la stessa lunghezza, poter 
    passare da una classe-dati che le rappresenta all'\,altra;
\item \textbf{toString}: fornire un metodo che mostra tutti i valori dei campi 
    che compongono la classe-dati.
\item (deep) \textbf{copy}: genera una copia della classe-dati;
\end{itemize}



\section{Generazione sorgente java --- \texttt{java-addon}}
%--------1---------2---------3---------4---------5---------6---------7---------8
Le classi generate dal \textsl{CodeProvider} java oltre ai setter e getter
hanno una serie di metodi ausiliari, vedi cod.~\ref{lst:Foo312:java}.


\ifesource
\begin{figure*}[!htb]
\begin{lstlisting}[language=java, caption=esempio di classe generata (Foo312), 
label=lst:Foo312:java]
public class Foo312 extends FixRecord {
    public static final int LRECL = 50;
    public Foo312() { /* ... */ }
    public static Foo312 of(FixRecord r) { /* ... */ }
    public static Foo312 decode(String s) { /* ... */ }
    public Foo312 copy() { /* ... */ }
    // setter and getter ...
    public boolean validateFails((*\hyperref[lst:FieldValidateHandler:java]{FieldValidateHandler}*) handler) { /* ... */ }
    public String toString() { /* ... */ }
    public String encode() { /* ... */ }    // from super class
}
\end{lstlisting}
\end{figure*}
\else
\begin{elisting}
\begin{javacode}
public class Foo312 extends FixRecord {
    public static final int LRECL = 50;
    public Foo312() { /* ... */ }
    public static Foo312 of(FixRecord r) { /* ... */ }
    public static Foo312 decode(String s) { /* ... */ }
    public Foo312 copy() { /* ... */ }
    // setter and getter ...
    public boolean validateFails(|\hyperref[lst:FieldValidateHandler:java]{FieldValidateHandler}| handler) { /* ... */ }
    public String toString() { /* ... */ }
    public String encode() { /* ... */ }    // from super class
}
\end{javacode}
\caption{esempio di classe generata (Foo312)}
\label{lst:Foo312:java}
\end{elisting}
\fi

%--------1---------2---------3---------4---------5---------6---------7---------8
\begin{itemize}
\item viene messa a disposizione una costante con la lunghezza della 
    stringa-dati attesa
\item viene fornito un costruttore senza argomenti, che crea la classe con i 
    valori a default
\item viene fornito un costruttore \textit{cast-like}, che prende come argomento
    qualunque altra classe che rappresenta una classe-dati
\item viene fornito un costruttore da stringa-dati (de-serializzatore)
\item viene fornito un metodo \textit{deep-copy} per duplicare la classe-dati
\item viene fornito un metodo di validazione
\item viene fornito un metodo di \texttt{toString}
\item viene fornito un metodo per generare la stringa-dati (serializzatore)
\end{itemize}


\ifesource
\begin{figure*}[!htb]
\begin{lstlisting}[language=java, 
caption={gestore errori \texttt{FieldValidateHandler}}, 
label=lst:FieldValidateHandler:java]
public interface FieldValidateHandler {
    void error((|hyperref[lst:FieldValidateError:java]{FieldValidateError}| fieldValidateError);
}
\end{lstlisting}
\end{figure*}
\else
\begin{elisting}
\begin{javacode}
public interface FieldValidateHandler {
    void error(|\hyperref[lst:FieldValidateError:java]{FieldValidateError}| fieldValidateError);
}
\end{javacode}
\caption{gestore errori \texttt{FieldValidateHandler}}
\label{lst:FieldValidateHandler:java}
\end{elisting}
\fi

\ifesource
\begin{figure*}[!htb]
\begin{lstlisting}[language=java, 
caption={dettaglio errore \texttt{FieldValidateError}}, 
label=lst:FieldValidateError:java]
public interface FieldValidateError {
    String name();          // field name in error
    int offset();           // field offset in error
    int length();           // field length in error
    String value();         // field value in error
    Integer column();       // column of the record with the wrong character
    (*\hyperref[lst:ValidateError:java]{ValidateError}*) code();   // error category
    Character wrong();      // wrong character
    String message();       // field message error
}
\end{lstlisting}
\end{figure*}
\else
\begin{elisting}
\begin{javacode}
public interface FieldValidateError {
    String name();          // field name in error
    int offset();           // field offset in error
    int length();           // field length in error
    String value();         // field value in error
    Integer column();       // column of the record with the wrong character
    |\hyperref[lst:ValidateError:java]{ValidateError}| code();   // error category
    Character wrong();      // wrong character
    String message();       // field message error
}
\end{javacode}
\caption{dettaglio errore \texttt{FieldValidateError}}
\label{lst:FieldValidateError:java}
\end{elisting}
\fi

\ifesource
\begin{figure*}[!htb]
\begin{lstlisting}[language=java, 
caption={categoria errore \texttt{ValidateError}}, 
label=lst:ValidateError:java]
public enum ValidateError {
    NotNumber, NotAscii, NotLatin, NotValid, NotDomain, NotBlank, NotEqual, NotMatch, NotDigitBlank
}
\end{lstlisting}
\end{figure*}
\else
\begin{elisting}
\begin{javacode}
public enum ValidateError {
    NotNumber, NotAscii, NotLatin, NotValid, NotDomain, NotBlank, NotEqual, NotMatch, NotDigitBlank
}
\end{javacode}
\caption{categoria errore \texttt{ValidateError}}
\label{lst:ValidateError:java}
\end{elisting}
\fi



%\appendix
%\input{appe01.tex}
\printindex

\end{document}
