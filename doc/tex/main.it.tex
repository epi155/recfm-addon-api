\documentclass[a4paper,10pt]{report}

\usepackage[italian]{babel}     % sillabazione italiana
\usepackage[body={17.8cm,24.7cm}]{geometry}

\usepackage[utf8x]{inputenc}
\usepackage[T1]{fontenc}
\usepackage{pifont}
\usepackage{array}

\usepackage{makeidx}
\usepackage{tocloft}

\usepackage{esourcecode}	% custom style
\usepackage{eflowchart}	% custom style

%\usepackage[pdftex]{graphicx}

\usepackage[pdftex,
%pdftitle={Graphics and Color with LaTeX},
%pdfauthor={Patrick W Daly},
%pdfsubject={Importing images and use of color in LaTeX},
%pdfkeywords={LaTeX, graphics, color},
pdfpagemode=UseOutlines,
bookmarks,bookmarksopen,
pdfstartview=FitH,
colorlinks,linkcolor=blue,citecolor=blue,
urlcolor=green,
]
{hyperref}

\renewcommand{\arraystretch}{1.2}

\usepackage{tikz}
%\usetikzlibrary{shapes,arrows}
%\usetikzlibrary{arrows,shapes,snakes,automata,backgrounds,petri}
\usetikzlibrary{arrows,shapes,decorations,automata,backgrounds,petri}

\usepackage{bera}

\title{RECFM}
\author{E.~Pistolesi}

\makeindex
\setcounter{secnumdepth}{3}
\setcounter{tocdepth}{3}	% default is 2

\cftsetindents{section}{0.5in}{0.5in}
\cftsetindents{subsection}{0.5in}{0.6in}
\cftsetindents{subsubsection}{0.5in}{0.7in}
%\cftsetindents{paragraph}{0.5in}{0.5in}
\cftsetindents{table}{0.25in}{0.5in}
\cftsetindents{listings}{0.25in}{0.5in}

\renewcommand{\lstlistlistingname}{Elenco dei sorgenti}


%--- begin - document ---------------------------------------------------------
\begin{document}

\maketitle

\begin{abstract}
...
\end{abstract}


%\input{cover.tex}

%\clearpage

\tableofcontents
\listoffigures
\listoftables
\lstlistoflistings

\clearpage

\chapter{Introduzione}
%--------1---------2---------3---------4---------5---------6---------7---------8
Spesso può capitare di avere a che fare con file (o aree di memoria)
posizionali, in questi casi è necessario perdere un sacco di tempo per fare una
classe dedicata a ogni tracciato con i setter e getter per leggere e scrivere
i valori. 
Questo gruppo di programmi si propone di minimizzare il tempo per creare queste
classi.


\begin{figure}[!htb]
\centering
\begin{tikzpicture}[>=latex,font={\sf}]
\node(u1) at (0,1.5) [manual input,text width=2cm,fill=blue!10]{maven plugin};
\node(u2) at (3,1.5) [manual input,text width=2cm,fill=blue!10]{gradle plugin};
\node(u3) at (6,1.5) [manual input,text width=2cm,fill=blue!10]{custom client};
\node(si) at (3,0.0) [preparation,fill=yellow!20]{addon-api};
\node(a1) at (0,-1.5) [process,text width=1.7cm,fill=green!10]{java addon};
\node(a2) at (3,-1.5) [process,text width=1.7cm,fill=green!10]{scala addon};
\node(a3) at (6,-1.5) [process,text width=1.7cm,fill=green!10]{custom provider};

\node at (9,1.5) {Service Consumer};
\node at (9,0.0) {Service Interface};
\node at (9,-1.5) {Service Provider};

\draw[arrow] (u1) -- (si.north);
\draw[arrow] (u2) -- (si.north);
\draw[arrow] (u3) -- (si.north);
\draw[arrow] (a1) -- (si.south);
\draw[arrow] (a2) -- (si.south);
\draw[arrow] (a3) -- (si.south);

\end{tikzpicture}
\caption{Struttura service-user, service-interface, service-provider} 
\label{fig:spi}
\end{figure}

%--------1---------2---------3---------4---------5---------6---------7---------8
Il programma è strutturato usando service provider interface, 
vedi fig.~\ref{fig:spi}, abbiamo un plugin, o un programma utente 
(\textsl{Service Consumer}), che vede direttamente le classi definite nella 
\textsl{Service Interface} e recupera la implementazione usando il 
\textsl{ServiceLoader}, in questo modo non ha una dipendenza specifica con una
delle implementazioni usate. 
Il \textsl{Service Provider} deve implementare le classi definite nella 
\textsl{Service Interface}.

%--------1---------2---------3---------4---------5---------6---------7---------8
Se il \verb!maven-plugin! trova in esecuzione la libreria con 
l'\,implementazione \verb!java-addon! genererà i sorgenti in java, ma se trova
l'\,implementazione \verb!scala-addon! genererà i sorgenti in scala.

\chapter{Service Interface}
%--------1---------2---------3---------4---------5---------6---------7---------8
L'\,artifatto \verb!recfm-addon-api! mette a disposizione una serie di 
interfacce, alcuni enum e java-bean per permettere al modulo client di definire
i tracciati. 
Il codice è compilato in moda da essere compatibile con il java 5, ma fornisce
il \verb!module-info! per essere utilizzabile propriamente anche con il java 9
e superiori.

\section{CodeProvider}\index{CodeProvider}
%--------1---------2---------3---------4---------5---------6---------7---------8
Il punto di partenza è l'\,interfaccia \textsl{CodeProvider}, vedi 
cod.~\ref{code:CodeProvider}, questa interfaccia fornisce l'\,istanza 
dell'\,interfaccia \textsl{CodeFactory}

\begin{figure*}[!htb]
\begin{lstlisting}[language=java, caption=interfaccia CodeProvider, 
label=code:CodeProvider]
public interface CodeProvider {
     CodeFactory getInstance();
}
\end{lstlisting}
\end{figure*}
%--------1---------2---------3---------4---------5---------6---------7---------8


\section{CodeFactory} \index{CodeFactory}
%--------1---------2---------3---------4---------5---------6---------7---------8
L\,interfaccia \textsl{CodeFactory}, vedi cod.~\ref{code:CodeFactory}, fornisce i 
metodi per definire tutti gli elementi della struttura.

\begin{figure*}[!htb]
\begin{lstlisting}[language=java, caption=interfaccia CodeFactory, 
label=code:CodeFactory]
public interface CodeFactory {
    ClassModel newClassModel();
    TraitModel newTraitModel();

    AbcModel newAbcModel();
    NumModel newNumModel();
    NuxModel newNuxModel();
    CusModel newCusModel();
    DomModel newDomModel();
    FilModel newFilModel();
    ValModel newValModel();
    GrpModel newGrpModel();
    OccModel newOccModel();
    EmbModel newEmbModel();
    GrpTraitModel newGrpTraitModel();
    OccTraitModel newOccTraitModel();
}
\end{lstlisting}
\end{figure*}
%--------1---------2---------3---------4---------5---------6---------7---------8

\section{Classi / Interfacce}
%--------1---------2---------3---------4---------5---------6---------7---------8
Il primo metodo dell\,interfaccia \textsl{CodeFactory} fornisce la definizione 
per una classe

\index{ClassModel}
\begin{lstlisting}[language=java, caption=interfaccia ClassModel, 
label=code:ClassModel]
public interface ClassModel {
    void setName(String name);
    void setLength(int length);
    void setOnOverflow(LoadOverflowAction onOverflow);
    void setOnUnderflow(LoadUnderflowAction onUnderflow);
    void setFields(List<FieldModel> fields);

    void create(String namespace, GenerateArgs ga, FieldDefault defaults);
}
\end{lstlisting}

%--------1---------2---------3---------4---------5---------6---------7---------8
\noindent e il secondo metodo dell\,interfaccia \textsl{CodeFactory} fornisce la 
definizione per una interfaccia

\index{TraitModel}
\begin{lstlisting}[language=java, caption=interfaccia TraitModel, 
label=code:TraitModel]
public interface TraitModel {
    void setName(String name);
    void setLength(int length);
    void setFields(List<FieldModel> fields);

    void create(String namespace, GenerateArgs ga, FieldDefault defaults);
}
\end{lstlisting}
%--------1---------2---------3---------4---------5---------6---------7---------8
\noindent entrambe le definizioni richiedono il nome della struttura, la sua
lunghezza, l'\,elenco dei campi che la compongono e mettono a disposizione
un metodo per generare il codice sorgente.
%--------1---------2---------3---------4---------5---------6---------7---------8
La definizione della classe richiede anche due parametri aggiuntivi per indicare
come comportarsi se la dimensione dei dati forniti fosse superiore o inferiore a
quella attesa.

%--------1---------2---------3---------4---------5---------6---------7---------8
Prima di vedere il dettaglio della definizione dei vari campi, vediamo il 
contenuto degli altri due parametri richiesti per la generazione del codice
sorgente.

\subsection{Argomenti globali --- GenerateArgs}
%--------1---------2---------3---------4---------5---------6---------7---------8
Vediamo in cosa consiste la classe \textsl{GenerateArgs}

\index{GenerateArgs}
\begin{lstlisting}[language=java, caption=interfaccia GenerateArgs, 
label=code:GenerateArgs]
@Builder
@RequiredArgsConstructor
public class GenerateArgs {
    @NonNull public final File sourceDirectory;
    public final String group;
    public final String artifact;
    public final String version;
    @Builder.Default
    public final boolean doc = true;
}
\end{lstlisting}
%--------1---------2---------3---------4---------5---------6---------7---------8
il primo parametro la directory sorgente radice dove generare i sorgenti,
i tre parametri successivi identificano il programma (o plugin) che ha fornito
la definizione del tracciato, questi parametri sono mostrati come commento
all'\,inizio dei file generati, e l'\,ultimo indica se generare o meno un minimo
di documentazione javadoc sui setter, getter e definizioni delle classi e 
interfacce.

\subsection{Default dei campi --- FieldDefault}
%--------1---------2---------3---------4---------5---------6---------7---------8
Vediamo in cosa consiste la classe \textsl{FieldDefault}

\index{FieldDefault}
\begin{lstlisting}[language=java, caption=class FieldDefault, 
label=code:FieldDefault]
@Data
public class FieldDefault {
    private ClsDefault cls = new ClsDefault();
    private AbcDefault abc = new AbcDefault();
    private NumDefault num = new NumDefault();
    private NuxDefault nux = new NuxDefault();
    private FilDefault fil = new FilDefault();
    private CusDefault cus = new CusDefault();
}
\end{lstlisting}
%--------1---------2---------3---------4---------5---------6---------7---------8
Il primo default riguarda il comportamento di default della classe quando viene 
creata partendo da una struttura (stringa), e questa ha una dimensione diversa 
da quella attesa

\begin{lstlisting}[language=java, caption=classe ClsDefault, 
label=code:ClsDefault]
@Data
public static class ClsDefault {
    private LoadOverflowAction onOverflow = LoadOverflowAction.Trunc;   // enum {Error, Trunc}
    private LoadUnderflowAction onUnderflow = LoadUnderflowAction.Pad;  // enum {Error, Pad}
}
\end{lstlisting}
%--------1---------2---------3---------4---------5---------6---------7---------8
nel caso che la lunghezza della struttura fornita sia maggiore di quella attesa
è possibile lanciare una eccezione e ignorare il contenuto in eccesso,
nel caso che la lunghezza della struttura fornita sia minore di quella atteso è
possibile lanciare una eccezione o completare la parte mancante con i valori di
default della parte mancante.

%--------1---------2---------3---------4---------5---------6---------7---------8
Gli altri default permettono di impostare i valori di default di alcuni 
parametri per cinque tipologie di campi. Non avendo mostrato il dettaglio delle 
definizione delle varie tipologie di campo, non è opportuno introdurre in 
questo punto il contenuto delle classi dei default, saranno mostrate insieme al 
campo corrispondente.

\section{Definizione di campi}
%--------1---------2---------3---------4---------5---------6---------7---------8
Nella definizione della classe, e dell'\,interfaccia l'\,elenco dei campi è
impostato come \verb!List<FieldModel>!, ma l'\,interfaccia \verb!FieldModel! è
una scatola vuota, serve solo per collegare a essa tutte le definizioni dei
campi. In generale tutti i campi hanno una posizione iniziale (offset) e una
dimensione (length); molti campi sono referenziabili tramite un nome, 
ma non tutti hanno necessariamente il nome; quando i campi hanno un nome
possono essere primari o sovra-definire (override) campi primari, in fase di 
inizializzazione dei campi di una classe vengono considerati solo le definizioni
dei campi primari.

\subsection{Campo Alfanumerico}
%--------1---------2---------3---------4---------5---------6---------7---------8

\index{AbcModel}
\begin{lstlisting}[language=java, caption=interfaccia AbcModel, 
label=code:AbcModel]
public interface AbcModel extends FieldModel {
    void setOffset(int offset);
    void setLength(int length);
    void setName(String name);

    void setOverride(boolean override);

    void setOnOverflow(OverflowAction onOverflow);
    void setOnUnderflow(UnderflowAction onUnderflow);
    void setCheck(CheckChar check);
    void setNormalize(NormalizeAbcMode normalize);
    void setCheckGetter(Boolean checkGetter);
    void setCheckSetter(Boolean checkSetter);
}
\end{lstlisting}


\index{AbcDefault}
\begin{lstlisting}[language=java, caption=class AbcDefault, 
label=code:AbcDefault]
@Data
public class AbcDefault {
    private OverflowAction onOverflow = OverflowAction.Trunc;
    private UnderflowAction onUnderflow = UnderflowAction.Pad;
    private CheckChar check = CheckChar.Ascii;
    private NormalizeAbcMode normalize = NormalizeAbcMode.None;
    private boolean checkGetter = true;
    private boolean checkSetter = true;
}
\end{lstlisting}


\subsection{Campo Numerico}
%--------1---------2---------3---------4---------5---------6---------7---------8

\index{NumModel}
\begin{lstlisting}[language=java, caption=interfaccia NumModel, 
label=code:NumModel]
public interface NumModel extends FieldModel {
    void setOffset(int offset);
    void setLength(int length);
    void setName(String name);

    void setOverride(boolean override);

    void setOnOverflow(OverflowAction onOverflow);
    void setOnUnderflow(UnderflowAction onUnderflow);
    void setNormalize(NormalizeNumMode normalize);
    void setAccess(AccesMode mode);
    void setWordWidth(WordWidth width);
}
\end{lstlisting}

\index{NumDefault}
\begin{lstlisting}[language=java, caption=class NumDefault, 
label=code:NumDefault]
@Data
public class NumDefault {
    private OverflowAction onOverflow = OverflowAction.Trunc;
    private UnderflowAction onUnderflow = UnderflowAction.Pad;
    private NormalizeNumMode normalize = NormalizeNumMode.None;
    private AccesMode access = AccesMode.String;
    private WordWidth wordWidth = WordWidth.W4;
}
\end{lstlisting}


\subsection{Campo Custom (alfanumerico)}
%--------1---------2---------3---------4---------5---------6---------7---------8

\index{CusModel}
\begin{lstlisting}[language=java, caption=interfaccia CusModel, 
label=code:CusModel]
public interface CusModel extends FieldModel {
    void setOffset(int offset);
    void setLength(int length);
    void setName(String name);
    
    void setOverride(boolean override);
    
    void setOnOverflow(OverflowAction onOverflow);
    void setOnUnderflow(UnderflowAction onUnderflow);
    void setPadChar(Character padChar);
    void setInitChar(Character character);
    void setCheck(CheckUser check);
    void setAlign(AlignMode align);
    void setNormalize(NormalizeAbcMode normalize);
    void setRegex(String regex);
    void setCheckGetter(Boolean checkGetter);
    void setCheckSetter(Boolean checkSetter);
}
\end{lstlisting}

\index{CusDefault}
\begin{lstlisting}[language=java, caption=class CusDefault, 
label=code:CusDefault]
@Data
public class CusDefault {
    private OverflowAction onOverflow = OverflowAction.Trunc;
    private UnderflowAction onUnderflow = UnderflowAction.Pad;
    private char pad = ' ';
    private char init = ' ';
    private CheckUser check = CheckUser.Ascii;
    private AlignMode align = AlignMode.LFT;
    private NormalizeAbcMode normalize = NormalizeAbcMode.None;
    private boolean checkGetter = true;
    private boolean checkSetter = true;
}
\end{lstlisting}

\subsection{Campo Numerico nullabile}
%--------1---------2---------3---------4---------5---------6---------7---------8

\index{NuxModel}
\begin{lstlisting}[language=java, caption=interfaccia NuxModel, 
label=code:NuxModel]
public interface NuxModel extends FieldModel {
    void setOffset(int offset);
    void setLength(int length);
    void setName(String name);
    
    void setOverride(boolean override);
    
    void setOnOverflow(OverflowAction onOverflow);
    void setOnUnderflow(UnderflowAction onUnderflow);
    void setAccess(AccesMode mode);
    void setWordWidth(WordWidth width);
    void setNormalize(NormalizeNumMode normalize);
    void setInitialize(InitializeNuxMode initialize);
}
\end{lstlisting}

\index{NuxDefault}
\begin{lstlisting}[language=java, caption=class NuxDefault, 
label=code:NuxDefault]
@Data
public class NuxDefault {
    private OverflowAction onOverflow = OverflowAction.Trunc;
    private UnderflowAction onUnderflow = UnderflowAction.Pad;
    private NormalizeNumMode normalize = NormalizeNumMode.None;
    private InitializeNuxMode initialize = InitializeNuxMode.Spaces;
    private WordWidth wordWidth = WordWidth.W4;
    private AccesMode access = AccesMode.String;
}
\end{lstlisting}

\subsection{Campo Dominio}
%--------1---------2---------3---------4---------5---------6---------7---------8

\index{DomModel}
\begin{lstlisting}[language=java, caption=interfaccia DomModel, 
label=code:DomModel]
public interface DomModel extends FieldModel {
    void setOffset(int offset);
    void setLength(int length);
    void setName(String name);
    
    void setOverride(boolean override);
    
    void setItems(String[] items);
}
\end{lstlisting}


\subsection{Campo Filler}
%--------1---------2---------3---------4---------5---------6---------7---------8

\index{FilModel}
\begin{lstlisting}[language=java, caption=interfaccia FilModel, 
label=code:FilModel]
public interface FilModel extends FieldModel {
    void setOffset(int offset);
    void setLength(int length);
    
    void setFill(Character fill);
}
\end{lstlisting}

\index{FilDefault}
\begin{lstlisting}[language=java, caption=class FilDefault, 
label=code:FilDefault]
@Data
public class FilDefault {
    private char fill = 0;
    private CheckChar check = CheckChar.None;
}
\end{lstlisting}

\subsection{Campo Valore costante}
%--------1---------2---------3---------4---------5---------6---------7---------8

\index{ValModel}
\begin{lstlisting}[language=java, caption=interfaccia ValModel, 
label=code:ValModel]
public interface ValModel extends FieldModel {
    void setOffset(int offset);
    void setLength(int length);
    
    void setValue(String value);
}
\end{lstlisting}

\subsection{Campo Gruppo di campi}
%--------1---------2---------3---------4---------5---------6---------7---------8

\index{GrpModel}
\begin{lstlisting}[language=java, caption=interfaccia GrpModel, 
label=code:GrpModel]
public interface GrpModel extends FieldModel {
    void setOffset(int offset);
    void setLength(int length);
    void setName(String name);
    
    void setOverride(boolean override);
    
    void setFields(List<FieldModel> fields);
}
\end{lstlisting}

\subsection{Campo Gruppo di campi ripetuto}
%--------1---------2---------3---------4---------5---------6---------7---------8

\index{OccModel}
\begin{lstlisting}[language=java, caption=interfaccia OccModel, 
label=code:OccModel]
public interface OccModel extends  FieldModel {
    void setOffset(int offset);
    void setLength(int length);
    void setName(String name);
    
    void setOverride(boolean override);
    
    void setFields(List<FieldModel> fields);
    void setTimes(int times);
}
\end{lstlisting}

\subsection{Campi incorporati mediante interfaccia}
%--------1---------2---------3---------4---------5---------6---------7---------8

\index{EmbModel}
\begin{lstlisting}[language=java, caption=interfaccia EmbModel, 
label=code:EmbModel]
public interface EmbModel extends FieldModel {
    void setOffset(int offset);
    void setLength(int length);
    
    void setSource(TraitModel source);
}
\end{lstlisting}

\subsection{Campo Gruppo di campi definito mediante interfaccia}
%--------1---------2---------3---------4---------5---------6---------7---------8

\index{GrpTraitModel}
\begin{lstlisting}[language=java, caption=interfaccia GrpTraitModel, 
label=code:GrpTraitModel]
public interface GrpTraitModel extends FieldModel {
    void setOffset(int offset);
    void setLength(int length);
    void setName(String name);
    
    void setOverride(boolean override);
    
    void setTypedef(TraitModel typedef);
}
\end{lstlisting}

\subsection{Campo Gruppo di campi ripetuto definito mediante interfaccia}
%--------1---------2---------3---------4---------5---------6---------7---------8

\index{OccTraitModel}
\begin{lstlisting}[language=java, caption=interfaccia OccTraitModel, 
label=code:OccTraitModel]
public interface OccTraitModel extends FieldModel {
    void setOffset(int offset);
    void setLength(int length);
    void setName(String name);
    
    void setOverride(boolean override);
    
    void setTypedef(TraitModel typedef);
    void setTimes(int times);
}
\end{lstlisting}


\chapter{Service Consumer}
%--------1---------2---------3---------4---------5---------6---------7---------8
L'\,interfaccia fissa semplicemente quello che può fare.
Il client cercherà nel classpath un provider che implementi l'\,interfaccia
\textsl{CodeProvider} e col meccanismo del ServiceLoader ne recupera una istanza.

\begin{lstlisting}[language=java, caption=recupero del CodeProvider, 
label=code:getCodeProvider]
        ServiceLoader<CodeProvider> loader = ServiceLoader.load(CodeProvider.class);
        CodeProvider codeProvider = loader.iterator().next();
\end{lstlisting}

%--------1---------2---------3---------4---------5---------6---------7---------8
Da questo recupera il \textsl{CodeFactory} e con questo può creare e valorizzare
la definizione delle strutture.

%--------1---------2---------3---------4---------5---------6---------7---------8
Sono stati sviluppati due client, uno sotto forma di maven plugin 
\verb!recfm-maven-plugin!, e l'\,altro sotto forma di gradle plugin 
\verb!recfm-gradle-plugin!. Il codice in gran parte è identico, cambia solo il 
meccanismo di innesco.

\section{Maven plugin}
%--------1---------2---------3---------4---------5---------6---------7---------8
Il maven plugin \verb!recfm-maven-plugin! permette di creare più classi e 
interfacce usando uno più file di configurazione in formato yaml.
Le librerie esterne utilizzate richiedono il java 8, quindi per eseguire questo 
plugin è necessario almeno il java 8.

Il plugin si aspetta come parametri di configurazione
\begin{lstlisting}[language=java, caption=parametri impostabili del maven plugin, 
label=code:spring-conf]
    @Parameter(defaultValue = "${project.build.directory}/generated-sources/recfm",
        property = "maven.recfm.generateDirectory", required = true)
    private File generateDirectory;

    @Parameter(defaultValue = "true", property = "maven.recfm.doc", required = true)
    private boolean doc;
    
    @Parameter(defaultValue = "${project.build.resources[0].directory}",
        property = "maven.recfm.settingsDirectory", required = true)
    private String settingsDirectory;
    
    @Parameter(required = true)
    private String[] settings;

    @Parameter(defaultValue = "true", property = "maven.recfm.addCompileSourceRoot")
    private boolean addCompileSourceRoot = true;

    @Parameter(defaultValue = "false", property = "maven.recfm.addTestCompileSourceRoot")
    private boolean addTestCompileSourceRoot = false;
    
    @Parameter
    private String codeProviderClassName;
\end{lstlisting}

\begin{itemize}
%--------1---------2---------3---------4---------5---------6---------7---------8
\item \index{plugin!generateDirectory}
	Il campo \fcolorbox{black}{yellow!75}{\texttt{generateDirectory}} indica la 
	directory root da utilizzare per la generazione dei sorgenti, viene usato 
	per valorizzare il campo \verb!sourceDirectory! della classe 
	\verb!GenerateArgs!, come si vede dalla definizione, se il parametro è 
	omesso viene utilizzata la directory \verb!target/generated-sources/recfm!, 
	normalmente può essere lasciato il valore di default.
%--------1---------2---------3---------4---------5---------6---------7---------8
\item \index{plugin!doc}
	Il campo \fcolorbox{black}{yellow!75}{\texttt{doc}} indica se generare o no 
	la documentazione javadoc nei sorgenti, viene usato per valorizzare il campo 
	\verb!doc! della classe \verb!GenerateArgs!, come si vede dalla definizione, 
	se il parametro è omesso viene usato il valore \verb!true!, normalmente può 
	essere lasciato il valore di default.
%--------1---------2---------3---------4---------5---------6---------7---------8
\item \index{plugin!settingsDirectory}
	Il campo \fcolorbox{black}{yellow!75}{\texttt{settingsDirectory}} indica la 
	directory che contiene i file	di configurazione, se il parametro è omesso 
	viene usato il valore \verb!src/main/resources!, normalmente può essere 
	lasciato il valore di default.
%--------1---------2---------3---------4---------5---------6---------7---------8
\item \index{plugin!settings}
	Il campo \fcolorbox{black}{yellow!75}{\texttt{settings}} indica l'elenco dei 
	file di configurazione da utilizzare per generare le classi/interfacce; il 
	parametro deve essere fornito al plugin.
%--------1---------2---------3---------4---------5---------6---------7---------8
\item \index{plugin!addCompileSourceRoot}
	Il campo \fcolorbox{black}{yellow!75}{\texttt{addCompileSourceRoot}} è un 
	campo tecnico, indica a maven che la directory dove sono stati generati i 
	sorgenti deve essere inclusa tra quelle utilizzate per la compilazione 
	principale, se il parametro è omesso viene utilizzato il valore 
	\verb!true!; il valore \verb!true! è opportuno quando viene usata una 
	directory di generazione del codice diversa da \verb!src/main/java!, 
	altrimenti è necessario usare plugin aggiuntivi per aggiungere il nuovo path 
	a quello di compilazione di maven.
%--------1---------2---------3---------4---------5---------6---------7---------8
\item \index{plugin!addTestCompileSourceRoot}
	Il campo \fcolorbox{black}{yellow!75}{\texttt{addTestCompileSourceRoot}} è 
	analogo al campo precedente, ma aggiunge la directory di generazione al path 
	di compilazione dei test, se omesso viene utilizzato il valore \verb!false!, 
	tranne in casi particolari può essere lasciato il valore di default.
%--------1---------2---------3---------4---------5---------6---------7---------8
\item \index{plugin!codeProviderClassName}
	Il campo \fcolorbox{black}{yellow!75}{\texttt{codeProviderClassName}} indica 
	quale è la classe concreata che implementa la \textsl{Service Interface}, se 
	omesso viene utilizzata la ``prima'' implementazione recuperata del 
	\textsl{ServiceLoader}; se in classpath è presente una sola implementazione, 
	non è necessario valorizzare il parametro. 
\end{itemize}
%--------1---------2---------3---------4---------5---------6---------7---------8
Nella classe \verb!GenerateArgs! (vedi \ref{code:GenerateArgs}) sono presenti 
altri tre parametri, questi vengono recuperati automaticamente a runtime dal 
plugin.

%--------1---------2---------3---------4---------5---------6---------7---------8
Vediamo un esempio di esecuzione del plugin:

%--------1---------2---------3---------4---------5---------6---------7---------8
\begin{lstlisting}[language=XML, caption=esempio minimale di esecuzione del 
plugin, label=code:spring-conf]
<plugin>
    <groupId>io.github.epi155</groupId>
    <artifactId>recfm-maven-plugin</artifactId>
    <version>0.7.0</version>
    <executions>
        <execution>
            <goals>
                <goal>generate</goal>
            </goals>
            <configuration>
                <settings>
                    <setting>foo.yaml</setting>
                    <setting>bar.yaml</setting>
                </settings>
            </configuration>
        </execution>
    </executions>
    <dependencies>
        <dependency>
            <groupId>io.github.epi155</groupId>
            <artifactId>recfm-java-addon</artifactId>
            <version>0.7.0</version>
        </dependency>
    </dependencies>
</plugin>
\end{lstlisting}
%--------1---------2---------3---------4---------5---------6---------7---------8
il plugin per essere eseguito deve avere come dipendenza una libreria che 
fornisca l'\,implementazione dell'\,inter\-fac\-cia, altrimenti il 
\verb!ServiceLoader! non trova nulla ed il plugin termina in errore.

%--------1---------2---------3---------4---------5---------6---------7---------8
Tutti gli altri parametri sono forniti nei file di configurazione.

\subsection{Struttura del file di configurazione}
%--------1---------2---------3---------4---------5---------6---------7---------8
Per gestire la configurazione dei tracciati il plugin definisce la classe
\textsl{MasterBook}

\begin{lstlisting}[language=java, caption=classe di configurazione MasterBook, 
label=code:MasterBook]
@Data
public class MasterBook {
    private FieldDefault defaults = new FieldDefault();
    private List<ClassPackage> packages = new ArrayList<>();
}
\end{lstlisting}

%--------1---------2---------3---------4---------5---------6---------7---------8
L'\,oggetto \textsl{FieldDefault} (vedi~\ref{code:FieldDefault}) è quello messo
a disposizione dalla \textsl{Service Interface}, espandendo tutte le componenti
in formato \textit{yaml} abbiamo:

\begin{lstlisting}[language=yaml, caption={configurazione, area default campi}, 
label=code:defaults-conf]
defaults:
  cls:
    onOverflow: Trunc   # Error, Trunc
    onUnderflow: Pad    # Error, Pad
  abc:
    check: Ascii        # None, Ascii, Latin1, Valid
    onOverflow: Trunc   # Error, Trunc
    onUnderflow: Pad    # Error, Pad
    normalize: None     # None, Trim, Trim1
    checkGetter: true
    checkSetter: true
  num:
    onOverflow: Trunc   # Error, Trunc
    onUnderflow: Pad    # Error, Pad
    normalize: None     # None, Trim
    wordWidth: W4       # W1(1,8-bit), W2(2,16-bit), W4(4,32-bit), W8(8,64-bit)
    access: String      # String(Str), Numeric(Num), Both(All)
  cus:
    pad: ' '
    init: ' '
    check: Ascii        # None, Ascii, Latin1, Valid, Digit, DigitOrBlank
    align: LFT          # LFT, RGT
    onOverflow: Trunc   # Error, Trunc
    onUnderflow: Pad    # Error, Pad
    normalize: None     # None, Trim, Trim1
    checkGetter: true
    checkSetter: true
  nux:
    onOverflow: Trunc   # Error, Trunc
    onUnderflow: Pad    # Error, Pad
    normalize: None     # None, Trim
    wordWidth: W4       # W1(1,8-bit), W2(2,16-bit), W4(4,32-bit), W8(8,64-bit)
    access: String      # String(Str), Numeric(Num), Both(All)
    initialize: Spaces  # Spaces(Space), Zeroes(Zero)
  fil:
    fill: 0             # \u0000

\end{lstlisting}
%--------1---------2---------3---------4---------5---------6---------7---------8
i valori mostrati nel sorgente~\ref{code:defaults-conf} sono i valori 
preimpostati, se viene omessa la valorizzazione di questa struttura, 
completamente o in parte verranno usati questi valori.
I parametri possono essere impostate anche a livello di singolo campo, le 
impostazioni a livello di campo, hanno la precedenza rispetto ai valori 
impostati in questa sezione. Se i parametri non sono presenti a livello di 
campo, vengono utilizzati i valori impostati in questa sezione.

%--------1---------2---------3---------4---------5---------6---------7---------8
Dopo la configurazione dei default segue un elenco di definizioni per 
\textsl{package} usando la classe del plugin \textsl{ClassPackage}

\begin{lstlisting}[language=java, caption=classe di configurazione ClassPackage, 
label=code:ClassPackage]
@Data
public class ClassPackage {
    private String name;     // package name
    private List<TraitModel> interfaces = new ArrayList<>();
    private List<ClassModel> classes = new ArrayList<>();
}
\end{lstlisting}
%--------1---------2---------3---------4---------5---------6---------7---------8
la classe permette di indicare il nome del package dove devono essere generate
le classi, e permette di definire un elenco di interfacce e di classi; 
espandendo i campi in formato \textit{yaml} abbiamo:

\begin{lstlisting}[language=yaml, caption={configurazione, area packages / interfaces / classes}, 
label=code:pakg-conf]
packages:
  - name: com.example.test  # package name
    interfaces:
      - &IFoo         # interface reference
        name: IFoo    # interface name
        length: 12    # interface length
        fields:
          - ...
    classes:
      - name: Foo           # class name
        length: 10          # class length
        onOverflow: Trunc   # Error, Trunc
        onUnderflow: Pad    # Error, Pad
        fields:
          - ...
\end{lstlisting}
%--------1---------2---------3---------4---------5---------6---------7---------8
se non vengono usate interfacce, il nodo \texttt{interfaces} può essere omesso,
sia per le classi che le interfacce il nome e la lunghezza del tracciato da 
associare devono essere impostate dall'\,utente, nella definizione della classe
può anche essere impostato il comportamento nel caso che venga fornita in fase
di de-serializzazione una struttura con una dimensione maggiore o minore di 
quella attesa.

\begin{table}[!htb]
\centering
\begin{tabular}{|>{\tt}l|>{\tt}c|>{\tt}c|l|}
\hline
\multicolumn{4}{|c|}{TraitModel --- interfaces}\\
\hline
\multicolumn{1}{|c|}{attributo} & \multicolumn{1}{c|}{alt} 
	& \multicolumn{1}{c|}{tipo} & \multicolumn{1}{c|}{note} \\
\hline
\hline
name       &     & String  & obbligatorio \\
\hline
length     & len & int     & obbligatorio \\
\hline
fields     &     & array & obbligatorio \\
\hline
\end{tabular}
%--------1---------2---------3---------4---------5---------6---------7---------8
\caption{Attributi impostabili per la definizione di una interfaccia} 
\label{tab:attr.trait}
\end{table}


%--------1---------2---------3---------4---------5---------6---------7---------8
Anche se tutti i tipi di campo hanno una posizione di inizio e una lunghezza,
il dettaglio dei parametri di configurazione varia da campo a campo ed è 
opportuno introdurre i parametri di configurazione campo per campo.

\begin{table}[!htb]
\centering
\begin{tabular}{|>{\tt}l|>{\tt}c|>{\tt}c|l|}
\hline
\multicolumn{4}{|c|}{ClassModel --- classes}\\
\hline
\multicolumn{1}{|c|}{attributo} & \multicolumn{1}{c|}{alt} 
	& \multicolumn{1}{c|}{tipo} & \multicolumn{1}{c|}{note} \\
\hline
\hline
name       &     & String  & obbligatorio \\
\hline
length     & len & int     & obbligatorio \\
\hline
onOverflow & ovf & enum    & default \texttt{\$\{defaults.cls.onOverflow:Trunc\}}\\
\hline
onUnderlow & unf & enum    & default \texttt{\$\{defaults.cls.onUnderflow:Pad\}}\\
\hline
fields     &     & array & obbligatorio \\
\hline
\end{tabular}
%--------1---------2---------3---------4---------5---------6---------7---------8
\caption{Attributi impostabili per la definizione di una classe} 
\label{tab:attr.class}
\end{table}

\subsection{Campo Alfanumerico}
%--------1---------2---------3---------4---------5---------6---------7---------8
La definizione yaml del campo alfanumerico riflette la struttura imposta dalla
service interface, vedi~\ref{code:AbcModel}

\begin{table}[!htb]
\centering
\begin{tabular}{|>{\tt}l|>{\tt}c|>{\tt}c|l|}
\hline
\multicolumn{4}{|c|}{AbcModel --- \texttt{!Abc}}\\
\hline
\multicolumn{1}{|c|}{attributo} & \multicolumn{1}{c|}{alt} 
	& \multicolumn{1}{c|}{tipo} & \multicolumn{1}{c|}{note} \\
\hline
\hline
offset     & at  & int     & obbligatorio \\
\hline
length     & len & int     & obbligatorio \\
\hline
name       &     & String  & obbligatorio \\
\hline
override   & ovr & boolean & default \texttt{false} \\
\hline
onOverflow & ovf & enum    & default \texttt{\$\{defaults.abc.onOverflow:Trunc\}}\\
\hline
onUnderlow & unf & enum    & default \texttt{\$\{defaults.abc.onUnderflow:Pad\}}\\
\hline
check      & chk & enum    & default \texttt{\$\{defaults.abc.check:Ascii\}}\\
\hline
normalize  & nrm & enum    & default \texttt{\$\{defaults.abc.normalize:None\}}\\
\hline
checkGetter & get & boolean & default \texttt{\$\{defaults.abc.checkGetter:true\}}\\
\hline
checkSetter & set & boolean & default \texttt{\$\{defaults.abc.checkSetter:true\}}\\
\hline
\end{tabular}
\caption{Attributi impostabili per un campo alfanumerico} \label{tab:attr.abc}
\end{table}

\begin{figure*}[!htb]
\begin{lstlisting}[language=yaml, caption={esempio definizione campi alfanumerici}, 
label=code:xmplAbc]
packages:
  - name: com.example.test
    classes:
      - name: Foo
        length: 50
        fields:
          - !Abc { name: cognome    , at:   1, len:    25 }
          - !Abc { name: nome       , at:  26, len:    20 }
          - !Abc { name: stCivile   , at:  46, len:     1 }
          - !Abc { name: nazionalita, at:  47, len:     3 }
          - !Abc { name: sesso      , at:  50, len:     1 }
\end{lstlisting}
\end{figure*}

\subsection{Campo Numerico}
%--------1---------2---------3---------4---------5---------6---------7---------8
La definizione yaml del campo numerico riflette la struttura imposta dalla
service interface, vedi~\ref{code:NumModel}

\begin{table}[!htb]
\centering
\begin{tabular}{|>{\tt}l|>{\tt}c|>{\tt}c|l|}
\hline
\multicolumn{4}{|c|}{NumModel --- \texttt{!Num}}\\
\hline
\multicolumn{1}{|c|}{attributo} & \multicolumn{1}{c|}{alt} 
	& \multicolumn{1}{c|}{tipo} & \multicolumn{1}{c|}{note} \\
\hline
\hline
offset     & at  & int     & obbligatorio \\
\hline
length     & len & int     & obbligatorio \\
\hline
name       &     & String  & obbligatorio \\
\hline
override   & ovr & boolean & default \texttt{false} \\
\hline
onOverflow & ovf & enum    & default \texttt{\$\{defaults.num.onOverflow:Trunc\}}\\
\hline
onUnderlow & unf & enum    & default \texttt{\$\{defaults.num.onUnderflow:Pad\}}\\
\hline
access     & acc & enum    & default \texttt{\$\{defaults.num.access:String\}}\\
\hline
wordWidth  & wid & enum    & default \texttt{\$\{defaults.num.wirdWidth:W4\}}\\
\hline
normalize  & nrm & enum    & default \texttt{\$\{defaults.num.normalize:None\}}\\
\hline
\end{tabular}
\caption{Attributi impostabili per un campo numerico} \label{tab:attr.num}
\end{table}

\subsection{Campo Custom (alfanumerico)}
%--------1---------2---------3---------4---------5---------6---------7---------8
La definizione yaml del campo numerico riflette la struttura imposta dalla
service interface, vedi~\ref{code:CusModel}

\begin{table}[!htb]
\centering
\begin{tabular}{|>{\tt}l|>{\tt}c|>{\tt}c|l|}
\hline
\multicolumn{4}{|c|}{CusModel --- \texttt{!Cus}}\\
\hline
\multicolumn{1}{|c|}{attributo} & \multicolumn{1}{c|}{alt} 
	& \multicolumn{1}{c|}{tipo} & \multicolumn{1}{c|}{note} \\
\hline
\hline
offset     & at  & int     & obbligatorio \\
\hline
length     & len & int     & obbligatorio \\
\hline
name       &     & String  & obbligatorio \\
\hline
override   & ovr & boolean & default \texttt{false} \\
\hline
onOverflow & ovf & enum    & default \texttt{\$\{defaults.cus.onOverflow:Trunc\}}\\
\hline
onUnderlow & unf & enum    & default \texttt{\$\{defaults.cus.onUnderflow:Pad\}}\\
\hline
padChar    & pad & char    & default \texttt{\$\{defaults.cus.pad:' '\}}\\
\hline
initChar   & ini & char    & default \texttt{\$\{defaults.cus.init:' '\}}\\
\hline
check      & chk & enum    & default \texttt{\$\{defaults.cus.check:Ascii\}}\\
\hline
align      &     & enum    & default \texttt{\$\{defaults.cus.align:LFT\}}\\
\hline
normalize  & nrm & enum    & default \texttt{\$\{defaults.cus.normalize:None\}}\\
\hline
regex      &     & String  & opzionale \\
\hline
checkGetter & get & boolean & default \texttt{\$\{defaults.cus.checkGetter:true\}}\\
\hline
checkSetter & set & boolean & default \texttt{\$\{defaults.cus.checkSetter:true\}}\\
\hline
\end{tabular}
\caption{Attributi impostabili per un campo custom} \label{tab:attr.cus}
\end{table}



\subsection{Campo Numerico nullabile}
%--------1---------2---------3---------4---------5---------6---------7---------8
La definizione yaml del campo numerico riflette la struttura imposta dalla
service interface, vedi~\ref{code:NuxModel}

\begin{table}[!htb]
\centering
\begin{tabular}{|>{\tt}l|>{\tt}c|>{\tt}c|l|}
\hline
\multicolumn{4}{|c|}{NuxModel --- \texttt{!Nux}}\\
\hline
\multicolumn{1}{|c|}{attributo} & \multicolumn{1}{c|}{alt} 
	& \multicolumn{1}{c|}{tipo} & \multicolumn{1}{c|}{note} \\
\hline
\hline
offset     & at  & int     & obbligatorio \\
\hline
length     & len & int     & obbligatorio \\
\hline
name       &     & String  & obbligatorio \\
\hline
override   & ovr & boolean & default \texttt{false} \\
\hline
onOverflow & ovf & enum    & default \texttt{\$\{defaults.nux.onOverflow:Trunc\}}\\
\hline
onUnderlow & unf & enum    & default \texttt{\$\{defaults.nux.onUnderflow:Pad\}}\\
\hline
access     & acc & enum    & default \texttt{\$\{defaults.nux.access:String\}}\\
\hline
wordWidth  & wid & enum    & default \texttt{\$\{defaults.nux.wirdWidth:W4\}}\\
\hline
normalize  & nrm & enum    & default \texttt{\$\{defaults.nux.normalize:None\}}\\
\hline
initialize & ini & enum    & default \texttt{\$\{defaults.nux.initialize:Space\}}\\
\hline
\end{tabular}
\caption{Attributi impostabili per un campo numerico nullabile} \label{tab:attr.nux}
\end{table}


\subsection{Campo Dominio}
%--------1---------2---------3---------4---------5---------6---------7---------8
La definizione yaml del campo numerico riflette la struttura imposta dalla
service interface, vedi~\ref{code:DomModel}

\begin{table}[!htb]
\centering
\begin{tabular}{|>{\tt}l|>{\tt}c|>{\tt}c|l|}
\hline
\multicolumn{4}{|c|}{DomModel --- \texttt{!Dom}}\\
\hline
\multicolumn{1}{|c|}{attributo} & \multicolumn{1}{c|}{alt} 
	& \multicolumn{1}{c|}{tipo} & \multicolumn{1}{c|}{note} \\
\hline
\hline
offset     & at  & int     & obbligatorio \\
\hline
length     & len & int     & obbligatorio \\
\hline
name       &     & String  & obbligatorio \\
\hline
override   & ovr & boolean & default \texttt{false} \\
\hline
items      &     & array  & obbligatorio (valori permessi)\\
\hline
\end{tabular}
\caption{Attributi impostabili per un campo dominio} \label{tab:attr.dom}
\end{table}

\subsection{Campo Filler}
%--------1---------2---------3---------4---------5---------6---------7---------8
La definizione yaml del campo numerico riflette la struttura imposta dalla
service interface, vedi~\ref{code:FilModel}

\begin{table}[!htb]
\centering
\begin{tabular}{|>{\tt}l|>{\tt}c|>{\tt}c|l|}
\hline
\multicolumn{4}{|c|}{FilModel --- \texttt{!Fil}}\\
\hline
\multicolumn{1}{|c|}{attributo} & \multicolumn{1}{c|}{alt} 
	& \multicolumn{1}{c|}{tipo} & \multicolumn{1}{c|}{note} \\
\hline
\hline
offset     & at  & int     & obbligatorio \\
\hline
length     & len & int     & obbligatorio \\
\hline
fill       &     & char    & default \texttt{\$\{defaults.fil.fill:0\}}\\
\hline
\end{tabular}
\caption{Attributi impostabili per un campo filler} \label{tab:attr.fill}
\end{table}


\subsection{Campo Costante}
%--------1---------2---------3---------4---------5---------6---------7---------8
La definizione yaml del campo numerico riflette la struttura imposta dalla
service interface, vedi~\ref{code:ValModel}

\begin{table}[!htb]
\centering
\begin{tabular}{|>{\tt}l|>{\tt}c|>{\tt}c|l|}
\hline
\multicolumn{4}{|c|}{ValModel --- \texttt{!Val}}\\
\hline
\multicolumn{1}{|c|}{attributo} & \multicolumn{1}{c|}{alt} 
	& \multicolumn{1}{c|}{tipo} & \multicolumn{1}{c|}{note} \\
\hline
\hline
offset     & at  & int     & obbligatorio \\
\hline
length     & len & int     & obbligatorio \\
\hline
value      & val & string  & obbligatorio \\
\hline
\end{tabular}
\caption{Attributi impostabili per un campo costante} \label{tab:attr.val}
\end{table}


\subsection{Campo Gruppo}
%--------1---------2---------3---------4---------5---------6---------7---------8
La definizione yaml del campo numerico riflette la struttura imposta dalla
service interface, vedi~\ref{code:GrpModel}

\begin{table}[!htb]
\centering
\begin{tabular}{|>{\tt}l|>{\tt}c|>{\tt}c|l|}
\hline
\multicolumn{4}{|c|}{GrpModel --- \texttt{!Grp}}\\
\hline
\multicolumn{1}{|c|}{attributo} & \multicolumn{1}{c|}{alt} 
	& \multicolumn{1}{c|}{tipo} & \multicolumn{1}{c|}{note} \\
\hline
\hline
offset     & at  & int     & obbligatorio \\
\hline
length     & len & int     & obbligatorio \\
\hline
name       &     & String  & obbligatorio \\
\hline
override   & ovr & boolean & default \texttt{false} \\
\hline
fields     &     & array  & obbligatorio (elenco campi)\\
\hline
\end{tabular}
\caption{Attributi impostabili per un gruppo} \label{tab:attr.grp}
\end{table}


\subsection{Campo Gruppo ripetuto}
%--------1---------2---------3---------4---------5---------6---------7---------8
La definizione yaml del campo numerico riflette la struttura imposta dalla
service interface, vedi~\ref{code:OccModel}

\begin{table}[!htb]
\centering
\begin{tabular}{|>{\tt}l|>{\tt}c|>{\tt}c|l|}
\hline
\multicolumn{4}{|c|}{OccModel --- \texttt{!Occ}}\\
\hline
\multicolumn{1}{|c|}{attributo} & \multicolumn{1}{c|}{alt} 
	& \multicolumn{1}{c|}{tipo} & \multicolumn{1}{c|}{note} \\
\hline
\hline
offset     & at  & int     & obbligatorio \\
\hline
length     & len & int     & obbligatorio \\
\hline
name       &     & String  & obbligatorio \\
\hline
override   & ovr & boolean & default \texttt{false} \\
\hline
times      & x   & int     & obbligatorio \\
\hline
fields     &     & array  & obbligatorio (elenco campi)\\
\hline
\end{tabular}
\caption{Attributi impostabili per un gruppo ripetuto} \label{tab:attr.occ}
\end{table}


\subsection{Campo incorporato da interfaccia}
%--------1---------2---------3---------4---------5---------6---------7---------8
La definizione yaml del campo numerico riflette la struttura imposta dalla
service interface, vedi~\ref{code:EmbModel}

\begin{table}[!htb]
\centering
\begin{tabular}{|>{\tt}l|>{\tt}c|>{\tt}c|l|}
\hline
\multicolumn{4}{|c|}{EmbModel --- \texttt{!Emb}}\\
\hline
\multicolumn{1}{|c|}{attributo} & \multicolumn{1}{c|}{alt} 
	& \multicolumn{1}{c|}{tipo} & \multicolumn{1}{c|}{note} \\
\hline
\hline
offset     & at  & int     & obbligatorio \\
\hline
length     & len & int     & obbligatorio \\
\hline
source     & src  & Trait  & obbligatorio (interfaccia)\\
\hline
\end{tabular}
\caption{Attributi impostabili per un elenco di campi importato da una interfaccia}
\label{tab:attr.grp}
\end{table}


\subsection{Campo Gruppo da interfaccia}
%--------1---------2---------3---------4---------5---------6---------7---------8
La definizione yaml del campo numerico riflette la struttura imposta dalla
service interface, vedi~\ref{code:GrpTraitModel}

\begin{table}[!htb]
\centering
\begin{tabular}{|>{\tt}l|>{\tt}c|>{\tt}c|l|}
\hline
\multicolumn{4}{|c|}{GrpTraitModel --- \texttt{!GRP}}\\
\hline
\multicolumn{1}{|c|}{attributo} & \multicolumn{1}{c|}{alt} 
	& \multicolumn{1}{c|}{tipo} & \multicolumn{1}{c|}{note} \\
\hline
\hline
offset     & at  & int     & obbligatorio \\
\hline
length     & len & int     & obbligatorio \\
\hline
name       &     & String  & obbligatorio \\
\hline
override   & ovr & boolean & default \texttt{false} \\
\hline
typedef    & as  & Trait   & obbligatorio (interfaccia)\\
\hline
\end{tabular}
\caption{Attributi impostabili per un gruppo da interfaccia} \label{tab:attr.grpt}
\end{table}


\subsection{Campo Gruppo ripetuto da interfaccia}
%--------1---------2---------3---------4---------5---------6---------7---------8
La definizione yaml del campo numerico riflette la struttura imposta dalla
service interface, vedi~\ref{code:OccTraitModel}

\begin{table}[!htb]
\centering
\begin{tabular}{|>{\tt}l|>{\tt}c|>{\tt}c|l|}
\hline
\multicolumn{4}{|c|}{OccTraitModel --- \texttt{!OCC}}\\
\hline
\multicolumn{1}{|c|}{attributo} & \multicolumn{1}{c|}{alt} 
	& \multicolumn{1}{c|}{tipo} & \multicolumn{1}{c|}{note} \\
\hline
\hline
offset     & at  & int     & obbligatorio \\
\hline
length     & len & int     & obbligatorio \\
\hline
name       &     & String  & obbligatorio \\
\hline
override   & ovr & boolean & default \texttt{false} \\
\hline
times      & x   & int     & obbligatorio \\
\hline
typedef    & as  & Trait   & obbligatorio (interfaccia)\\
\hline
\end{tabular}
\caption{Attributi impostabili per un gruppo ripetuto da interfaccia} \label{tab:attr.grpt}
\end{table}


\chapter{Service Provider}
%--------1---------2---------3---------4---------5---------6---------7---------8
Nei capitoli precedenti abbiamo visto la \textsl{Service Interface}, che 
definisce delle interfacce e delle classi che permettono di definire i 
traccciati, e indicare alcuni comportamenti che dovranno essere usati in fase di
utilizzazione dei tracciati; e alcuni esempi di \textsl{Service Consumer}, che
semplicemente valorizza gli oggetti messi a disposizione della 
\textsl{Service Interface}, ma il vero lavoro di generazione del codice è fatto
dal \textsl{Service Provider}.

%--------1---------2---------3---------4---------5---------6---------7---------8
La struttuta SPI consente di avere codice generato diverso, implementato in modo
diverso, o addirittura generare sorgente in un linguaggio diverso.

%--------1---------2---------3---------4---------5---------6---------7---------8
Qualunque sia il linguaggio generato e il dettaglio della implementazione il
\textsl{Service Provider} dovrà fornire alcune funzionalità generali.

\begin{itemize}
%--------1---------2---------3---------4---------5---------6---------7---------8
\item \textbf{decode}: partendo dalla stringa-dati, deve instanziare la 
    classe-dati;
\item \textbf{setter, getter}: la classe-dati generata deve fornire i metodi di 
    accesso ai singoli campi;
\item \textbf{costruttore vuoto}: la classe-dati può essere instanziata con i 
    valori di default dei campi;
\item \textbf{encode}: la classe-dati può essere serializzata nella 
    stringa-dati.
\end{itemize}
%--------1---------2---------3---------4---------5---------6---------7---------8
Sarebbe gradita anche qualche funzionalià accessoria:
\begin{itemize}
\item \textbf{validate}: validare la stringa-dati prima della 
    deserializzaione, in modo da segnalare tutte le aree che non possono essere
    assegnate ai relativi campi, tipicamente caratteri non numerici in campi di
    tipo numerico;
\item \textbf{cast}: se due stringhe-dati hanno la stessa lunghezza, poter 
    passare da una classe-dati che le rappresenta all'\,altra;
\item \textbf{toString}: fornire un metodo che mostra tutti i valori dei campi 
    che compongono la classe-dati.
\end{itemize}



\section{Generazione sorgente java --- \texttt{java-addon}}
%--------1---------2---------3---------4---------5---------6---------7---------8
Questa implementazione


%\appendix
%\input{appe01.tex}
\printindex

\end{document}
