\chapter{Maven plugin}\label{sec:maven}
%--------1---------2---------3---------4---------5---------6---------7---------8
The \verb!recfm-maven-plugin! maven plugin allows you to create multiple classes 
and interfaces using one or more configuration files in yaml format. 
The external libraries used require java 8, so you need at least java 8 to run 
this plugin.

The plugin expects configuration parameters as parameters:
\begin{elisting}[!htb]
\begin{javacode}
    @Parameter(defaultValue = "${project.build.directory}/generated-sources/recfm",
        property = "maven.recfm.generateDirectory", required = true)
    private File generateDirectory;

    @Parameter(defaultValue = "${project.build.resources[0].directory}",
        property = "maven.recfm.settingsDirectory", required = true)
    private File settingsDirectory;
    
    @Parameter(required = true)
    private String[] settings;

    @Parameter(defaultValue = "true", property = "maven.recfm.addCompileSourceRoot")
    private boolean addCompileSourceRoot = true;

    @Parameter(defaultValue = "false", property = "maven.recfm.addTestCompileSourceRoot")
    private boolean addTestCompileSourceRoot = false;
    
    @Parameter
    private String codeProviderClassName;
\end{javacode}
\caption{settable parameters of the maven plugin}
\label{lst:maven.conf}
\end{elisting}

\begin{itemize}
%--------1---------2---------3---------4---------5---------6---------7---------8
\item \index{plugin!generateDirectory}
    The \fcolorbox{black}{yellow!75}{\texttt{generateDirectory}} field indicates 
    the root directory to use for generating the sources, it is used to set the 
    value of the \verb!sourceDirectory! field of the 
    \hyperref[lst:GenerateArgs]{\texttt{GenerateArgs}} class, as can be seen 
    from the definition, if the parameter is omitted the 
    \verb!target/generated-sources/recfm! directory is used, normally the 
    default value can be left. 
    The other three parameters of \texttt{GenerateArgs} are an identifier of the 
    \textsl{service} program and are set automatically.
%--------1---------2---------3---------4---------5---------6---------7---------8
\item \index{plugin!settingsDirectory}
    The \fcolorbox{black}{yellow!75}{\texttt{settingsDirectory}} field indicates 
    the directory containing the configuration files, if the parameter is 
    omitted the value \verb!src/main/resources! is used, normally the default 
    value can be left.
%--------1---------2---------3---------4---------5---------6---------7---------8
\item \index{plugin!settings}
    The \fcolorbox{black}{yellow!75}{\texttt{settings}} field indicates the list 
    of configuration files to be used to generate the classes/interfaces; the 
    parameter must be supplied to the plugin.
%--------1---------2---------3---------4---------5---------6---------7---------8
\item \index{plugin!addCompileSourceRoot}
    The \fcolorbox{black}{yellow!75}{\texttt{addCompileSourceRoot}} field is a 
    technical field, it indicates to maven that the directory where the sources 
    were generated must be included among those used for the main compilation, 
    if the parameter is omitted the value \verb!true! is used; the \verb!true! 
    value is appropriate when a code generation directory other than 
    \verb!src/main/java! is used, otherwise it is necessary to use additional 
    plugins to add the new path to the maven compilation path.
%--------1---------2---------3---------4---------5---------6---------7---------8
\item \index{plugin!addTestCompileSourceRoot}
    The \fcolorbox{black}{yellow!75}{\texttt{addTestCompileSourceRoot}} field is 
    similar to the previous field, but adds the generation directory to the test 
    compilation path, if omitted the value \verb!false! is used, except in 
    special cases the default value can be left.
%--------1---------2---------3---------4---------5---------6---------7---------8
\item \index{plugin!codeProviderClassName}
    The \fcolorbox{black}{yellow!75}{\texttt{codeProviderClassName}} field 
    indicates which is the concreated class that implements the 
    \textsl{Service Interface}, if omitted the ``first'' implementation of the 
    \textsl{ServiceLoader} retrieved is used; if there is only one 
    implementation in the classpath, it is not necessary to set the parameter.
\end{itemize}
%--------1---------2---------3---------4---------5---------6---------7---------8

%--------1---------2---------3---------4---------5---------6---------7---------8
\begin{elisting}[!htb]
\begin{xmlcode}
<plugin>
    <groupId>io.github.epi155</groupId>
    <artifactId>recfm-maven-plugin</artifactId>
    <version>0.7.0</version>
    <executions>
        <execution>
            <goals>
                <goal>generate</goal>
            </goals>
            <configuration>
                <settings>
                    <setting>foo.yaml</setting>
                </settings>
            </configuration>
        </execution>
    </executions>
    <dependencies>
        <dependency>
            <groupId>io.github.epi155</groupId>
            <artifactId>recfm-java-addon</artifactId>
            <version>0.7.0</version>
        </dependency>
    </dependencies>
</plugin>
\end{xmlcode}
\caption{minimal example of plugin execution}
\label{lst:mvn-xmpl}
\end{elisting}
%--------1---------2---------3---------4---------5---------6---------7---------8
An example of plugin execution is shown in the lst.~\ref{lst:mvn-xmpl}, the 
plugin to be executed must have as dependency a library that provides the 
implementation of the interface, otherwise the \verb!ServiceLoader! does not 
find anything and the plugin terminates in error.

%--------1---------2---------3---------4---------5---------6---------7---------8
All other parameters are provided in the configuration files.
