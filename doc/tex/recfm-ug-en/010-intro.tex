\section*{Introduction}
%--------1---------2---------3---------4---------5---------6---------7---------8
Sometimes it may happen that you have to deal with positional files (or memory 
areas), see fig.~\ref{fig:str.data}, in these cases you need to waste a lot of 
time to make a class dedicated to each string-data with the setters and getters
to read and write values\footnote{%
there is \texttt{com.ancientprogramming.fixedformat4j:fixedformat4j} 
which provides some basic functionality, but in most situations it is not 
flexible enough.}.

\begin{figure}[!htb]
\centering\small
\scalebox{0.7}[1]{
\texttt{%
\fbox[lb]{S}%01
\fbox[lb]{C}%02
\fbox[lb]{A}%03
\fbox[lb]{R}%04
\fbox[lb]{L}%05
\fbox[lb]{E}%06
\fbox[lb]{T}%07
\fbox[lb]{T}%08
\fbox[lb]{ }%09
\fbox[lb]{ }%10
\fbox[lb]{ }%11
\fbox[lb]{ }%12
\fbox[lb]{ }%13
\fbox[lb]{ }%14
\fbox[lb]{ }%15
\fbox[lb]{J}%16
\fbox[lb]{O}%17
\fbox[lb]{H}%18
\fbox[lb]{A}%19
\fbox[lb]{N}%20
\fbox[lb]{S}%21
\fbox[lb]{S}%22
\fbox[lb]{O}%23
\fbox[lb]{N}%24
\fbox[lb]{ }%25
\fbox[lb]{ }%26
\fbox[lb]{ }%27
\fbox[lb]{ }%28
\fbox[lb]{ }%29
\fbox[lb]{ }%30
\fbox[lb]{1}%31
\fbox[lb]{9}%32
\fbox[lb]{8}%33
\fbox[lb]{4}%34
\fbox[lb]{1}%35
\fbox[lb]{1}%36
\fbox[lb]{2}%37
\fbox[lb]{2}%38
\fbox[lb]{N}%39
\fbox[lb]{E}%40
\fbox[lb]{W}%31
\fbox[lb]{ }%32
\fbox[lb]{Y}%33
\fbox[lb]{O}%34
\fbox[lb]{R}%35
\fbox[lb]{K}%36
\fbox[lb]{ }%37
\fbox[lb]{ }%38
\fbox[lb]{ }%39
\fbox[lb]{ }%40
\fbox[lb]{ }%41
\fbox[lb]{ }%42
\fbox[lb]{U}%43
\fbox[lb]{S}%44
\fbox[lbr]{A}%45
}}

\iffalse
\scalebox{0.7}[1]{
\texttt{%
\fbox[lb]{O}%01
\fbox[lb]{L}%02
\fbox[lb]{G}%03
\fbox[lb]{A}%04
\fbox[lb]{ }%05
\fbox[lb]{ }%06
\fbox[lb]{ }%07
\fbox[lb]{ }%08
\fbox[lb]{ }%09
\fbox[lb]{ }%10
\fbox[lb]{ }%11
\fbox[lb]{ }%12
\fbox[lb]{ }%13
\fbox[lb]{ }%14
\fbox[lb]{ }%15
\fbox[lb]{K}%16
\fbox[lb]{U}%17
\fbox[lb]{R}%18
\fbox[lb]{Y}%19
\fbox[lb]{L}%20
\fbox[lb]{E}%21
\fbox[lb]{N}%22
\fbox[lb]{K}%23
\fbox[lb]{O}%24
\fbox[lb]{ }%25
\fbox[lb]{ }%26
\fbox[lb]{ }%27
\fbox[lb]{ }%28
\fbox[lb]{ }%29
\fbox[lb]{ }%30
\fbox[lb]{1}%31
\fbox[lb]{9}%32
\fbox[lb]{7}%33
\fbox[lb]{9}%34
\fbox[lb]{1}%35
\fbox[lb]{1}%36
\fbox[lb]{1}%37
\fbox[lb]{4}%38
\fbox[lb]{B}%39
\fbox[lb]{E}%40
\fbox[lb]{R}%31
\fbox[lb]{D}%32
\fbox[lb]{J}%33
\fbox[lb]{A}%34
\fbox[lb]{N}%35
\fbox[lb]{S}%36
\fbox[lb]{'}%37
\fbox[lb]{K}%38
\fbox[lb]{ }%39
\fbox[lb]{ }%40
\fbox[lb]{ }%41
\fbox[lb]{ }%42
\fbox[lb]{U}%43
\fbox[lb]{K}%44
\fbox[lbr]{R}%45
}}
\fi

\scalebox{0.7}[1]{
\texttt{%
\fbox[lb]{A}%01
\fbox[lb]{N}%02
\fbox[lb]{A}%03
\fbox[lb]{ }%04
\fbox[lb]{ }%05
\fbox[lb]{ }%06
\fbox[lb]{ }%07
\fbox[lb]{ }%08
\fbox[lb]{ }%09
\fbox[lb]{ }%10
\fbox[lb]{ }%11
\fbox[lb]{ }%12
\fbox[lb]{ }%13
\fbox[lb]{ }%14
\fbox[lb]{ }%15
\fbox[lb]{D}%16
\fbox[lb]{E}%17
\fbox[lb]{ }%18
\fbox[lb]{A}%19
\fbox[lb]{R}%20
\fbox[lb]{M}%21
\fbox[lb]{A}%22
\fbox[lb]{S}%23
\fbox[lb]{ }%24
\fbox[lb]{ }%25
\fbox[lb]{ }%26
\fbox[lb]{ }%27
\fbox[lb]{ }%28
\fbox[lb]{ }%29
\fbox[lb]{ }%30
\fbox[lb]{1}%31
\fbox[lb]{9}%32
\fbox[lb]{8}%33
\fbox[lb]{8}%34
\fbox[lb]{0}%35
\fbox[lb]{4}%36
\fbox[lb]{3}%37
\fbox[lb]{0}%38
\fbox[lb]{H}%39
\fbox[lb]{A}%40
\fbox[lb]{V}%31
\fbox[lb]{A}%32
\fbox[lb]{N}%33
\fbox[lb]{A}%34
\fbox[lb]{ }%35
\fbox[lb]{ }%36
\fbox[lb]{ }%37
\fbox[lb]{ }%38
\fbox[lb]{ }%39
\fbox[lb]{ }%40
\fbox[lb]{ }%41
\fbox[lb]{ }%42
\fbox[lb]{C}%43
\fbox[lb]{U}%44
\fbox[lbr]{B}%45
}}

\caption{Positional data-file example} 
\label{fig:str.data}
\end{figure}



%--------1---------2---------3---------4---------5---------6---------7---------8
This group of programs aims to minimize the time to create these classes. 
In practice, the structure of the data-string is defined with a configuration 
file, this is fed to a plugin that generates the corresponding data-class, 
which can be used without anything else user intervention.

%--------1---------2---------3---------4---------5---------6---------7---------8
Programs are structured using service provider interface, see 
fig.~\ref{fig:spi}, we have a plugin, or user program (\textsl{Service}), which 
directly sees the classes defined in the \textsl{Service Provider Interface} and 
retrieves them the implementation using the \textsl{ServiceLoader}, this way it 
doesn't have a specific dependency with one of the the implementations used. 
The \textsl{Service Provider} must implement the classes defined in the 
\textsl{Service Provider Interface}.

\begin{figure}[!htb]
\centering
\begin{tikzpicture}[>=latex,font={\sf}]
\node(u1) at (0,1.5) [manual input,text width=2cm,fill=blue!10]{maven plugin};
\node(u2) at (3,1.5) [manual input,text width=2cm,fill=blue!10]{gradle plugin};
\node(u3) at (6,1.5) [manual input,text width=2cm,fill=blue!10]{custom service};
\node(si) at (3,0.0) [preparation,fill=yellow!20]{addon-api};
\node(a1) at (0,-1.5) [process,text width=1.7cm,fill=green!10]{java addon};
\node(a2) at (3,-1.5) [process,text width=1.7cm,fill=green!10]{scala addon};
\node(a3) at (6,-1.5) [process,text width=1.7cm,fill=green!10]{custom provider};

\node at (9,1.5) {Service};
\node at (9,0.0) {Service Provider Interface};
\node at (9,-1.5) {Service Provider};

\draw[arrow] (u1) -- (si.north);
\draw[arrow] (u2) -- (si.north);
\draw[arrow] (u3) -- (si.north);
\draw[arrow] (a1) -- (si.south);
\draw[arrow] (a2) -- (si.south);
\draw[arrow] (a3) -- (si.south);

\end{tikzpicture}
\caption{Structure service, service-provider-interface, service-provider} 
\label{fig:spi}
\end{figure}

%--------1---------2---------3---------4---------5---------6---------7---------8
If the \verb!maven-plugin! finds the library with the \verb!java-addon! 
implementation running it will generate the sources in java, but if it finds 
the \verb!scala-addon! implementation it will generate the scala sources.

%--------1---------2---------3---------4---------5---------6---------7---------8
The documentation is divided into three parts. In the first, \ref{vol:spi}, a 
detailed description of the classes is given defined in the 
\textsl{service provider interface}, this part is useful for those wishing to 
develop a \textit{custom service} or a \textit{custom provider}. 
If you are only interested in how to generate code starting from config files 
it can be skipped.

%--------1---------2---------3---------4---------5---------6---------7---------8
In the second part, \ref{vol:srv}, a description of two plugins used to generate 
the code is given. In particular how to define the layouts with the 
configuration files and how to activate the plugin.

%--------1---------2---------3---------4---------5---------6---------7---------8
In the third part, \ref{vol:sp}, a description of the \textsl{service provider} 
that generates the java source showing also some additional features of 
generated classes beyond simple setters and getters.

%\input{cover.tex}

\clearpage
