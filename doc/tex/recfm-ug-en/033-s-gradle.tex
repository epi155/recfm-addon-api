\chapter{Gradle plugin}\label{sec:gradle}
%--------1---------2---------3---------4---------5---------6---------7---------8
The gradle plugin \verb!recfm-gradle-plugin! allows you to create multiple 
classes and interfaces using one or more configuration files in yaml format. 
The external libraries used require java 8, so you need at least java 8 to run 
this plugin.

%--------1---------2---------3---------4---------5---------6---------7---------8
The \verb!recfm-gradle-plugin! gradle plugin is simply an adaptation of the 
\verb!recfm-maven-plugin! for use in a gradle project. 
Apart from how the input parameters are passed, and how the plugin is activated, 
the code is the copy of the maven version.

\begin{elisting}[!htb]
\begin{javacode}
@Data
public class RecordFormatExtension {
    private String generateDirectory; // default: "${project.buildDir}/generated-sources/recfm"
    private String settingsDirectory; // default: "${project.projectDir}/src/main/resources"
    private String[] settings;
    private boolean addCompileSourceRoot = true;
    private boolean addTestCompileSourceRoot = false;
    private String codeProviderClassName;
}
\end{javacode}
\caption{settable parameters of the gradle plugin}
\label{lst:gradle.conf}
\end{elisting}
%--------1---------2---------3---------4---------5---------6---------7---------8
As can be seen by comparing it with the lst.~\ref{lst:maven.conf} the parameters 
used are the same as those of the maven-plugin, for the description of the 
parameters refer to \S~\ref{sec:maven}.

\begin{elisting}[!htb]
\begin{minted}[frame=single,framesep=2mm,bgcolor=bg,fontsize=\footnotesize]{groovy}
buildscript {
    dependencies {
        // plugin
        classpath 'io.github.epi155:recfm-gradle-plugin:0.7.0'
        // addon for java code generation
        classpath 'io.github.epi155:recfm-java-addon:0.7.0'
    }
}
apply plugin: 'io.github.epi155.recfm-gradle-plugin'
recfm {
    settings = [ 'refcm-foo.yaml', 'refcm-bar.yaml' ]
}
\end{minted}
\caption{minimal example of plugin execution}
\label{lst:grd-xmpl}
\end{elisting}
