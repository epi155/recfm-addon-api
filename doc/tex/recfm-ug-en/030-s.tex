%  ____                  _          
% / ___|  ___ _ ____   _(_) ___ ___ 
% \___ \ / _ \ '__\ \ / / |/ __/ _ \
%  ___) |  __/ |   \ V /| | (_|  __/
% |____/ \___|_|    \_/ |_|\___\___|
%                                  

\chapter*{Service}
%--------1---------2---------3---------4---------5---------6---------7---------8
The \textsl{Service Provider Interface} simply fixes the general structure, but 
contains only interfaces and java-beans.
The \textsl{Service} is the application that allows the user to provide the 
input required by the \textsl{Service Provider Interface}. 
To produce the output, the \textsl{Service} will use the \textsl{ServiceLoader} 
to search the classpath for a \textsl{Service Provider} that implements the 
\textsl{Service Provider Interface}, and the \textsl{Service Provider} will 
generate the output from the input.

%--------1---------2---------3---------4---------5---------6---------7---------8
The heart of the \textsl{Service Provider Interface}, \verb!recfm-addon-api!, is 
the \textsl{CodeProvider} interface. The implementation of the interface is 
searched with the ServiceLoader mechanism, lst.~\ref{lst:getCodeProvider}.

\begin{elisting}[!htb]
\begin{javacode}
        ServiceLoader<CodeProvider> loader = ServiceLoader.load(CodeProvider.class);
        |\hyperref[lst:CodeProvider]{CodeProvider}| codeProvider = loader.iterator().next();
        |\hyperref[lst:CodeFactory]{CodeFactory}| factory = codeProvider.getInstance();
\end{javacode}
\caption{retrieve of the CodeProvider}
\label{lst:getCodeProvider}
\end{elisting}

%--------1---------2---------3---------4---------5---------6---------7---------8
Once an instance of \textsl{CodeFactory} has been retrieved, it is possible to 
create the definitions of the data-strings and generate the sources of the 
data-classes.

%--------1---------2---------3---------4---------5---------6---------7---------8
Two \textsl{clients} have been developed, one in the form of a maven plugin 
(\verb!recfm-maven-plugin!), and the other in the form of a gradle plugin 
(\verb!recfm-gradle-plugin!). 
The code is largely identical, only the trigger mechanism changes.
