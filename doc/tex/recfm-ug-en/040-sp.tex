\chapter{Service Provider}
%--------1---------2---------3---------4---------5---------6---------7---------8
In the previous chapters we have seen the \textsl{Service Provider Interface}, 
which defines interfaces and classes that allow you to define the layouts, and 
indicate some behaviors that must be used when using the layouts; 
and some examples of \textsl{Service}, which simply values the objects made 
available to the \textsl{Service Provider Interface}, but the real work of 
generating the code is done by the \textsl{Service Provider}.

%--------1---------2---------3---------4---------5---------6---------7---------8
The SPI structure allows you to have code generated differently, implemented 
differently, or even generate source in a different language.

%--------1---------2---------3---------4---------5---------6---------7---------8
Whatever the language generated and the detail of the implementation, the 
\textsl{Service Provider} will have to provide some general functions.

\begin{itemize}\setlength\itemsep{-0.5ex}
%--------1---------2---------3---------4---------5---------6---------7---------8
\item \textbf{decode}: starting from the data-string, it must instantiate the 
    data-class;
\item \textbf{setter, getter}: the generated data-class must provide access 
    methods to the individual fields;
\item \textbf{empty constructor}: the data-class can be instantiated with the 
    default values of the fields;
\item \textbf{encode}: data-class can be serialized into data-string.
\end{itemize}
%--------1---------2---------3---------4---------5---------6---------7---------8
Some additional features would also be welcome:
\begin{itemize}\setlength\itemsep{-0.5ex}
\item \textbf{validate}: validate the data-string before de-serialization, in 
    order to signal all the areas that cannot be assigned to the relative 
    fields, typically non-numerical characters in numeric-type fields;
\item \textbf{cast}: if two data-strings have the same length, being able to 
    pass from one data-class that represents them to another;
\item \textbf{toString}: provide a method that displays all the values of the 
    fields that make up the data-class (in a human readable way, not the encode 
    method);
\item (deep) \textbf{copy}: generates a copy of the data-class;.
\end{itemize}
